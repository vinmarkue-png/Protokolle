\documentclass[a4paper,12pt,bibliography=totocnumbered]{scrartcl}

\usepackage[utf8]{inputenc} 
\usepackage[T1]{fontenc}
\usepackage[ngerman]{babel} 
\usepackage{amsmath, amssymb,amsfonts}
\usepackage{csquotes}
\usepackage{graphicx}
\usepackage[bookmarks,colorlinks=true]{hyperref}
\usepackage{geometry}
\usepackage{float}
\usepackage[final]{pdfpages}
\usepackage{framed, color} 
\usepackage{scrlayer-scrpage}
\usepackage{siunitx}
\usepackage{subfigure}
\renewcaptionname{ngerman}{\figurename}{Abb.}
\sisetup{detect-weight=true, detect-family=true,locale=DE,range-phrase={\,bis\,},list-final-separator ={\,\linebreak[0] \text{und}\,},separate-uncertainty=true,per-mode = symbol-or-fraction}
\DeclareSIUnit\curie{Ci}
\usepackage[backend=biber, style=chem-angew]{biblatex} 
\addbibresource{lit.bib} 

\usepackage{chemgreek}
\usepackage{chemformula}
\usepackage{chemscheme}

\urlstyle{same}
%Hyperlinks-Setup
\hypersetup{
	colorlinks,
	linktocpage,
	citecolor=black,
	filecolor=black,
	linkcolor=black,
	urlcolor=black
}

%\numberwithin{equation}{section}

\setlength{\parindent}{0 mm}
\setlength{\parskip}{2 mm} 



\pagestyle{scrheadings}
\ohead{\empty} %Header oben links auf linker Seite (ungerade Seitenzahl) und oben rechts auf rechter Seite (gerade Seitenzahl), beinhaltet gruppennummer und Versuchskürzel. Im Fall eine einseitigen Dokuments: Header oben rechts
\ihead{\VerfasserEINS\;\&\;\VerfasserZWEI} %Header oben rechts auf linker Seite und oben links auf rechter Seite. Beinhaltet die Namen der Verfasser. Im Fall eine einseitigen Dokuments: Header oben links!
\ofoot{\thepage} %Footer unten links auf linker und unten rechts auf rechter Seite, enthält die jeweilige Seitenzahl. Im Fall eines einseitigen Elements: Footer unten rechts!
\cfoot{\empty} %Mittig unten im Footer soll nichts eingetragen werden 
\ifoot{\empty} %Footer unten rechts auf linker und unten links auf rechter Seite. Hier ebenfalls leer.


\newcommand{\VERSUCHSDATUM}{29.10.2025 - 19.11.2025}
\newcommand{\PROTOKOLLDATUM}{\today}

\newcommand{\VerfasserEINS}{Vincent Kümmerle}
\newcommand{\MatNoEINS}{3712667}
\newcommand{\EmailEINS}{st187541@stud.uni-stuttgart.de}
\newcommand{\StudiengangEINS}{B.Sc. Chemie}

\newcommand{\VerfasserZWEI}{Elvis Gnaglo}
\newcommand{\MatNoZWEI}{3710504}
\newcommand{\EmailZWEI}{st189318@stud.uni-stuttgart.de}
\newcommand{\StudiengangZWEI}{B.Sc. Chemie}

\newcommand{\BETREUER}{Manuel Pech}
\newcommand{\GRUPPENNR}{A05}

\newcommand{\VERSUCHSNR}{Komplex Präparat}
\newcommand{\VERSUCHSNAME}{Synthese und Phosphoreszenz von \ch{[Re(CO)3Cl(N2Tz)]}}


\begin{document}
\thispagestyle{empty}


\begin{titlepage}

\begin{center}
\Huge{\textbf{\VERSUCHSNR\ - \\ \VERSUCHSNAME}}\\
\vspace{10mm}% Abstand
\Large{Protokoll zum Versuch des AC2 Praktikums von \\ \textbf{\VerfasserEINS\;\& \VerfasserZWEI}}\\
\vspace{10mm} 
\Large{Universität Stuttgart}\\
\end{center}
\vspace{1cm}
\begin{center}
\begin{tabular}{ll}
\large{Verfasser:}		& \large{\VerfasserEINS,} \large{\MatNoEINS} \\
 						& \large{\EmailEINS} \\
 						\vspace{0cm}\\
						& \large{\VerfasserZWEI,} \large{\MatNoZWEI} \\
                        & \large{\EmailZWEI} \\
						\vspace{0cm}\\
\large{Gruppennummer:}	& \large{\GRUPPENNR} \\
\vspace{0cm}\\
\large{Versuchszeitraum:}	& \large{\VERSUCHSDATUM} \\
\vspace{0cm}\\
\large{Assistent:}		& \large{\BETREUER} \\
\vspace{0cm}\\
\large{Abgabenummer:} & \large{1. Abgabe}
\end{tabular}
\end{center}
\vspace{15mm}

\begin{center}
Stuttgart, den \PROTOKOLLDATUM
\end{center}

\end{titlepage}


\thispagestyle{empty}

\tableofcontents 

\clearpage

\renewcommand{\thepage}{\arabic{page}}
\setcounter{page}{1}

\section{Einleitung}
Tetrazine als Klasse heterocyclischer Liganden sind bereits seit der Synthese von A. Pinner 1893 bekannt\supercite{Pinner} und sind aufgrund ihrer starken $\pi$-Akzeptorfunktion von Bedeutung für die Koordination an Übergangsmetallkomplexe und Klick-Reaktionen.
Durch die vier elektronegativen Stickstoffatome weisen Tetrazine ein sehr elektronenarmes aromatisches System mit energetisch sehr tief liegenden $\pi^*$-Molekülorbitalen auf, die energiearme n $\rightarrow \pi^*$ Übergänge mit charakteristischer Absorption ermöglichen.\supercite{Dissertation}
Deshalb sind Tetrazine sehr leicht reduzierbar und erscheinen sehr oft pink. 
Tetrazine können durch die Interaktion der tief liegenden $\pi^*$-Molekülorbitale (LUMOs) mit den HOMOs (Lowest/Highest Occupied Molecular Orbitals) eines Diens Diels-Alder-Reaktionen mit inversem Elektronenbedarf (invers Electron Demand Diels Alder, iEDDA) eingehen, die auch als Klick-Reaktionen bezeichnet werden.
Zudem sind die $\pi^*$-MOs Zielorbitale für Metal-Ligand-Charge-Transfers (MLCT).
Durch die Koordination von Tetrazin-Derivaten an Übergangsmetalle wie Rhenium bilden sich Komplexe mit höherer Komplexstabilität, da die Derivate als mehrzähnige Liganden Chelat-Komplexe ausbilden.
Häufig wird hierfür Pentacarbonylrhenium(I)chlorid verwendet, in dem die Carbonyl-Liganden als $\pi$-Akzeptoren und der Chlorid-Ligand als $\sigma$-Donor oktaedrisch um das $d_6$-System des Rheniums koordiniert sind.
Die hierbei entstehenden Komplexe der Art \ch{[Re(CO)3Cl(L_x)]} finden Anwendung als Photosensibilisatoren, Katalysatoren für die elektrochemische Reduktion von \ch{CO2} zu \ch{CO} und in der Biodiagnostik, da sie bioorthogonal sind und damit $in~vivo$ genutzt werden können. \\
Somit ist das Ziel des Versuchs im ersten Teil 3-Methyl-6-(pyrimidin-2-yl)-1,2,4,5-tetrazin (N$_2$-Tz) und ausgehend davon \ch{[Re(CO)3Cl(N2Tz)]} zu synthetisieren. 
Im zweiten Teil wird eine iEDDA-Reaktion mit \ch{[Re(CO)3Cl(N2Tz)]} durchgeführt und die Phosphoreszenz des $fac$-\ch{Re(CO)3Cl}-Komplexes untersucht.

\subsection{Syntheseweg}
Die erste Stufe der Synthese von \ch{[Re(CO)3Cl(N2Tz)]} wurde nach der Vorschrift von Schnierle et al. durch eine modifizierte Pinner Synthese durchgeführt.\supercite{modPinner}$^,$\supercite{Synthese}
Wie der Syntheseweg in \autoref{fig: Stufe1} zeigt, wird elementarer Schwefel als aktivierende Spezies verwendet, um die Nucleophilie des Hydraziniumhydroxids zu erhöhen und somit den Angriff des Stickstoff-Elektronenpaars an der Carbonitrilgruppe zu erleichtern.
Durch einen weiteren nucleophilen Angriff an der Nitrilgruppe des Acetonitrils wird das Sechsring-System des Dihydrotetrazins gebildet.

\begin{figure}[H]
    \centering
    \includegraphics[width=1\linewidth]{Bilder/Stufe 1&2.png}
    \caption{Syntheseweg der Synthese von 3-Methyl-6-(pyrimidin-2-yl)-1,2,4,5-tetrazin (N$_2$-Tz) ausgehend von 2-Cyanopyrimidin.\supercite{Synthese}}
    \label{fig: Stufe1}
\end{figure}
In der zweiten Stufe wurde das Dihydrotetrazin im sauren Milieu mit \ch{NaNO2} oxidiert und nach Neutralisation mit Chloroform extrahiert, um das 3-Methyl-6-(pyrimidin-2-yl)-1,2,4,5-tetrazin (N$_2$-Tz) zu synthetisieren.
Die dritte und vierte Stufe wurden nach dem Syntheseweg, der in \autoref{fig: Stufe4} abgebildet ist, durchgeführt.

\begin{figure}[H]
    \centering
    \includegraphics[width=1\linewidth]{Bilder/Stufe4.png}
    \caption{Syntheseweg der Synthese von [\ch{Re(CO)3Cl}(4,5-BCN-3-(2-pyrimidyl)-\\6-methylpyridazin)] ausgehend von N$_2$-Tz und \ch{[Re(MeCN)2(CO)3Cl]}.\supercite{Manu}}
    \label{fig: Stufe4}
\end{figure}
Dabei wurde zuerst das N$_2$-Tz mit \ch{[Re(MeCN)2(CO)3Cl]} an Rhenium komplexiert und somit \ch{[Re(CO)3Cl(N2Tz)]} synthetisiert.
In der vierten Stufe wurde aus dem Rhenium-Komplex und Bicyclononin (BCN) über eine iEDDA-Reaktion [\ch{Re(CO)3Cl}-(4,5-BCN-3-(2-pyrimidyl)-6-methylpyridazin)] synthetisiert.


\section{Ergebnisse und Diskussion}

\subsection{$^{1}$H-NMR}
Die \autoref{fig: TzNMR} zeigt das $\mathrm{^{1}H}$-NMR Spektrum des synthetisierten 3-Methyl-6-(pyrimidin-2-yl)-1,2,4,5-tetrazins.

\begin{figure}[H]
    \centering
    \includegraphics[width=1\textwidth]{Bilder/1H_Tz.2.png}
    \caption{In \ch{CDCl3} aufgenommenes $\mathrm{^{1}H}$-NMR Spektrum von 3-Methyl-6-(pyrimidin-2-yl)-1,2,4,5-tetrazin (N$_2$-Tz).}
    \label{fig: TzNMR}
\end{figure}
Ausgehend von der in \autoref{fig: Tz} gezeigten Struktur des Tetrazins lassen sich die Signale wie folgt zuordnen:

\begin{figure}[H]
    \centering
    \includegraphics[width=0.27\textwidth]{Bilder/Tz.png}
    \caption{Struktur des 3-Methyl-6-(pyrimidin-2-yl)-1,2,4,5-tetrazin (N$_2$-Tz) mit chemisch inäquivalenten Wasserstoffatomen.}
    \label{fig: Tz}
\end{figure}

Das Dublett bei 9,11 ppm mit einem Integral von 2H lässt sich den beiden aromatischen Wasserstoffatomen H$\mathrm{_B}$ im Pyrimidinring an Position 4 und 6 zuordnen, da diese über eine $^3 J_{\mathrm{H-H}}$-Kopplung zum Wasserstoffatom H$\mathrm{_A}$ an Position 5 koppeln ($^3 J_{\mathrm{H-H}}=$5 Hz).
Das Triplett bei 7,60 ppm (1H) kann dementsprechend dem Wasserstoffatom H$\mathrm{_A}$ an Position 5 im Pyrimidinring zugeordnet werden, da es über eine $^3 J_{\mathrm{H-H}}$-Kopplung von 5 Hz zu den anderen beiden Wasserstoffatomen koppelt.
Das Singulett bei 3,22 ppm mit einem Integral von 3H lässt sich den Wasserstoffatomen der Methylgruppe an Position 3 im Tetrazinring zuweisen.
Die Tieffeldverschiebung der Signal befindet sich im Bereich der Literaturwerte, wobei I. Löw in \ch{CD3CN} folgende Werte gemessen hat:
$\mathit{\delta}\,[\mathrm{ppm}] = 
9.08\,(\mathrm{d},\, 2\mathrm{H},\, \mathrm{C{-}}\mathit{H{-}4,6},\, J = 5.0\,\mathrm{Hz}),\;
7.66\,(\mathrm{t},\, 1\mathrm{H},\, \mathrm{C{-}}\mathit{H{-5}},\, J = 5.0\,\mathrm{Hz}),\;
3.09\,(\mathrm{s},\, 3\mathrm{H},\, \mathrm{C}\mathit{H}_3).$\supercite{Dissertation} \\
Die weiteren Signale in \autoref{fig: TzNMR} stammen von anderen Produkten der Reaktion wie dem 3,6-Dimethyltetrazin, die nicht durch Säulenchromatographie abgetrennt werden konnten, sowie von Verunreinigungen.
So ist das Signal bei 2,1 ppm auf Acetonitril und das bei 0,0 ppm auf Polysiloxan Schlifffett zurückzuführen.\\

Die \autoref{fig: ReNMR} zeigt das $\mathrm{^{1}H}$-NMR Spektrum des synthetisierten \ch{[Re(CO)3Cl(N2Tz)]}-Komplexes.

\begin{figure}[H]
    \centering
    \includegraphics[scale = 0.1]{Bilder/1H_Re-Tz.png}
    \caption{Aufgenommenes $\mathrm{^{1}H}$-NMR Spektrum von \ch{[Re(CO)3Cl(N2Tz)]} in \ch{CDCl3}.}
    \label{fig: ReNMR}
\end{figure}
Dabei haben sich im Vergleich zu \autoref{fig: TzNMR} die Signale des Tripletts (1H) und des Singuletts (3H) hochfeldverschoben, während sich das Signal des Dubletts leicht tieffeldverschoben hat.
Der Grund dafür ist, dass durch das Entfernen der 2 Stickstoffatome das $\pi$-System des Komplexes geschwächt wird, wodurch das LUMO angehoben wird. 
Das Dublett mit einem Integral von 2H liegt nun bei 9,13 ppm.
Das Triplett mit dem Integral von 1H liegt nun bei 7,50 ppm und das Singulett mit dem Integral von 3H liegt bei 2,78 ppm.
Dabei ist zu beachten, dass die Probe sehr stark verunreinigt ist.
Dies lässt sich dadurch erklären, dass bei der Komplexierung andere Komplexe als \ch{[Re(CO)3Cl(N2Tz)]} entstanden sind, die nicht durch die Säulenchromatographie entfernt werden konnten.  
Die Signale des Acetonitrils und des Polysiloxan Schlifffetts haben sich nicht verändert.

\subsection{UV/Vis}
Die iEDDA-Reaktion mit Bicyclononin schwächt durch das Entfernen von zwei Stickstoffatomen das $\pi$-System, wodurch der Energieunterschied zwischen den Grenzorbitalen vergrößert wird. 
Da in der UV/Vis-Spektroskopie die Übergänge vom HOMO zum LUMO angeregt werden, ist zu erwarten, dass nach der Reaktion die Absorption geringer ist als davor.
\autoref{fig: UV/Vis} zeigt die aufgenommenen UV/Vis-Spektren vor und nach der Reaktion mit BCN.

\begin{figure}[H]
    \centering
    \includegraphics[width=1\textwidth]{Bilder/UV_Vis.png}
    \caption{UV/Vis-Spektren von \ch{[Re(CO)3Cl(N2Tz)]} (blau) und von [\ch{Re(CO)3Cl}(4,5-BCN-3-(2-pyrimidyl)-6-methylpyridazin)] (rot) zwischen 200 und 900 nm, gemessen in \ch{CH3CN}.}
    \label{fig: UV/Vis}
\end{figure}
Die Absorption des $^1$MLCT bei 453 nm wird vom Rhenium-Tetrazin-Komplex zum Rhenium-Pyridazin-BCN-Komplex im Spektrum nach links verschoben.
Dies lässt sich durch die erhöhten LUMOs erklären, da nach der Reaktion mit BCN mehr Energie und somit Strahlung einer geringeren Wellenlänge benötigt wird, um die Elektronenübergänge anzuregen.
Außerdem ist die Bande schmaler, da ab einer Wellenlänge von ca. 469,8 nm die Strahlung nicht mehr genug Energie besitzt, um diese Übergänge anzuregen.
Die Absorption der intramolekularen Ligand-Übergänge bei 273 nm ändert sich geringfügig in ihrer Breite und etwas stärker in ihrer Intensität.

%UV/Vis: 460 nm: MLCT, 250 nm: Intra-Ligand-Übergänge (n $\rightarrow \pi^*$)
%nachher: höheres LUMO, d.h. mehr Energie für Übergang -> kleinere Wellenlänge wird absorbiert

\subsection{Photolumineszenz}
Durch die iEDDA-Reaktion mit Bicyclononin wird die Elektronendichte im Ring vom Tetrazin zum Pyridazin erhöht, was sich strukturell durch weniger Stickstoffatome im Ring erklären lässt.
Durch die erhöhte Elektronendichte wird die $^3$MLCT Phosphoreszenz zwischen Rhenium und Ligand aktiviert.
Dies lässt sich in den gemessenen Photolumineszenz-Spektren in \autoref{fig: Fluoro} erkennen.
\begin{figure}[H]
    \centering
    \includegraphics[width=1\textwidth]{Bilder/Fluoro.png}
    \caption{Photolumineszenz-Spektren von \ch{[Re(CO)3Cl(N2Tz)]} (blau) und von [\ch{Re(CO)3Cl}(4,5-BCN-3-(2-pyrimidyl)-6-methylpyridazin)] (rot) zwischen 475 und 875 nm, gemessen in \ch{CH3CN}.}
    \label{fig: Fluoro}
\end{figure}
Die Intensität der Phosphoreszenz nimmt vom Rhenium-Tetrazin-Komplex zum Rhenium-Pyridazin-BCN-Komplex sehr stark zu. 
Am Intensitätsmaximum des Spektrums von [\ch{Re(CO)3Cl}(4,5-BCN-3-(2-pyrimidyl)-6-methylpyridazin)] bei einer Wellenlänge von $\lambda=$ 674 nm weist der Pyridazin-Komplex eine 54-fach größere Intensität als der Tetrazin-Komplex auf.
Der Intensitätsanstieg nach 850 nm ist auf den Einsatz eines Filters zurückzuführen, der Streulicht bis zu dieser Wellenlänge herausfiltert.
%Phosphoreszenz nach Zugabe von BCN




\newpage

\section{Experimenteller Teil}

\subsection{Methoden und Geräte}

\subsubsection{UV/Vis Spektroskopie}
Zur Untersuchung der Absorption des Rhenium-Komplexes vor und nach der Zugabe von Bicyclononin wurde ein UV/Vis-Spektrometer der Serie V-770 verwendet.
Dabei wird die Probe über ein Prisma nacheinander mit Strahlung einer spezifischen Wellenlänge bestrahlt.
Der Detektor ist dabei in einer geraden Linie zur Strahlungsquelle platziert, da so die Absorption durch den Unterschied der Strahlungsintensität vor und nach der Probe gemessen werden kann.  

\subsubsection{Fluoreszenzspektrometrie}
Zur Untersuchung der Phosphoreszenz des Rhenium-Komplexes vor und nach der iEDDA-Reaktion mit BCN wurde ein Photolumineszenzspektrometer der Serie RF-6000 verwendet.
Fluoreszenz bezeichnet die Emission von Licht nach Absorption eines Lichtquants, wohingegen Phosphoreszenz die Emission von Licht nach Übergang eines Elektrons zurück in den Grundzustand beschreibt, die länger andauert als die Fluoreszenz.
Anders als im UV/Vis Spektrometer wurde im Winkel von 90° gemessen, um die Anregungsstrahlung der Lampe nicht mit zu messen. 
Zudem wurde ein Filter hinter die Lichtquelle eingebaut, um Streulicht herauszufiltern.


\subsection{Synthesen}

\subsubsection{3-Methyl-6-(pyrimidin-2-yl)-1,2,4,5-tetrazin (N$_2$-Tz)}
Die Synthese von 3-Methyl-6-(pyrimidin-2-yl)-1,2,4,5-tetrazin (N$_2$-Tz) wurde nach dem Syntheseweg von Schnierle et al. durchgeführt.\supercite{Synthese}
In einem 1l Dreihalskolben wurde 2-Cyanopyrimidin (4,158 g; 39,8 mmol; 1 Äq) in Ethanol (150 ml) gelöst. 
Anschließend wurden Hydraziniumhydroxid (50 ml; 1,026 mol; 26 Äq), Acetonitril (130 ml; 2,365 mol; 60 Äq) und Schwefel (2,24 g; 70 mmol) zugegeben und die Lösung wurde bei 55 $^\circ$C über Nacht gerührt. 
Anschließend wurde das Lösungsmittel unter vermindertem Druck vollständig entfernt. 
Der entstandene Feststoff wurde anschließend in Phosphorsäure und Wasser (150 ml; 3:2) gelöst, langsam mit Natriumnitrit (26,21 g; 380 mmol; 9,55 Äq) versetzt und über Nacht gerührt. 
Danach wurde die Lösung mit einer Natriumhydrogencarbonatlösung (300 ml) versetzt und mit Chloroform (3 $\times$ 300 ml) extrahiert.
Nach der Extraktion wurde die Lösung über Natriumsulfat getrocknet und filtriert. 
Das Lösungsmittel des Filtrats wurde anschließend unter vermindertem Druck entfernt und der Feststoff bei -39$^\circ$C über Nacht auskristallisiert.
Der pinke Feststoff wurde danach über Säulenchromatographie auf Silica (Chloroform) aufgereinigt. 
Anschließend wurden die Fraktionen zusammengeführt und das Lösungsmittel wurde erneut unter vermindertem Druck entfernt, wodurch 3-Methyl-6-(pyrimidin-2-yl)-1,2,4,5-tetrazin (0,49 g; 2,82 mmol) als pinker Feststoff in einer Ausbeute von 7,08\% erhalten wurde.

\subsubsection{\ch{[Re(CO)3Cl(N2Tz)]}}
Danach wurde das Tetrazin (0,49 g; 2,82 mmol) in Chloroform (100 ml) gelöst und mit \ch{[Re(MeCN)2(CO)3Cl]} (0,517 g; 1,33 mmol) versetzt und bei Raumtemperatur über Nacht gerührt. 
Im Anschluss wurde das Lösungsmittel unter vermindertem Druck entfernt und durch Säulenchromatographie auf Silica (Chloroform) gereinigt. 
Daraufhin wurde das Lösungsmittel erneut unter vermindertem Druck entfernt und \ch{[Re(CO)3Cl(N2Tz)]} (0,643 g) wurde als schwarz-roter Feststoff erhalten.

\subsubsection{[\ch{Re(CO)3Cl}(4,5-BCN-3-(2-pyrimidyl)-6-methylpyridazin)]}
\ch{[Re(CO)3Cl(N2Tz)]} (ca. 3 mg) wurde bei Raumtemperatur in einer Küvette in Acetonitril (2 ml) gelöst und mit Bicyclononin (ca. 3 mg) versetzt, wodurch die rotbraune Lösung gelblich und anschließend farblos wurde.\supercite{Manu}


\section{Zusammenfassung}
Im Rahmen dieses Versuchs wurde der Einkernkomplex \ch{[Re(CO)3Cl(N2Tz)]} mit dem Tetrazinligand 3-Methyl-6-(pyrimidin-2-yl)-1,2,4,5-tetrazin (N$_2$-Tz) ausgehend von 2-Cyanopyrimidin synthetisiert und über eine iEDDA-Reaktion zu [\ch{Re(CO)3Cl}-(4,5-BCN-3-(2-pyrimidyl)-6-methylpyridazin)] umgesetzt.
Der synthetisierte Ligand und Rhenium-Komplex wurden mittels $\mathrm{^{1}H}$ charakterisiert und \ch{[Re(CO)3Cl(N2Tz)]} sowie [\ch{Re(CO)3Cl}(4,5-BCN-3-(2-pyrimidyl)-6-methylpyridazin)] mittels UV/Vis-\\Spektroskopie und Fluoreszenzspektrometrie untersucht.
%1-2 Sätze zu UV/Vis und Phosphoreszenz Ergebnis

Im Photolumineszenz-Spektrum konnte die $^3$MLCT Phosphoreszenz des Produkt-Komplexes der iEDDA-Reaktion durch einen über 50-fachen Intensitätsanstieg nach Zugabe von Bicyclononin nachgewiesen werden.

\printbibliography[title={Literatur}]


\end{document}
