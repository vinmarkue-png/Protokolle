\documentclass[a4paper,12pt,bibliography=totocnumbered]{scrartcl}

\usepackage[utf8]{inputenc} 
\usepackage[T1]{fontenc}
\usepackage[ngerman]{babel} 
\usepackage{amsmath, amssymb,amsfonts}
\usepackage{csquotes}
\usepackage{graphicx}
\usepackage[bookmarks,colorlinks=true]{hyperref}
\usepackage{geometry}
\usepackage{float}
\usepackage[final]{pdfpages}
\usepackage{framed, color} 
\usepackage{scrlayer-scrpage}
\usepackage{siunitx}
\usepackage{subfigure}
\renewcaptionname{ngerman}{\figurename}{Abb.}
\sisetup{detect-weight=true, detect-family=true,locale=DE,range-phrase={\,bis\,},list-final-separator ={\,\linebreak[0] \text{und}\,},separate-uncertainty=true,per-mode = symbol-or-fraction}
\DeclareSIUnit\curie{Ci}
\usepackage[backend=biber, style=chem-angew]{biblatex} 
\addbibresource{lit.bib} 

\usepackage{chemgreek}
\usepackage{chemformula}
\usepackage{chemscheme}

\urlstyle{same}
%Hyperlinks-Setup
\hypersetup{
	colorlinks,
	linktocpage,
	citecolor=black,
	filecolor=black,
	linkcolor=black,
	urlcolor=black
}

%\numberwithin{equation}{section}

\setlength{\parindent}{0 mm}
\setlength{\parskip}{2 mm} 



\pagestyle{scrheadings}
\ohead{\empty} %Header oben links auf linker Seite (ungerade Seitenzahl) und oben rechts auf rechter Seite (gerade Seitenzahl), beinhaltet gruppennummer und Versuchskürzel. Im Fall eine einseitigen Dokuments: Header oben rechts
\ihead{\VerfasserEINS\;\&\;\VerfasserZWEI} %Header oben rechts auf linker Seite und oben links auf rechter Seite. Beinhaltet die Namen der Verfasser. Im Fall eine einseitigen Dokuments: Header oben links!
\ofoot{\thepage} %Footer unten links auf linker und unten rechts auf rechter Seite, enthält die jeweilige Seitenzahl. Im Fall eines einseitigen Elements: Footer unten rechts!
\cfoot{\empty} %Mittig unten im Footer soll nichts eingetragen werden 
\ifoot{\empty} %Footer unten rechts auf linker und unten links auf rechter Seite. Hier ebenfalls leer.


\newcommand{\VERSUCHSDATUM}{29.10.2025 - 19.11.2025}
\newcommand{\PROTOKOLLDATUM}{\today}

\newcommand{\VerfasserEINS}{Vincent Kümmerle}
\newcommand{\MatNoEINS}{3712667}
\newcommand{\EmailEINS}{st187541@stud.uni-stuttgart.de}
\newcommand{\StudiengangEINS}{B.Sc. Chemie}

\newcommand{\VerfasserZWEI}{Elvis Gnaglo}
\newcommand{\MatNoZWEI}{3710504}
\newcommand{\EmailZWEI}{st189318@stud.uni-stuttgart.de}
\newcommand{\StudiengangZWEI}{B.Sc. Chemie}

\newcommand{\BETREUER}{Manuel Pech}
\newcommand{\GRUPPENNR}{A05}

\newcommand{\VERSUCHSNR}{Komplex Präparat}
\newcommand{\VERSUCHSNAME}{Synthese und Phosphoreszenz von \ch{[Re(CO)3Cl(N2Tz)]}}


\begin{document}
\thispagestyle{empty}


\begin{titlepage}

\begin{center}
\Huge{\textbf{\VERSUCHSNR\ - \\ \VERSUCHSNAME}}\\
\vspace{10mm}% Abstand
\Large{Protokoll zum Versuch des AC2 Praktikums von \\ \textbf{\VerfasserEINS\;\& \VerfasserZWEI}}\\
\vspace{10mm} 
\Large{Universität Stuttgart}\\
\end{center}
\vspace{1cm}
\begin{center}
\begin{tabular}{ll}
\large{Verfasser:}		& \large{\VerfasserEINS,} \large{\MatNoEINS} \\
 						& \large{\EmailEINS} \\
 						\vspace{0cm}\\
						& \large{\VerfasserZWEI,} \large{\MatNoZWEI} \\
                        & \large{\EmailZWEI} \\
						\vspace{0cm}\\
\large{Gruppennummer:}	& \large{\GRUPPENNR} \\
\vspace{0cm}\\
\large{Versuchszeitraum:}	& \large{\VERSUCHSDATUM} \\
\vspace{0cm}\\
\large{Betreuer:}		& \large{\BETREUER} \\
\vspace{0cm}\\
\large{Abgabenummer:} & \large{1. Abgabe}
\end{tabular}
\end{center}
\vspace{15mm}

\begin{center}
Stuttgart, den \PROTOKOLLDATUM
\end{center}

\end{titlepage}


\thispagestyle{empty}

\tableofcontents 

\clearpage

\renewcommand{\thepage}{\arabic{page}}
\setcounter{page}{1}

\section{Einleitung}
Tetrazine als Klasse heterocyclischer Liganden sind bereits seit der Synthese von A. Pinner 1893 bekannt\supercite{Pinner} und sind aufgrund ihrer starken $\pi$-Akzeptorfunktion von Bedeutung für die Koordination an Übergangsmetallkomplexe und Klick-Reaktionen.
Durch die vier elektronegativen Stickstoffatome weisen Tetrazine ein sehr elektronenarmes aromatisches System mit energetisch sehr tief liegenden $\pi^*$-Molekülorbitalen auf, die energiearme n $\rightarrow \pi^*$ Übergänge mit charakteristischer Absorption ermöglichen.\supercite{Dissertation}
Deshalb sind Tetrazine sehr leicht reduzierbar und erscheinen sehr oft pink. 
Tetrazine können durch die Interaktion der tief liegenden $\pi^*$-Molekülorbitale (LUMOs) mit den HOMOs (Lowest/Highest Occupied Molecular Orbitals) eines Diens Diels-Alder-Reaktionen mit inversem Elektronenbedarf eingehen, die auch als Klick-Reaktionen bezeichnet werden.
Zudem sind die $\pi^*$-MOs Zielorbitale für Metal-Ligand-Charge-Transfers (MLCT).
Durch die Koordination von Tetrazin-Derivaten an Übergangsmetalle wie Rhenium bilden sich Komplexe mit höherer Komplexstabilität, da die Derivate als mehrzähnige Liganden Chelat-Komplexe ausbilden.
Häufig wird hierfür Pentacarbonylrhenium(I)chlorid verwendet, in dem die Carbonyl-Liganden als $\pi$-Akzeptoren und der Chlorid-Ligand als $\sigma$-Donor oktaedrisch um das $d_6$-System des Rheniums koordiniert sind.
Die hierbei entstehenden Komplexe der Art \ch{[Re(CO)3Cl(L_x)]} finden Anwendung als Photosensibilisatoren, Katalysatoren für die elektrochemische Reduktion von \ch{CO2} zu \ch{CO} und in der Biodiagnostik, da sie bioorthogonal sind und damit $in~vivo$ genutzt werden können. \\
Somit ist das Ziel des Versuchs im ersten Teil 3-Methyl-6-(pyrimidin-2-yl)-1,2,4,5-tetrazin (N$_2$-Tz) und ausgehend davon \ch{[Re(CO)3Cl(N2Tz)]} zu synthetisieren. 
Im zweiten Teil wird eine Klick-Reaktion mit \ch{[Re(CO)3Cl(N2Tz)]} durchgeführt und die Phosphoreszenz des $fac$-\ch{Re(CO)3Cl} Komplexes untersucht.

\subsection{Syntheseweg}
Die erste Stufe der Synthese von \ch{[Re(CO)3Cl(N2Tz)]} wurde nach der Vorschrift von Schnierle et al. durch eine modifizierte Pinner Synthese durchgeführt.\supercite{modPinner}$^,$\supercite{Synthese}
Wie der Syntheseweg in \autoref{fig: Stufe1} zeigt, wird elementarer Schwefel als aktivierende Spezies verwendet, um die Nucleophilie des Hydraziniumhydroxids zu erhöhen und somit den Angriff des Stickstoff-Elektronenpaars an der Carbonitrilgruppe zu erleichtern.
Durch einen weiteren nucleophilen Angriff am Acetonitril wird das Sechsring-System des Dihydrotetrazins gebildet.

\begin{figure}[H]
    \centering
    \includegraphics[width=1\linewidth]{Bilder/Stufe 1&2.png}
    \caption{Syntheseweg der Synthese von 3-Methyl-6-(pyrimidin-2-yl)-1,2,4,5-tetrazin (N$_2$-Tz).\supercite{Synthese}}
    \label{fig: Stufe1}
\end{figure}
In der zweiten Stufe wurde das Dihydrotetrazin im sauren Milieu mit \ch{NaNO2} oxidiert und nach Neutralisation mit Chloroform extrahiert, um das 3-Methyl-6-(pyrimidin-2-yl)-1,2,4,5-tetrazin (N$_2$-Tz) zu synthetisieren.
Die dritte und vierte Stufe wurden nach dem Syntheseweg, der in \autoref{fig: Stufe4} abgebildet ist, durchgeführt.

\begin{figure}[H]
    \centering
    \includegraphics[width=1\linewidth]{Bilder/Stufe4.png}
    \caption{Syntheseweg der Synthese von [\ch{Re(CO)3Cl}(4,5-BCN-3-(2-pyrimidyl)-6-methylpyridazin)] ausgehend von N$_2$-Tz und \ch{[Re(MeCN)2(CO)3Cl]}.\supercite{Manu}}
    \label{fig: Stufe4}
\end{figure}
Dabei wurde zuerst das N$_2$-Tz mit \ch{[Re(MeCN)2(CO)3Cl]} an Rhenium komplexiert und somit \ch{[Re(CO)3Cl(N2Tz)]} synthetisiert.
In der vierten Stufe wurde aus dem Rhenium-Komplex und Bicyclononin (BCN) über eine Klick-Reaktion [\ch{Re(CO)3Cl}(4,5-BCN-3-(2-pyrimidyl)-6-methylpyridazin)] synthetisiert.


\section{Ergebnisse}

%\begin{figure}[H]
 %   \centering
  %  \includegraphics[scale=0.95]{.png}
   % \caption{.}
    %\label{fig: NMR}
%\end{figure}

0.0 ppm: Polysiloxan Schlifffett
2.1 ppm: Acetonitril


%UV/Vis: 460 nm: MLCT, 250 nm: Intra-Ligand-Übergänge (n $\rightarrow \pi^*$)
%nachher: höheres LUMO, d.h. mehr Energie für Übergang -> kleinere Wellenlänge wird absorbiert

%Phosphoreszenz nach Zugabe von BCN

\section{Diskussion}



\newpage

\section{Experimenteller Teil}
Die Synthese von  wurde nach dem Syntheseweg von Schnierle et al. durchgeführt. \cite{Synthese}.
In einem 1l Dreihalskolben wurde 2-Cyanopyrimidin (4,158 g; 39,8 mmol; 1 Äq) in Ethanol (150 ml) gelöst. 
Anschließend wurden Hydraziniumhydroxid (50 ml; 1026 mmol; 26 Äq), Acetonitril (130 ml; 2365 mmol; 60 Äq) und Schwefel (2,24 g; 70 mmol) zugegeben und die Lösung wurde bei 55 $^\circ$C über Nacht gerührt. 
Anschließend wurde das Lösungsmittel unter vermindertem Druck vollständig entfernt. 
Der entstandene Feststoff wurde anschließend in einem Lösungsmittelgemisch aus Phosphorsäure und Wasser (150 ml; 3:2) gelöst, langsam mit Natriumnitrit (26,21 g; 380 mmol; 9,55 Äq) versetzt und über Nacht gerührt. 
Danach wurde die Lösung mit einer Natriumhydrogencarbonatlösung (300 ml) versetzt und mit Chloroform (3 $\times$ 300 ml) extrahiert. 
Nach der Extraktion wurde die Lösung über Natriumsulfat getrocknet und filtriert. Das Lösungsmittel des Filtrats wurde anschließend unter vermindertem Druck entfernt und der Feststoff wurde bei 80 $^\circ$C über Nacht getrocknet. 
Der pinke Feststoff wurde danach über Säulenchromatographie auf Silica (Chloroform) aufgereinigt. 
Anschließend wurden die Fraktionen zusammengeführt und das Lösungsmittel wurde erneut unter vermindertem Druck entfernt, wodurch 3-Methyl-6-(pyrimidin-2-yl)-1,2,4,5-tetrazin (0,49 g; 2,82 mmol) als pinker Feststoff mit einer Ausbeute von 7,08/11,78 \% erhalten.\\
Danach wurde das Tetrazin (0,49 g; 2,82 mmol) in Chloroform (100 ml) gelöst und mit \ch{[Re(MeCN)2(CO)3Cl]} (0,517 g; 1,33 mmol) versetzt und bei Raumtemperatur über Nacht gerührt. 
Im Anschluss wurde das Lösungsmittel unter vermindertem Druck entfernt und durch Säulenchromatographie auf Silica (Chloroform) gereinigt. 
Daraufhin wurde das Lösungsmittel erneut unter vermindertem Druck entfernt und \ch{[Re(CO)3Cl(N2Tz)]} (0,643 g) wurde als schwarz-roter Feststoff erhalten.\\
Zum Schluss wurde der Komplex mit Bicyclononin versetzt, wobei die rotbraune Lösung gelblich wurde.

\subsection{Methoden und Geräte}

\subsubsection{UV/Vis Spektroskopie}


\subsubsection{Fluoreszenzspektrometrie}
Zur Untersuchung der Phosphoreszenz des Rhenium-Komplexes vor und nach der Klick-Reaktion mit BCN wurde ein Photolumineszenzspektrometer der Serie RF-6000 verwendet.
Fluoreszenz bezeichnet die Emission von Licht nach Absorption eines Lichtquants, wohingegen Phosphoreszenz die Emission von Licht nach Übergang eines Elektrons zurück in den Grundzustand beschreibt, die länger andauert als die Fluoreszenz.
Anders als im UV/Vis Spektrometer wurde im Winkel von 90° gemessen, um die Anregungsstrahlung der Lampe nicht mit zu messen. 
Zudem wurde ein Filter hinter die Lichtquelle eingebaut, um Streulicht herauszufiltern.


\subsection{Synthesen}

\subsubsection{3-Methyl-6-(pyrimidin-2-yl)-1,2,4,5-tetrazin (N$_2$-Tz)}


\subsubsection{\ch{[Re(CO)3Cl(N2Tz)]}}

\subsubsection{[\ch{Re(CO)3Cl}(4,5-BCN-3-(2-pyrimidyl)-6-methylpyridazin)]}

\section{Zusammenfassung}



%Fehlerquellen: Waschen mit THF statt DCM, 

\printbibliography[title={Literatur}]


\end{document}
