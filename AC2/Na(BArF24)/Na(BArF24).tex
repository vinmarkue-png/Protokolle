\documentclass[a4paper,12pt,bibliography=totocnumbered]{scrartcl}

\usepackage[utf8]{inputenc} 
\usepackage[T1]{fontenc}
\usepackage[ngerman]{babel} 
\usepackage{amsmath, amssymb,amsfonts}
\usepackage{csquotes}
\usepackage{graphicx}
\usepackage[bookmarks,colorlinks=true]{hyperref}
\usepackage{geometry}
\usepackage{float}
\usepackage[final]{pdfpages}
\usepackage{framed, color} 
\usepackage{scrlayer-scrpage}
\usepackage{siunitx}
\usepackage{subfigure}
\renewcaptionname{ngerman}{\figurename}{Abb.}
\sisetup{detect-weight=true, detect-family=true,locale=DE,range-phrase={\,bis\,},list-final-separator ={\,\linebreak[0] \text{und}\,},separate-uncertainty=true,per-mode = symbol-or-fraction}
\DeclareSIUnit\curie{Ci}
\usepackage[backend=biber, style=chem-angew]{biblatex} 
\addbibresource{lit.bib} 

\usepackage{chemgreek}
\usepackage{chemformula}
\usepackage{chemscheme}

\urlstyle{same}
%Hyperlinks-Setup
\hypersetup{
	colorlinks,
	linktocpage,
	citecolor=black,
	filecolor=black,
	linkcolor=black,
	urlcolor=black
}

%\numberwithin{equation}{section}

\setlength{\parindent}{0 mm}
\setlength{\parskip}{2 mm} 



\pagestyle{scrheadings}
\ohead{\empty} %Header oben links auf linker Seite (ungerade Seitenzahl) und oben rechts auf rechter Seite (gerade Seitenzahl), beinhaltet gruppennummer und Versuchskürzel. Im Fall eine einseitigen Dokuments: Header oben rechts
\ihead{\VerfasserEINS\;\&\;\VerfasserZWEI} %Header oben rechts auf linker Seite und oben links auf rechter Seite. Beinhaltet die Namen der Verfasser. Im Fall eine einseitigen Dokuments: Header oben links!
\ofoot{\thepage} %Footer unten links auf linker und unten rechts auf rechter Seite, enthält die jeweilige Seitenzahl. Im Fall eines einseitigen Elements: Footer unten rechts!
\cfoot{\empty} %Mittig unten im Footer soll nichts eingetragen werden 
\ifoot{\empty} %Footer unten rechts auf linker und unten links auf rechter Seite. Hier ebenfalls leer.


\newcommand{\VERSUCHSDATUM}{21.10.2025 - 11.11.2025}
\newcommand{\PROTOKOLLDATUM}{\today}

\newcommand{\VerfasserEINS}{Vincent Kümmerle}
\newcommand{\MatNoEINS}{3712667}
\newcommand{\EmailEINS}{st187541@stud.uni-stuttgart.de}
\newcommand{\StudiengangEINS}{B.Sc. Chemie}

\newcommand{\VerfasserZWEI}{Elvis Gnaglo}
\newcommand{\MatNoZWEI}{3710504}
\newcommand{\EmailZWEI}{st189318@stud.uni-stuttgart.de}
\newcommand{\StudiengangZWEI}{B.Sc. Chemie}

\newcommand{\BETREUER}{Tobias Heitkemper}
\newcommand{\GRUPPENNR}{A05}

\newcommand{\VERSUCHSNR}{Elementorganik Präparat}
\newcommand{\VERSUCHSNAME}{Synthese von Na(BArF24)}


\begin{document}
\thispagestyle{empty}


\begin{titlepage}

\begin{center}
\Huge{\textbf{\VERSUCHSNR\ - \\ \VERSUCHSNAME}}\\
\vspace{10mm}% Abstand
\Large{Protokoll zum Versuch des AC2 Praktikums von \\ \textbf{\VerfasserEINS\;\& \VerfasserZWEI}}\\
\vspace{10mm} 
\Large{Universität Stuttgart}\\
\end{center}
\vspace{1cm}
\begin{center}
\begin{tabular}{ll}
\large{Verfasser:}		& \large{\VerfasserEINS,} \large{\MatNoEINS} \\
 						& \large{\EmailEINS} \\
 						\vspace{0cm}\\
						& \large{\VerfasserZWEI,} \large{\MatNoZWEI} \\
                        & \large{\EmailZWEI} \\
						\vspace{0cm}\\
\large{Gruppennummer:}	& \large{\GRUPPENNR} \\
\vspace{0cm}\\
\large{Versuchszeitraum:}	& \large{\VERSUCHSDATUM} \\
\vspace{0cm}\\
\large{Betreuer:}		& \large{\BETREUER} \\
\vspace{0cm}\\
\large{Abgabenummer:} & \large{1. Abgabe}
\end{tabular}
\end{center}
\vspace{15mm}

\begin{center}
Stuttgart, den \PROTOKOLLDATUM
\end{center}

\end{titlepage}


\thispagestyle{empty}

\tableofcontents 

\clearpage

\renewcommand{\thepage}{\arabic{page}}
\setcounter{page}{1}

\section{Einleitung}
In den vergangenen Jahren gab es immer mehr Studien zu organometallischen Verbindungen von schwach koordinierenden Anionen (weakly coordinating anions, WCAs) wie \ch{BF4}$^-$, \ch{PF6}$^-$ und \ch{AsF6}$^-$. 
Diese Komplexe bestehen entweder aus einem harten Kation und weichem Anion oder einem weichen Kation und einem harten Anion. 
Sie zeichnen sich besonders dadurch aus, dass sie sehr gute Abgangsgruppen sind und auch bei milden Reaktionsbedingungen durch andere Liganden ausgetauscht werden.
Durch diese Eigenschaft werden WCAs sehr häufig als Edukt für Organometallsynthesen verwendet, da sie die positive Ladung von Kationen gut stabilisieren können. 
Dadurch ermöglichen sie, dass auch sehr reaktive Kationen in Synthesen verwendet werden können.\supercite{Einleitung}\\
Die Gruppe der Borat-basierten WCAs ist dabei besonders zu beachten, da sie die am häufigsten verwendeten Anionen enthält. 
Das bereits lange bekannte \ch{[BPh4]}$^-$-Anion kann durch Austausch der Fluoridatome in \ch{BF4}$^-$ durch Phenylgruppen erhalten werden. 
Es koordiniert aufgrund seiner Größe schwächer an Metallkationen, da die negative Ladung über das gesamte Molekül verteilt wird. 
Allerdings ist dieses Anion anfällig gegenüber Hydrolyse und nach wie vor ein relativ stark koordinierender Ligand. 
Um diese Probleme zu beheben, wurden die Phenylgruppen fluoriert, sodass die Anionen \ch{[B(C6F5)4]}$^-$ und \ch{[B(C6H3(CF3)2)4]}$^-$ (im Folgenden \ch{[BArF24]}$^-$) vorliegen. 
Dabei ist das Anion \ch{[BArF24]}$^-$ besonders von Bedeutung. Es ist eines der meistverwendeten Anionen, da es leicht durch Synthesen gewonnen werden kann, seine Salze gut löslich sind und es in NMR-Spektren gut erkennbar ist. 
Außerdem finden die Alkalimetallsalze, besonders aber das Natriumsalz, Anwendung in der Methatese, als Initiator für Polymerisationen und werden in der Elektrochemie verwendet.\supercite{Aufarbeitung}

%\cite{WCA}.


\subsection{Syntheseweg}
Bei der Synthese des Natrium Tetrakis[3,5-bis(trifluormethyl)phenyl]borats mit Magnesium wurde von Leazer et al. festgestellt, dass sich das (Trifluormethyl)aryl Grignard Reagenz in Anwesenheit von überschüssigem Magnesium explosiv und exotherm zersetzen kann.\supercite{Gefahr}
Deshalb wird das Grignard Reagenz durch Metall-Halogen-Austausch nach dem sichereren Syntheseweg von Yakelis et al. durchgeführt, der in \autoref{Rgl} abgebildet ist.\supercite{Synthese}
Dabei wird das Magnesium im iso-Propylmagnesiumchlorid durch das Bromid des \text{3,5-Bis(trifluormethyl)-1-brom-}\\benzols
 ausgetauscht, wodurch das stabilere Grignard Reagenz %3,5-Bis(trifluormethyl)-1-magnesiumchloridbenzol
gebildet wird.
Durch Quenchen mit \ch{NaBF4} wird das Borat-Anion gebildet und danach ein wässriger Kationenaustausch mit \ch{Na2CO3} durchgeführt.
 
\begin{scheme}[H]
    \centering
    \includegraphics[scale=0.36]{Bilder/NaBArF Rgl.png}
    \caption{Syntheseweg der Synthese von Natrium Tetrakis[3,5-bis(trifluormethyl)- \\phenyl]borat.\supercite{Synthese}}
    \label{Rgl}
\end{scheme}

\section{Ergebnisse und Diskussion}
Zur Diskussion der gemessenen NMR-Spektren sollen zunächst die erwarteten Signale ausgehend von der Struktur abgeleitet werden.
Dazu zeigt \autoref{fig: Struktur} die Wasserstoff- und Kohlenstoffatome, die nach chemischer Äquivalenz exemplarisch in einem der 3,5-Bis(trifluormethyl)phenyl-Reste markiert sind.


\begin{figure}[H]
    \centering
    \includegraphics[scale=0.4]{Bilder/BArF Struktur.png}
    \caption{Struktur des \ch{Na[BArF24]} mit chemisch unterschiedlichen Wasserstoff- und Kohlenstoffatomen.}
    \label{fig: Struktur}
\end{figure}




\newpage

\section{Experimenteller Teil}
Die Synthese von Natrium Tetrakis[3,5-bis(trifluormethyl)phenyl]borat wird nach dem Syntheseweg von Yakelis et al. durchgeführt.\supercite{Synthese}
In einem 500 ml Dreihalskolben wird 3,5-Bis(trifluormethyl)-1-brombenzol (19 ml; 0,104 mol; 5,8 Äq.) in THF (90 ml) vorgelegt und das farblose Reaktionsgemisch auf -20$^\circ$C gekühlt.
Dann wird über 45 min unter weiterer Kühlung eine braune Lösung aus \text{\ch{iPrMgCl}} in THF (2 $\mathrm{\frac{mol}{l}}$; 60 ml; 0,119 mmol; 6,6 Äq.) zugetropft und das rot-violette Gemisch über 50 min auf 0°C aufgewärmt.
Anschließend wird \ch{NaBF4} (1,98 g; 0,018 mol; 1,0 Äq.) für 4h im Vakuum bei 120$^\circ$C getrocknet, unter Gegenstrom zugegeben und das tiefviolette Reaktionsgemisch für 92h unter \ch{N2}-Atmosphäre gerührt.
Die nicht-luftempfindliche Aufarbeitung wird nach Martínez-Martínez et al. durchgeführt.\supercite{Aufarbeitung}
Dazu wird das Reaktionsgemisch in eine Lösung von \ch{Na2CO3} (45 g) in Wasser (450 ml) gegeben und 1h gerührt.
Danach wird mit Diethylether (3$\times$120 ml) extrahiert und die vereinigten organischen Phasen mit \ch{NaSO4} getrocknet.
Die verbliebenen Lösungsmittel werden unter vermindertem Druck entfernt, sodass ein öliger, brauner Rückstand zurück bleibt.
Dieser Rückstand wird für 48h bei -20$^\circ$C aus einer Mischung aus DCM und THF (1:1, 40 ml) umkristallisiert.
Die Kristalle werden abfiltriert und mit THF gewaschen. Das Filtrat wird nochmals für 48h bei -20$^\circ$C umkristallisiert und mit einer Mischung aus THF und DCM (2:1) gewaschen.
Der isolierte farblose Feststoff wird im Vakuum für 48h bei 80$^\circ$C getrocknet.
Das Produkt wird in die Glovebox überführt, unter Schutzgas-Atmosphäre abgewogen und gelagert.
Ausbeute: 5,645 g; 6,37 mmol; 35\%.

\section{Zusammenfassung}



%Fehlerquellen: Waschen mit THF statt DCM, 

\printbibliography[title={Literatur}]


\end{document}
