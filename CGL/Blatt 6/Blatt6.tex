\documentclass[a4paper,11pt]{scrartcl} 
\usepackage{geometry} 
\geometry{left=2.5cm} \geometry{top=3cm}

\usepackage[utf8]{inputenc} 
\usepackage[ngerman]{babel} 
\usepackage[T1]{fontenc} 
\usepackage{amsmath} 
\usepackage{amssymb} 
\usepackage[onehalfspacing]{setspace} 
\usepackage{graphicx} 
\usepackage{epstopdf} 
\usepackage{csquotes} 
\usepackage{array} 
\usepackage{upgreek} 
\usepackage{float} 
\usepackage{tikz} 
\usepackage[font=small]{caption}
\usepackage[backend=biber, style=chem-angew]{biblatex} 
\addbibresource{lit.bib} 
\usepackage{tabularx, booktabs, multirow} 
\usepackage{listings}
\usepackage{chemgreek}
\usepackage{chemformula}
\captionsetup{format=plain}
%\usepackage{hyperref}
\usepackage[hidelinks]{hyperref}
\usepackage{siunitx}
\sisetup{detect-weight=true, detect-family=true,locale=DE,range-phrase={\,bis\,},list-final-separator ={\,\linebreak[0] \text{und}\,},separate-uncertainty=true,per-mode = symbol-or-fraction}

\parindent0pt %Kein Einzug am Anfang von Absätzen
\sloppy %Besserer Blocksatz
%\renewcaptionname{ngerman}{\figurename}{Abb.} 
%\renewcaptionname{ngerman}{\tablename}{Tab.} 
\setkomafont{section}{\normalsize}

\lstset{
    language=Python,
    numbers=left,
    numberstyle=\tiny\color{gray},
    numbersep=8pt,
    stepnumber=1,
    frame=single,
    tabsize=4,
    showstringspaces=false,
    breaklines=true,
    keywordstyle=\color{blue}\bfseries,
    stringstyle=\color{orange},
    commentstyle=\color{green!50!black},
    basicstyle=\ttfamily\small,
    literate={ö}{{\"o}}1
             {ä}{{\"a}}1
             {ü}{{\"u}}1
             {ß}{{\ss}}1
}

\title{Blatt 6}
\author{Vincent Kümmerle und Elvis Gnaglo}
\date{\today}

\begin{document}

\maketitle

\section{Listen}
Für die Liste a = [2, "d", 5, 8, 233, "dx", 54, "we", "g", ..., 72, 23, "g"] sind die Zugriff Befehle in \autoref{tab: Liste} aufgeführt.

\begin{table}[H]
    \centering
    \caption{Zugriff auf verschiedene Listenelemente in Python.}
    \label{tab: Liste}
    \begin{tabular}{c|c}
        Element & Zugriff \\
    \hline
        viertes & a[3] \\
        vorletztes & a[-2] \\
        drittes bis drittletztes & a[2:-2] \\
        jedes 2. ab dem 4.& a[3::2] \\
        jedes 3. rückwärts ab dem vorletzten & a[-2::-3] \\
        7. entfernen & del a[6]
    \end{tabular}
\end{table}

\section{Datentypen und Ausdrücke}
Die erwarteten Ergebnisse sind mit Begründungen in \autoref{tab: Datentypen} aufgelistet.
\begin{table}[H]
    \centering
    \caption{Erwartete Ergebnisse für verschiedene Ausdrücke in Python.}
    \label{tab: Datentypen}
    \begin{tabular}{c|c|c}
        Ausdruck & Ergebnis & Begründung \\
        \hline
        3 + 5 & 8 & Ganzzahlsummation: int + int = int \\
        3 + 5.0 & 8.0 & Typkonvertierung: Addition mit float ergibt float \\
        ``3'' + ``5'' & ``35'' & Zeichenketten werden aneinandergehängt (Konkatenation) \\
        ``3'' * 5 & ``33333'' & String wird fünfmal wiederholt \\
        3 // 2 & 1 & Ganzzahldivision; Ergebnis wird abgerundet \\
        3 / 2 & 1.5 & Normale Division ergibt float \\
        int(2.71828) & 2 & int() schneidet Nachkommastellen ab \\
        round(2.71828) & 3 & Mathematische Rundung auf ganzzahliges Ergebnis \\
        ``hallo'' + ``Welt'' & ``halloWelt'' & String-Konkatenation \\
        
    \end{tabular}
\end{table}

\section{Gerade / Ungerade}

\begin{lstlisting}
def is_even(number):
    """Check, if a given number is even."""
    assert isinstance(number, int) and number >= 1, "Die gegebene Zahl ist keine positive, natürliche Zahl."
    return number % 2 == 0

eingabe_text = input("Bitte gib eine Zahl ein: ")

try:
    zahl = int(eingabe_text)
    
    if is_even(zahl):
        print(f"Die Zahl {zahl} ist gerade.")
    else:
        print(f"Die Zahl {zahl} ist ungerade.")

except ValueError:
    print("Das war keine gültige ganze Zahl!")
except AssertionError as e:
    print(f"Fehler: {e}")
\end{lstlisting}

\section{Summation}

\begin{lstlisting}

\end{lstlisting}

\section{Fakultät}

\begin{lstlisting}

\end{lstlisting}

\end{document}