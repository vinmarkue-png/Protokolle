\documentclass[a4paper,11pt]{scrartcl} 
\usepackage{geometry} 
\geometry{left=2.5cm} \geometry{top=3cm}

\usepackage[utf8]{inputenc} 
\usepackage[ngerman]{babel} 
\usepackage[T1]{fontenc} 
\usepackage{amsmath} 
\usepackage{amssymb} 
\usepackage[onehalfspacing]{setspace} 
\usepackage{graphicx} 
\usepackage{epstopdf} 
\usepackage{csquotes} 
\usepackage{array} 
\usepackage{upgreek} 
\usepackage{float} 
\usepackage{tikz} 
\usepackage[font=small]{caption}
\usepackage[backend=biber, style=chem-angew]{biblatex} 
\addbibresource{Blatt 7.bib} 
\usepackage{tabularx, booktabs, multirow} 
\usepackage{listings}
\usepackage{chemgreek}
\usepackage{chemformula}
\captionsetup{format=plain}
%\usepackage{hyperref}
\usepackage[hidelinks]{hyperref}
\usepackage{siunitx}
\sisetup{detect-weight=true, detect-family=true,locale=DE,range-phrase={\,bis\,},list-final-separator ={\,\linebreak[0] \text{und}\,},separate-uncertainty=true,per-mode = symbol-or-fraction}

\parindent0pt %Kein Einzug am Anfang von Absätzen
\sloppy %Besserer Blocksatz
%\renewcaptionname{ngerman}{\figurename}{Abb.} 
%\renewcaptionname{ngerman}{\tablename}{Tab.} 
\setkomafont{section}{\normalsize}

\lstset{
    language=Python,
    numbers=left,
    numberstyle=\tiny\color{gray},
    numbersep=8pt,
    stepnumber=1,
    frame=single,
    tabsize=4,
    showstringspaces=false,
    breaklines=true,
    keywordstyle=\color{blue}\bfseries,
    stringstyle=\color{red},
    commentstyle=\color{green!50!black},
    basicstyle=\ttfamily\small,
    literate={ö}{{\"o}}1
             {ä}{{\"a}}1
             {ü}{{\"u}}1
             {ß}{{\ss}}1
}

\title{Blatt 10}
\author{Vincent Kümmerle und Elvis Gnaglo}
\date{\today}

\begin{document}

\maketitle

\section{Problem des Handlungsreisenden}

\begin{lstlisting}
import itertools
import math
import random
import matplotlib.pyplot as plt
import time

def distanz(p1, p2):
    """Berechnet die euklidische Distanz zwischen zwei Punkten p1 und p2."""
    return math.sqrt((p1[0] - p2[0])**2 + (p1[1] - p2[1])**2)

def pfad_laenge(pfad):
    """Berechnet die Gesamtlänge eines Pfades (Liste von Punkten)."""
    laenge = 0
    # Iterieration von 0 bis zum vorletzten Punkt und addition der Distanz zum Nachfolger
    for i in range(len(pfad) - 1):
        laenge += distanz(pfad[i], pfad[i+1])
    return laenge

def main():
    n = 10  # Anzahl der Punkte
    random.seed() # Setzt den Seed für reproduzierbare Ergebnisse, ohne Wert ist das Ergebnis für jede Ausführung des Programms unterschiedlich  
    
    # Zufällige Punkte für x und y zwischen 0 und 100 generieren  
    punkte = [(random.uniform(0, 100), random.uniform(0, 100)) for _ in range(n)]
    
    print(f"Berechne kürzesten Pfad für {n} Punkte...")
    print(f"Zu prüfende Permutationen: {math.factorial(n):,}")
    
    start_time = time.time()

    # Alle Permutationen berechnen und die mit der kürzesten Strecke finden
    kuerzeste_strecke = float('inf')
    bester_pfad = None

    # itertools.permutations erstellt alle möglichen Reihenfolgen
    for perm in itertools.permutations(punkte):
        aktuelle_laenge = pfad_laenge(perm)
        
        if aktuelle_laenge < kuerzeste_strecke:
            kuerzeste_strecke = aktuelle_laenge
            bester_pfad = perm

    end_time = time.time()
    print(f"Fertig in {end_time - start_time:.2f} Sekunden.")
    print(f"Kürzeste Strecke: {kuerzeste_strecke:.2f}")

    
    if bester_pfad:
        # Koordinaten für den Plot entpacken
        x_coords = [p[0] for p in bester_pfad]
        y_coords = [p[1] for p in bester_pfad]

        plt.figure(figsize=(8, 6))
        
        # Den Weg zeichnen
        plt.plot(x_coords, y_coords, color='black', linestyle='-', linewidth=2, zorder=1, label='Kürzester Pfad')
        
        # Punkte anzeigen
        plt.scatter(x_coords, y_coords, color='green', s=100, zorder=2, label='Orte')
        
        # Start- und Endpunkt beschriften
        plt.text(x_coords[0], y_coords[0], ' Start', verticalalignment='bottom', fontweight='bold')
        plt.text(x_coords[-1], y_coords[-1], ' Ende', verticalalignment='bottom', fontweight='bold')

        plt.title(f'Kürzester Pfad durch {n} zufällige Punkte\nLänge: {kuerzeste_strecke:.2f}')
        plt.xlabel('X-Koordinate')
        plt.ylabel('Y-Koordinate')
        plt.grid(True)
        plt.legend()
        
        plt.show()

if __name__ == "__main__":
    main()
\end{lstlisting}
\begin{figure}[H]
    \centering
    \includegraphics[width=0.85\textwidth]{Reisender1.pdf}    
    \caption{Plot einer zufälligen Reisestrecke} 
\end{figure}

\begin{figure}[H]
    \centering
    \includegraphics[width=0.85\textwidth]{Reisender2.pdf}    
    \caption{Plot einer anderen zufälligen Reisestrecke} 
\end{figure}

\section{Fehleranalyse einer gedämpften Schwingung}

\begin{lstlisting}

\end{lstlisting}
% \begin{figure}[H]
%     \centering
%     \includegraphics[width=0.85\textwidth]{Sympy.pdf}
%     \caption{Plot der Funktion und ihrer Ableitung} 
%     \label{fig: sympy}
% \end{figure}

\end{document}