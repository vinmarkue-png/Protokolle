\documentclass[a4paper,11pt]{scrartcl} 
\usepackage{geometry} 
\geometry{left=2.5cm} \geometry{top=3cm}

\usepackage[utf8]{inputenc} 
\usepackage[ngerman]{babel} 
\usepackage[T1]{fontenc} 
\usepackage{amsmath} 
\usepackage{amssymb} 
\usepackage[onehalfspacing]{setspace} 
\usepackage{graphicx} 
\usepackage{epstopdf} 
\usepackage{csquotes} 
\usepackage{array} 
\usepackage{upgreek} 
\usepackage{float} 
\usepackage{tikz} 
\usepackage[font=small]{caption}
\usepackage[backend=biber, style=chem-angew]{biblatex} 
%\addbibresource{Blatt 7.bib} 
\usepackage{tabularx, booktabs, multirow} 
\usepackage{listings}
\usepackage{chemgreek}
\usepackage{chemformula}
\captionsetup{format=plain}
%\usepackage{hyperref}
\usepackage[hidelinks]{hyperref}
\usepackage{siunitx}
\sisetup{detect-weight=true, detect-family=true,locale=DE,range-phrase={\,bis\,},list-final-separator ={\,\linebreak[0] \text{und}\,},separate-uncertainty=true,per-mode = symbol-or-fraction}

\parindent0pt %Kein Einzug am Anfang von Absätzen
\sloppy %Besserer Blocksatz
%\renewcaptionname{ngerman}{\figurename}{Abb.} 
%\renewcaptionname{ngerman}{\tablename}{Tab.} 
\setkomafont{section}{\normalsize}

\lstset{
    language=Python,
    numbers=left,
    numberstyle=\tiny\color{gray},
    numbersep=8pt,
    stepnumber=1,
    frame=single,
    tabsize=4,
    showstringspaces=false,
    breaklines=true,
    keywordstyle=\color{blue}\bfseries,
    stringstyle=\color{red},
    commentstyle=\color{green!50!black},
    basicstyle=\ttfamily\small,
    literate={ö}{{\"o}}1
             {ä}{{\"a}}1
             {ü}{{\"u}}1
             {ß}{{\ss}}1
}

\title{Blatt 14}
\author{Vincent Kümmerle und Elvis Gnaglo}
\date{\today}

\begin{document}

\maketitle

\section{Sortieralgorithmen in C++}

\begin{lstlisting}
#include <iostream>
#include <string>
#include <vector>
#include <algorithm>
class Person {
public:
	std::string name;
	int age;
};

bool is_younger(const Person& a, const Person& b) {
return a.age < b.age; // Ausgabe true, wenn a jünger ist als b
}
int main() {
std::vector<Person> people{
{"Alice", 30},
{"Bob", 24},
{"Clara", 41}
};
// std::sort mit der Funktion is_younger aufrufen
std::sort(people.begin(), people.end(), is_younger);
// Ausgabe zur Ueberpruefung
for (const auto& person : people) {
std::cout << person.name << ": " << person.age << " Jahre" << std::endl;}
}
\end{lstlisting}

Ausgabe: \\
Bob: 24 Jahre \\
Alice: 30 Jahre \\
Clara: 41 Jahre

\section{Git Grundlagen}

Zu Beginn wurde versucht die Aufgabe, nach der Anleitung auf dem Übungsblatt, über die Kommandozeile zu lösen. Nachdem sich das als unnötig kompliziert erwies wurden die in \autoref{Kommunismus} dargestellten Schritte in Visual Studio Code wiederholt. 
\begin{figure}[H]
    \centering
    \includegraphics[width=1\textwidth]{C:/Studium/5. Semester/AC II lab/Protokolle/CGL/Blatt 14/Bilder/Komando.jpeg}    
    \caption{Screenshot der Kommandozeile.}
    \label{Kommunismus} 
\end{figure}
Anschließend wurde das Repository geforked, über den Link in VSC geklont und ein neuer Branch erstellt, wie in \autoref{DIE GABEL} dargestellt. 
\begin{figure}[H]
    \centering
    \includegraphics[width=1\textwidth]{C:/Studium/5. Semester/AC II lab/Protokolle/CGL/Blatt 14/Bilder/Branch_created.jpeg}    
    \caption{Screenshot von Git nach erstellen des neuen Branches.} 
    \label{DIE GABEL}
\end{figure}
Danach wurde der Inhalt der Datei von ,,Hallo'' zu ,,Hallo Welt'' geändert und die Änderungen wurden anschließend commited, wie in \autoref{Commitment ist wichtig}
\begin{figure}[H]
    \centering
    \includegraphics[width=1\textwidth]{C:/Studium/5. Semester/AC II lab/Protokolle/CGL/Blatt 14/Bilder/Commit.jpeg}    
    \caption{Screenshot des commit Vorgangs.} 
    \label{Commitment ist wichtig}
\end{figure}
Nach dem Commit wurde anschließend der ganze Branch gepublished, was in \autoref{Ich bin ein Verlag} zu sehen ist.
\begin{figure}[H]
    \centering
    \includegraphics[width=1\textwidth]{C:/Studium/5. Semester/AC II lab/Protokolle/CGL/Blatt 14/Bilder/Branch.jpeg}    
    \caption{Screenshot des Branches nachdem der neu erstellte gepublished wurde.}
    \label{Ich bin ein Verlag} 
\end{figure}
Zum Schluss wurde eine Pull request in Git durchgeführt, was in \autoref{Pull me closer} dergestellt ist.
\begin{figure}[H]
    \centering
    \includegraphics[width=1\textwidth]{C:/Studium/5. Semester/AC II lab/Protokolle/CGL/Blatt 14/Bilder/Pull.jpeg}    
    \caption{Screenshot von Git nach dem die Pull request ausgeführt wurde.} 
    \label{Pull me closer}
\end{figure}

\end{document}