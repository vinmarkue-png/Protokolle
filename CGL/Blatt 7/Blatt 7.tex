\documentclass[a4paper,11pt]{scrartcl} 
\usepackage{geometry} 
\geometry{left=2.5cm} \geometry{top=3cm}

\usepackage[utf8]{inputenc} 
\usepackage[ngerman]{babel} 
\usepackage[T1]{fontenc} 
\usepackage{amsmath} 
\usepackage{amssymb} 
\usepackage[onehalfspacing]{setspace} 
\usepackage{graphicx} 
\usepackage{epstopdf} 
\usepackage{csquotes} 
\usepackage{array} 
\usepackage{upgreek} 
\usepackage{float} 
\usepackage{tikz} 
\usepackage[font=small]{caption}
\usepackage[backend=biber, style=chem-angew]{biblatex} 
\addbibresource{Blatt 7.bib} 
\usepackage{tabularx, booktabs, multirow} 
\usepackage{listings}
\usepackage{chemgreek}
\usepackage{chemformula}
\captionsetup{format=plain}
%\usepackage{hyperref}
\usepackage[hidelinks]{hyperref}
\usepackage{siunitx}
\sisetup{detect-weight=true, detect-family=true,locale=DE,range-phrase={\,bis\,},list-final-separator ={\,\linebreak[0] \text{und}\,},separate-uncertainty=true,per-mode = symbol-or-fraction}

\parindent0pt %Kein Einzug am Anfang von Absätzen
\sloppy %Besserer Blocksatz
%\renewcaptionname{ngerman}{\figurename}{Abb.} 
%\renewcaptionname{ngerman}{\tablename}{Tab.} 
\setkomafont{section}{\normalsize}

\lstset{
    language=Python,
    numbers=left,
    numberstyle=\tiny\color{gray},
    numbersep=8pt,
    stepnumber=1,
    frame=single,
    tabsize=4,
    showstringspaces=false,
    breaklines=true,
    keywordstyle=\color{blue}\bfseries,
    stringstyle=\color{orange},
    commentstyle=\color{green!50!black},
    basicstyle=\ttfamily\small,
    literate={ö}{{\"o}}1
             {ä}{{\"a}}1
             {ü}{{\"u}}1
             {ß}{{\ss}}1
}

\title{Blatt 7}
\author{Vincent Kümmerle und Elvis Gnaglo}
\date{\today}

\begin{document}

\maketitle

\section{Zotero}
Machinelles Lernen wird verwendet, um Muster in Datensammlungen herauszufinden. \cite{quantum}
\begin{figure}[H]
    \centering
    \includegraphics[width=1\textwidth]{Quantum.png}
    \caption{Zeitschriftenartikel über Quantum Machine Learning in der Zotero-Bibliothek.}
    \label{Quantum}
\end{figure}
Das Buch “Experimentalphysik 1” von Wolfgang Demtröder erklärt die Grundlagen der Mechanik und Wärmelehre. \cite{fisiks}
\begin{figure}[H]
    \centering
    \includegraphics[width=1\textwidth]{fisiks.jpg}
    \caption{Buch über die Mechanik und Wärmelehre in der Physik.}
    \label{Fisiks}
\end{figure}

\section{List Comprehension}
\begin{lstlisting}
import random
import math
import matplotlib.pyplot as plt

# Gegebene Funktion (angenommene Implementierung)
def generate_random_point_in_square(side_length):
    """Generiert einen zufälligen Punkt (x, y) in einem Quadrat um den Ursprung."""
    # Bereich: -side_length/2 bis +side_length/2
    half_side = side_length / 2
    x = random.uniform(-half_side, half_side)
    y = random.uniform(-half_side, half_side)
    return (x, y)

# --- DEIN CODE HIER ---

def distance(p1, p2):
    """Berechnet den Euklidischen Abstand zwischen zwei Punkten p1 und p2."""
    return math.sqrt((p1[0] - p2[0])**2 + (p1[1] - p2[1])**2)

def point_in_circle(point, center, radius):
    """Prüft, ob point innerhalb des Kreises um center mit radius liegt."""
    return distance(point, center) <= radius

# Parameter
seitenlaenge = 2.0
kreis_radius = 1.0
kreis_mittelpunkt = (0, 0)
anzahl_punkte = 1000

# 1. Liste mit 1000 Punkten erzeugen (List Comprehension)
points = [generate_random_point_in_square(seitenlaenge) for _ in range(anzahl_punkte)]

# 2. Punkte im Kreis filtern (List Comprehension)
points_inside = [p for p in points if point_in_circle(p, kreis_mittelpunkt, kreis_radius)]

# 3. Verhältnis berechnen
verhaeltnis = len(points_inside) / len(points)
print(f"Anzahl Punkte gesamt: {len(points)}")
print(f"Anzahl Punkte im Kreis: {len(points_inside)}")
print(f"Verhältnis: {verhaeltnis}")

# Begründung der Konvergenz:
# Das Verhältnis konvergiert gegen Pi / 4.
# Grund: 
# Fläche des Quadrats (A_q) = (2r)² = 4r²
# Fläche des Kreises (A_k) = pi * r²
# Verhältnis A_k / A_q = (pi * r²) / (4r²) = pi / 4
# Bei r=1 und Seitenlänge=2 ist das Verhältnis also pi/4 (~0.785).

# 4. Listen für Plot erstellen (List Comprehension)
# Alle Punkte (zur Darstellung des Quadrats)
x_all = [p[0] for p in points]
y_all = [p[1] for p in points]

# Nur Punkte im Kreis (zur farblichen Unterscheidung)
x_in = [p[0] for p in points_inside]
y_in = [p[1] for p in points_inside]

# Plotten
plt.figure(figsize=(6, 6))
plt.scatter(x_all, y_all, color='blue', s=5, label='Außerhalb')
plt.scatter(x_in, y_in, color='red', s=5, label='Innerhalb')
plt.title(f'Monte Carlo Simulation (Verhältnis = {verhaeltnis})')
plt.legend(loc='upper right')
plt.axis('equal') # Wichtig, damit der Kreis rund aussieht
plt.show()
\end{lstlisting}

\section{Dateien lesen und Daten plotten}
\begin{lstlisting}

\end{lstlisting}
\printbibliography[title={Literaturverzeichnis}]

\end{document}