\documentclass[a4paper,11pt]{scrartcl} 
\usepackage{geometry} 
\geometry{left=2.5cm} \geometry{top=3cm}

\usepackage[utf8]{inputenc} 
\usepackage[ngerman]{babel} 
\usepackage[T1]{fontenc} 
\usepackage{amsmath} 
\usepackage{amssymb} 
\usepackage[onehalfspacing]{setspace} 
\usepackage{graphicx} 
\usepackage{epstopdf} 
\usepackage{csquotes} 
\usepackage{array} 
\usepackage{upgreek} 
\usepackage{float} 
\usepackage{tikz} 
\usepackage[font=small]{caption}
\usepackage[backend=biber, style=chem-angew]{biblatex} 
\addbibresource{lit.bib} 
\usepackage{tabularx, booktabs, multirow} 

\usepackage{chemgreek}
\usepackage{chemformula}
\captionsetup{format=plain}
%\usepackage{hyperref}
\usepackage[hidelinks]{hyperref}
\usepackage{siunitx}
\sisetup{detect-weight=true, detect-family=true,locale=DE,range-phrase={\,bis\,},list-final-separator ={\,\linebreak[0] \text{und}\,},separate-uncertainty=true,per-mode = symbol-or-fraction}

\parindent0pt %Kein Einzug am Anfang von Absätzen
\sloppy %Besserer Blocksatz
%\renewcaptionname{ngerman}{\figurename}{Abb.} 
%\renewcaptionname{ngerman}{\tablename}{Tab.} 
\setkomafont{section}{\normalsize}


\title{Säure-Base-Titration einer Schwefelsäurelösung mit Natronlauge}
\author{Anonym}
\date{\today}

\begin{document}

\maketitle


\section{Abstract}
In der Analyse wurde eine unbekannte Stoffmenge an Schwefelsäure über eine potentiometrisch verfolgte Säure-Base-Titration mit 0,1-molarer \ch{NaOH}-Maßlösung bestimmt. Dazu wurde die Titration mit zwei Aliquoten durchgeführt, wobei die ermittelte Stoffmenge \SI{2,442}{mmol} um 1{,}4 \% von der enthaltenen (\SI{2,4075}{mmol}) abweichte.

\section{Durchführung}
Die Analysenlösung wurde mit demin. Wasser bis zur Ringmarkierung aufgefüllt und durch Schütteln homogenisiert. Anschließend wurde ein 25 ml-Aliquot entnommen und in einem 250 ml-Becherglas mit 50 ml demin. Wasser versetzt. Zudem wurde ein abgespülter Rührfisch ins Becherglas gegeben, ein pH-Meter mit Abstand zum Rührfisch befestigt und ein Magnetrührer unter dem Becherglas platziert. Zu Beginn der Titration wurde alle 0,5 ml ein Messwert aufgenommen, in der Nähe vom Äquivalenzpunkt alle 0,2 ml und am Äquivalenzpunkt alle 0,05 bis 0,1 ml, was wenigen Tropfen entspricht, die langsam zugegeben wurden. \cite{Skript} 
Nach jeder Zugabe wurde gewartet und der sich einstellende pH-Wert notiert.
Die Durchführung wurde für zwei weitere 25 ml-Aliquots sowie für einen 10 ml-Aliquot wiederholt.

\section{Messwerte} \label{Messwerte} % für weiteren Querverweis
\begin{figure}[H]
    \centering
    \includegraphics[scale=0.4]{Aliquot 1.png}
    \caption{Titrationskurve des 1. Aliquots einer \ch{H2SO4}-Analysenlösung mit 0,1-molarer NaOH.}
    \label{fig: Aliquot 1}
\end{figure}

\begin{figure}[H]
    \centering
    \includegraphics[scale=0.4]{Aliquot 2.png}
    \caption{Titrationskurve des 2. Aliquots einer \ch{H2SO4}-Lösung mit NaOH.}
    \label{fig: Aliquot 2}
\end{figure}

Aus den Titrationskurven in Abbildung \ref{fig: Aliquot 1} und \ref{fig: Aliquot 2} werden die Äquivalenzpunkte mit der Tangentenmethode \cite{Tangente} bestimmt und in Tabelle \ref{tab: Äquivalenz} aufgeführt.
\begin{table}[H]
    \centering
    \caption{Zugegebene Volumina bis zum Äquivalenzpunkt.}
    \label{tab: Äquivalenz}
    \begin{tabular}{l|r}
        Aliquot & Volumen [ml] \\
    \hline
        1 & 12,19 \\
        2 & 12,23 \\
    \end{tabular}
\end{table}

\newpage

\section{Berechnung des Analysenergebnisses}
Zuerst wird der Mittelwert der zugegebenen Volumina aus dem Kapitel "\nameref{Messwerte}" \ als $\overline{V} = \SI{12,21}{ml}$ bestimmt.
Bei der Titration wird ein Mol Schwefelsäure mit zwei Mol NaOH nach Reaktionsgleichung (\ref{Reaktionsgleichung}) umgesetzt. \cite{RGL}
\begin{equation}
    \ch{H2SO4} + 2~\ch{NaOH} \rightleftharpoons \ch{Na2SO4} + 2~\ch{H2O}
    \label{Reaktionsgleichung}
\end{equation}
Die unbekannte Stoffmenge an \ch{H2SO4} am Äquivalenzpunkt entspricht der halben Stoffmenge an NaOH, lässt sich für einen Aliquoten nach Gleichung (\ref{Aliquot}) berechnen.
\begin{align}
    \begin{split}
    n_\mathrm{Aliquot} &= \frac{1}{2} \cdot c_\mathrm{\ch{NaOH}} \cdot V_\mathrm{\ch{NaOH}} \\
    &= \frac{1}{2} \cdot \SI{0,1}{\frac{mol}{l}} \cdot \SI{0,01221}{l} = \SI{0,6105}{mmol}
    \end{split}
    \label{Aliquot}
\end{align}
Für die Gesamtstoffmenge an Schwefelsäure wird die berechnete Stoffmenge des Aliquots mit vier multipliziert und nach Gleichung (\ref{Gesamt}) berechnet.
\begin{align}
    \begin{split}
    n_\mathrm{ges} &= 4 \cdot n_\mathrm{Aliquot} \\
    &= 4 \cdot \SI{0,6105}{mmol} = \SI{2,442}{mmol}
    \end{split}
    \label{Gesamt}
\end{align}

\section{Analysenergebnis}
Die Analyse enthielt 2,4075 mmol Schwefelsäure.
Somit lässt sich die Abweichung der ermittelten Stoffmenge von der tatsächlich enthaltenen folgendermaßen berechnen:
\begin{align*}
    \frac{n_{\mathrm{ermittelt}} - n_{\mathrm{enthalten}}} {n_{\mathrm{enthalten}}} \cdot 100\% = \frac{\SI{2,442}{mmol} - \SI{2,4075}{mmol}}{\SI{2,4075}{mmol}} \cdot 100\% = 1{,}4 \%
\end{align*}


\printbibliography[title={Bibliografie}]

\end{document}