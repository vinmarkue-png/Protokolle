\documentclass[a4paper,11pt]{scrartcl} 
\usepackage{geometry} 
\geometry{left=2.5cm} \geometry{top=3cm}

\usepackage[utf8]{inputenc} 
\usepackage[ngerman]{babel} 
\usepackage[T1]{fontenc} 
\usepackage{amsmath} 
\usepackage{amssymb} 
\usepackage[onehalfspacing]{setspace} 
\usepackage{graphicx} 
\usepackage{epstopdf} 
\usepackage{csquotes} 
\usepackage{array} 
\usepackage{upgreek} 
\usepackage{float} 
\usepackage{tikz} 
\usepackage[font=small]{caption}
\usepackage[backend=biber, style=chem-angew]{biblatex} 
\addbibresource{lit.bib} 
\usepackage{tabularx, booktabs, multirow} 

\usepackage{chemgreek}
\usepackage{chemformula}
\captionsetup{format=plain}
%\usepackage{hyperref}
\usepackage[hidelinks]{hyperref}
\usepackage{siunitx}
\sisetup{detect-weight=true, detect-family=true,locale=DE,range-phrase={\,bis\,},list-final-separator ={\,\linebreak[0] \text{und}\,},separate-uncertainty=true,per-mode = symbol-or-fraction}

\parindent0pt %Kein Einzug am Anfang von Absätzen
\sloppy %Besserer Blocksatz
%\renewcaptionname{ngerman}{\figurename}{Abb.} 
%\renewcaptionname{ngerman}{\tablename}{Tab.} 
\setkomafont{section}{\normalsize}


\title{Säure-Base-Titration einer Schwefelsäurelösung mit Natronlauge}
\author{Vincent Kümmerle und Elvis Gnaglo}
\date{\today}

\begin{document}

\maketitle

\tableofcontents

\newpage

\section{Messwerte} \label{Messwerte} % für weiteren Querverweis
\begin{figure}[H]
    \centering
    \includegraphics[scale=0.4]{Aliquot 1.png}
    \caption{Titrationskurve des 1. Aliquots einer \ch{H2SO4}-Analysenlösung mit 0,1-molarer NaOH.}
    \label{fig: Aliquot 1}
\end{figure}

Aus der Titrationskurve in Abbildung \ref{fig: Aliquot 1} werden die Äquivalenzpunkte mit der Tangentenmethode \cite{Tangente} bestimmt und in Tabelle \ref{tab: Äquivalenz} aufgeführt.
\begin{table}[H]
    \centering
    \caption{Zugegebene Volumina bis zum Äquivalenzpunkt.}
    \label{tab: Äquivalenz}
    \begin{tabular}{l|r}
        Aliquot & Volumen [ml] \\
    \hline
        1 & 12,19 \\
        2 & 12,23 \\
    \end{tabular}
\end{table}


\section{Berechnung des Analysenergebnisses}
Zuerst wird der Mittelwert der zugegebenen Volumina aus dem Kapitel "\nameref{Messwerte}" \ als $\overline{V} = \SI{12,21}{ml}$ bestimmt.
Bei der Titration wird ein Mol Schwefelsäure mit zwei Mol NaOH nach Reaktionsgleichung (\ref{Reaktionsgleichung}) umgesetzt. \cite{RGL}
\begin{equation}
    \ch{H2SO4} + 2~\ch{NaOH} \rightleftharpoons \ch{Na2SO4} + 2~\ch{H2O}
    \label{Reaktionsgleichung}
\end{equation}
Die unbekannte Stoffmenge an \ch{H2SO4} am Äquivalenzpunkt entspricht der halben Stoffmenge an NaOH, lässt sich für einen Aliquoten nach Gleichung \eqref{Aliquot} berechnen. \cite{Skript}

\begin{align}
    \begin{split}
    n_\mathrm{Aliquot} &= \frac{1}{2} \cdot c_\mathrm{\ch{NaOH}} \cdot V_\mathrm{\ch{NaOH}} \\
    &= \frac{1}{2} \cdot \SI{0,1}{\frac{mol}{l}} \cdot \SI{0,01221}{l} = \SI{0,6105}{mmol}
    \end{split}
    \label{Aliquot}
\end{align}

Für die Gesamtstoffmenge an Schwefelsäure wird die berechnete Stoffmenge des Aliquots mit vier multipliziert und nach Gleichung \eqref{Gesamt} berechnet.

\begin{align}
    \begin{split}
    n_\mathrm{ges} &= 4 \cdot n_\mathrm{Aliquot} \\
    &= 4 \cdot \SI{0,6105}{mmol} = \SI{2,442}{mmol}
    \end{split}
    \label{Gesamt}
\end{align}


\section{Fließumgebungen}

Das Konzept der Fließumgebungen (floating) basiert auf Umgebungen, deren Position im Dokument nicht durch ihre Position im Quelltext vorgegeben ist.
Sie beeinhalten Objekte wie Grafiken, Tabellen, Formeln und Gleichungen.
Der Positionsparameter [h] platziert das Gleitobjekt so nah wie möglich an derselben Stelle wie im Quelltext.
Der Parameter [t] platziert das Objekt bei der nächsten Möglichkeit an einem Seitenanfang.
Der Parameter [b] platziert es bei der nächsten Möglichkeit an einem Seitenende.
Und der Parameter [H] platziert das Objekt exakt an der Position von der PDF wie im Quelltext und verschiebt bei Notwendigkeit anderen Code Inhalt.

\section{Naturkonstanten}
Die Lichtgeschwindigkeit $c$ beträgt $\SI{2,99e8}{\mathrm{\frac{m}{s}}}$. \\
Die Elementarladung $e$ beträgt $e \approx \SI{1,602e-19}{\coulomb}$. \\
Das Plancksche Wirkungsquantum beträgt $e \approx \SI{6,626e-34}{\joule\second}$.

\listoffigures

\listoftables

\printbibliography[title={Quellenverzeichnis}]

\end{document}