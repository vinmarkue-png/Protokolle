\documentclass[a4paper,11pt]{scrartcl} 
\usepackage{geometry} 
\geometry{left=2.5cm} \geometry{top=3cm}

\usepackage[utf8]{inputenc} 
\usepackage[ngerman]{babel} 
\usepackage[T1]{fontenc} 
\usepackage{amsmath} 
\usepackage{amssymb} 
\usepackage[onehalfspacing]{setspace} 
\usepackage{graphicx} 
\usepackage{epstopdf} 
\usepackage{csquotes} 
\usepackage{array} 
\usepackage{upgreek} 
\usepackage{float} 
\usepackage{tikz} 
\usepackage[font=small]{caption}
\usepackage[backend=biber, style=chem-angew]{biblatex} 
\addbibresource{Blatt 7.bib} 
\usepackage{tabularx, booktabs, multirow} 
\usepackage{listings}
\usepackage{chemgreek}
\usepackage{chemformula}
\captionsetup{format=plain}
%\usepackage{hyperref}
\usepackage[hidelinks]{hyperref}
\usepackage{siunitx}
\sisetup{detect-weight=true, detect-family=true,locale=DE,range-phrase={\,bis\,},list-final-separator ={\,\linebreak[0] \text{und}\,},separate-uncertainty=true,per-mode = symbol-or-fraction}

\parindent0pt %Kein Einzug am Anfang von Absätzen
\sloppy %Besserer Blocksatz
%\renewcaptionname{ngerman}{\figurename}{Abb.} 
%\renewcaptionname{ngerman}{\tablename}{Tab.} 
\setkomafont{section}{\normalsize}

\lstset{
    language=C++,
    numbers=left,
    numberstyle=\tiny\color{gray},
    numbersep=8pt,
    stepnumber=1,
    frame=single,
    tabsize=4,
    showstringspaces=false,
    breaklines=true,
    keywordstyle=\color{blue}\bfseries,
    stringstyle=\color{red},
    commentstyle=\color{green!50!black},
    basicstyle=\ttfamily\small,
    literate={ö}{{\"o}}1
             {ä}{{\"a}}1
             {ü}{{\"u}}1
             {ß}{{\ss}}1
}

\title{Blatt 12}
\author{Vincent Kümmerle und Elvis Gnaglo}
\date{\today}

\begin{document}

\maketitle

\section{Datentypen}

\begin{lstlisting}
#include <iostream>
#include <string>
#include <cmath>
#include <typeinfo>

int main() {
    auto v1 = 3 + 5; // Ganzzahl-Addition
    std::cout << "3 + 5 = " << v1 << " | Typ: " << typeid(v1).name() << std::endl;

    auto v2 = 3 + 5.0; // Misch-Addition
    std::cout << "3 + 5.0 = " << v2 << " | Typ: " << typeid(v2).name() << std::endl;

    // "3" + "5" würde einen Compilerfehler verursachen (Zeiger-Addition)

    auto v4 = std::string("3") + "5"; // String-Zusammenfügung
    std::cout << "std::string(\"3\") + \"5\" = " << v4 << " | Typ: " << typeid(v4).name() << std::endl;

    auto v5 = 3 / 2; // Ganzzahl-Division
    std::cout << "3 / 2 = " << v5 << " | Typ: " << typeid(v5).name() << std::endl;

    auto v6 = 3.0 / 2; // Gleitkomma-Division
    std::cout << "3.0 / 2 = " << v6 << " | Typ: " << typeid(v6).name() << std::endl;

    auto v7 = int(2.71828); // Explizite Typumwandlung von Kommazahl zu Ganzzahl
    std::cout << "int(2.71828) = " << v7 << " | Typ: " << typeid(v7).name() << std::endl;

    auto v8 = std::round(2.71828); // Mathematisches Runden
    std::cout << "std::round(2.71828) = " << v8 << " | Typ: " << typeid(v8).name() << std::endl;

    return 0;
}
}
\end{lstlisting}
Output:
int = Ganzzahl, double = Kommazahl
\begin{lstlisting}
3 + 5 = 8 | Typ: int
3 + 5.0 = 8 | Typ: double
std::string("3") + "5" = 35 | Typ: class std::basic_string<char,struct std::char_traits<char>,class std::allocator<char> > 
    // C++ String-Objekt 
3 / 2 = 1 | Typ: int
3.0 / 2 = 1.5 | Typ: double
int(2.71828) = 2 | Typ: int
std::round(2.71828) = 3 | Typ: double
\end{lstlisting}


\section{}

\begin{lstlisting}

\end{lstlisting}


\end{document}