\documentclass[a4paper,12pt,bibliography=totocnumbered]{scrartcl}

\usepackage[utf8]{inputenc} 
\usepackage[T1]{fontenc}
\usepackage[english]{babel}
\usepackage{amsmath, amssymb,amsfonts}
\usepackage{graphicx}
\usepackage{csquotes}
\usepackage[bookmarks,colorlinks=true]{hyperref}
\usepackage{geometry}
\usepackage{float}
\usepackage[final]{pdfpages}
\usepackage{framed, color} 
\usepackage{scrlayer-scrpage}
\usepackage{siunitx}
\usepackage{subcaption}
\renewcaptionname{english}{\figurename}{Fig.}
\renewcaptionname{english}{\tablename}{Tab.}
\sisetup{
    detect-weight=true, 
    detect-family=true,
    locale=UK,
    exponent-product = \cdot,
    range-phrase={\,bis\,},
    list-final-separator ={\,\linebreak[0] \text{and}\,},
    separate-uncertainty=true,
    per-mode = symbol-or-fraction
}
%macht komata anstatt kreuze bei Zehnerpotenzen

\DeclareSIUnit{\angstrom}{\text{\AA}}
\usepackage[backend=biber, style=chem-angew]{biblatex} 
\addbibresource{lit.bib} 

\usepackage{chemgreek}
\usepackage{chemformula}
\geometry{left = 2.5cm} \geometry{top = 3cm}

\urlstyle{same}
%Hyperlinks-Setup
\hypersetup{
	colorlinks,
	linktocpage,
	citecolor=black,
	filecolor=black,
	linkcolor=black,
	urlcolor=black
}

%\numberwithin{equation}{section}

\setlength{\parindent}{0 mm}
\setlength{\parskip}{2 mm} 



\pagestyle{scrheadings}
%Header oben links auf linker Seite (ungerade Seitenzahl) und oben rechts auf rechter Seite (gerade Seitenzahl), beinhaltet gruppennummer und Versuchskürzel. Im Fall eine einseitigen Dokuments: Header oben rechts
\ihead{\VERSUCHSNR} %Header oben rechts auf linker Seite und oben links auf rechter Seite. Beinhaltet die Namen der Verfasser. Im Fall eine einseitigen Dokuments: Header oben links!
\ohead{\GRUPPENNR}
\ofoot{\thepage} 
\cfoot{\empty}  
\ifoot{\empty} 


\newcommand{\VERSUCHSDATUM}{21.01.2026}
\newcommand{\PROTOKOLLDATUM}{\today}

\newcommand{\VerfasserEINS}{Vincent Kümmerle}
\newcommand{\MatNoEINS}{3712667}
\newcommand{\EmailEINS}{st187541@stud.uni-stuttgart.de}
\newcommand{\StudiengangEINS}{B.Sc. Chemie}

\newcommand{\VerfasserZWEI}{Elvis Gnaglo}
\newcommand{\MatNoZWEI}{3710504}
\newcommand{\EmailZWEI}{st189318@stud.uni-stuttgart.de}
\newcommand{\StudiengangZWEI}{B.Sc. Chemie}

\newcommand{\VerfasserDREI}{Julian Brügger}
\newcommand{\MatNoDREI}{3715444}
\newcommand{\EmailDREI}{st190050@stud.uni-stuttgart.de}
\newcommand{\StudiengangDREI}{B.Sc. Chemie}

\newcommand{\BETREUER}{Mansha Shafquath}
\newcommand{\GRUPPENNR}{A05}

\newcommand{\VERSUCHSNR}{IR + Raman}
\newcommand{\VERSUCHSNAME}{IR- and Raman-Spectroscopy}


\begin{document}
\thispagestyle{empty}


\begin{titlepage}

\begin{center}
\Huge{\textbf{\VERSUCHSNR\ - \VERSUCHSNAME}}\\
\vspace{10mm}% Abstand
\Large{Protocol for the PC 2 lab course by \\ \textbf{\VerfasserEINS\;\& \VerfasserZWEI\;\& \VerfasserDREI}}\\
\vspace{10mm} 
\Large{University of Stuttgart}\\
\end{center}
\vspace{0cm}
\begin{center}
\begin{tabular}{ll}
\large{authors:}		& \large{\VerfasserEINS,} \large{\MatNoEINS} \\
 						& \large{\EmailEINS} \\
						\vspace{0cm}\\
						& \large{\VerfasserZWEI,} \large{\MatNoZWEI} \\
                        & \large{\EmailZWEI} \\
						\vspace{0cm}\\
						& \large{\VerfasserDREI,} \large{\MatNoDREI} \\
                        & \large{\EmailDREI} \\
						\vspace{0cm}\\
\large{group number:}	& \large{\GRUPPENNR} \\
\vspace{0cm}\\
\large{date of experiment:}	& \large{\VERSUCHSDATUM} \\
\vspace{0cm}\\
\large{supervisor:}		& \large{\BETREUER} \\
\vspace{0cm}\\
\large{submission date:} & \large{\PROTOKOLLDATUM}
\end{tabular}
\end{center}

\vspace{1cm}


\textbf{Abstract:}

\end{titlepage}


\thispagestyle{empty}

\tableofcontents 

\clearpage

\renewcommand{\thepage}{\arabic{page}}
\setcounter{page}{1}


\section{Theory}

\subsection{IR-Spectroscopy}

When a dipolar molecule is exposed to an electromagnetic field  the partially positive charged atom is pushed along the direction of the magnetic field lines. So if a molecule is exposed to an oscillating field, that oscillates at the natural vibration frequency of said molecule, the molecule will be in an excited vibrational state.

\subsection{Raman-Spectroscopy}

\subsection{DFT-Calculations}


\supercite{Skript}






\newpage
\section{Procedure}
To simulate and calculate the vibrational normal modes, the program \texttt{Avogadro2} was used. The structures of the molecules methane, chloromethane, dichloromethane, dibromomethane, chloroform, deuterated chloroform, tetrachloromethane and tetrachloroethylene were built, their geometry was optimized and the optimized coordinates were used to calculate the vibrational modes with the \texttt{ORCA} software, resulting in a list of IR and Raman frequencies and intensities for each molecule. \\
In the experimental part, the Raman spectra of dichloromethane, dibromomethane, chloroform, deuterated chloroform, tetrachloromethane and tetrachloroethylene were measured and analysized with the \texttt{WPenlighten} software. The IR spectra of dichloro-\\methane, dibromomethane, chloroform and tetrachloroethylene were measured using an ATR spectrometer and analysized with the \texttt{Opus} software.


\section{Results and Analysis}

\subsection{Methane}

\subsubsection{IR}
The calculated vibrational modes of methane are summarized in \autoref{tab: CH4} with the corresponding wavenumber, intensitiy and vibration type of each mode. 

\begin{table}[H]
    \centering
    \caption{Listed are the calculated wavenumbers and intensities of the IR signals of \ch{CH4} with the corresponding type of the vibrational mode.}
    \begin{tabular}{c|c|c|c}
    signal & wavenumber $\tilde{\nu}$ / cm$^{-1}$ & intensity / KM$\cdot\mathrm{mol^{-1}}$ & vibration type \\
    \hline
    1 & 1313.45 & 13.30 & asym. bending \\ 
    2 & 1313.68 & 13.25 & asym. bending \\
    3 & 1313.73 & 13.25 & asym. bending \\
    4 & 1530.79 & 0 & sym. bending \\
    5 & 1531.05 & 0 & sym. bending \\
    6 & 3019.38 & 0 & sym. stretching \\
    7 & 3152.03 & 17.69 & asym. stretching \\
    8 & 3152.33 & 17.64 & asym. stretching \\
    9 & 3152.45 & 17.64 & asym. stretching 
    \end{tabular}
    \label{tab: CH4}
\end{table}
As can be seen in \autoref{tab: CH4}, only the asymmetric bending and stretching modes are IR-active, while the symmetric bending and stretching modes are IR-inactive.
Furthermore, the asymmetric stretching mode shows the highest wavenumber among the IR-active modes, meaning it requires the most energy to be excited. 

\subsubsection{Raman}
The calculated Raman-active vibrational modes of methane are summarized in \autoref{tab: CH4_R} with the corresponding wavenumber, Raman intensity and vibration type of each mode.
\begin{table}[H]
    \centering
    \caption{Listed are the calculated wavenumbers and intensities of the Raman signals of \ch{CH4} with the corresponding type of the vibrational mode.}
    \begin{tabular}{c|c|c|c}
    signal & wavenumber $\tilde{\nu}$ / cm$^{-1}$ & Raman intensity / \AA$^4\cdot\mathrm{amu^{-1}}$ & vibration type \\
    \hline
    1 & 1313.38 & 1.64419 & asym. bending \\ 
    2 & 1313.61 & 1.6422 & asym. bending \\
    3 & 1314.1 & 1.6484 & asym. bending \\
    4 & 1531.00 & 27.4565 & sym. bending \\
    5 & 1531.09 & 27.449 & sym. bending \\
    6 & 3019.41 & 145.177 & sym. stretching \\
    7 & 3150.24 & 62.8181 & asym. stretching \\
    8 & 3150.27 & 62.8724 & asym. stretching \\
    9 & 3150.79 & 62.8305 & asym. stretching 
    \end{tabular}
    \label{tab: CH4_R}
\end{table}
In contrast to the IR spectrum, both the symmetric bending and stretching modes are Raman-active. The symmetric stretching mode shows the highest Raman intensity among all vibrational modes, indicating that it is most prominent mode to be observed in a Raman spectrum.

\newpage
\subsection{Chloromethane}

\subsubsection{IR}


\subsubsection{Raman}


\subsection{Dichloromethane}

\subsubsection{IR}
The measured IR spectrum of dichloromethane is shown in \autoref{fig: CH2Cl2_IR}, plotting the intensity of the absorption against the wavenumber $\tilde{\nu}$.


\begin{figure}[H]
    \centering
    \includegraphics[width=0.8\textwidth]{IR/CH2Cl2_spectrum.pdf}
    \caption{Measured IR spectrum of dichloromethane.} 
    \label{fig: CH2Cl2_IR}
\end{figure}



% --- LaTeX Tabelle für CH2Cl2 ---
\begin{table}[H]
    \centering
    \caption{Listed are the measured wavenumbers and intensities of the IR signals of \ch{CH2Cl2}.}
    \begin{tabular}{c|c|c}
    signal & wavenumber $\tilde{\nu}$ / cm$^{-1}$ & intensity / a.u. \\
    \hline
    1 & 704.00 & 0.45 \\
    2 & 730.53 & 1.01 \\
    3 & 895.82 & 0.04 \\
    4 & 1265.17 & 0.29 \\
    5 & 1422.29 & ~0.02
    \label{tab: CH2Cl2}
    \end{tabular}
\end{table}

\begin{table}[H]
    \centering
    \caption{Listed are the simulated wavenumbers and intensities of the vibrational modes of \ch{CH2Cl2}.}
    \begin{tabular}{c|c|c}
    Mode & Wavenumber $\tilde{\nu}$ / cm$^{-1}$ & Intensity / KM$\cdot\mathrm{mol^{-1}}$ \\
    \hline
    1 & 277.23 & 0.64 \\
    2 & 703.86 & 14.19 \\
    3 & 733.80 & 137.83 \\
    4 & 889.17 & 1.20 \\
    5 & 1153.54 & 0.00 \\
    6 & 1272.86 & 41.21 \\
    7 & 1441.46 & 0.01 \\
    8 & 3107.43 & 9.81 \\
    9 & 3194.30 & 0.64 
    \label{tab: CH2Cl2_sim}
    \end{tabular}
\end{table}





\subsubsection{Raman}

\begin{figure}[H]
    \centering
    \includegraphics[width=0.8\textwidth]{Raman/CH2Cl2_raman_spectrum.pdf}
    \caption{Measured raman spectrum of dichloromethane.} 
    \label{fig: CH2Cl2_Ram}
\end{figure}

\begin{table}[H]
    \centering
    \caption{Listed are the measured Raman shifts and intensities of the signals of \ch{CH2Cl2}.}
    \begin{tabular}{c|c|c}
    signal & Raman Shift $\Delta \tilde{\nu}$ / cm$^{-1}$ & intensity / a.u. \\
    \hline
    1 & 281.99 & 2542.67 \\
    2 & 697.77 & 4210.67 \\
    3 & 1418.07 & 1563.33 \\
    4 & 2984.85 & 5336.00 \\
    5 & 3051.13 & ~1775.33
    \label{tab: raman_CH2Cl2}
    \end{tabular}
\end{table}

\begin{table}[H]
    \centering
    \caption{Listed are the simulated wavenumbers and raman intensities of the vibrational modes of \ch{CH2Cl2}.}
    \begin{tabular}{c|c|c}
    Mode & Wavenumber $\tilde{\nu}$ / cm$^{-1}$ & Raman Intensity / \AA$^4$ amu$^{-1}$ \\
    \hline
    1 & 277.06 & 6.83 \\
    2 & 703.48 & 12.27 \\
    3 & 732.67 & 5.02 \\
    4 & 888.90 & 3.13 \\
    5 & 1153.83 & 11.78 \\
    6 & 1272.67 & 3.01 \\
    7 & 1441.64 & 12.42 \\
    8 & 3106.65 & 108.70 \\
    9 & 3193.09 & 62.65 
    \label{tab: raman_CH2Cl2_sim}
    \end{tabular}
\end{table}

\subsection{Dibromomethane}

\subsubsection{IR}
The measured IR spectrum of dibromomethane is shown in \autoref{fig: CH2Br2_IR}, plotting the intensity of the absorption against the wavenumber $\tilde{\nu}$.
\begin{figure}[H]
    \centering
    \includegraphics[width=0.8\textwidth]{IR/CH2Br2_spectrum.pdf}
    \caption{Measured IR spectrum of dibromomethane.} 
    \label{fig: CH2Br2_IR}
\end{figure}


% --- LaTeX Tabelle für CH2Br2 ---
\begin{table}[H]
    \centering
    \caption{Listed are the measured wavenumbers and intensities of the IR signals of \ch{CH2Br2}.}
    \begin{tabular}{c|c|c}
    signal & wavenumber $\tilde{\nu}$ / cm$^{-1}$ & intensity / a.u. \\
    \hline
    1 & 455.05 & 0.52 \\
    2 & 577.49 & 0.49 \\
    3 & 632.58 & 0.95 \\
    4 & 677.48 & 0.23 \\
    5 & 812.16 & 0.23 \\
    6 & 1095.80 & 0.08 \\
    7 & 1189.66 & 0.53 \\
    8 & 1389.64 & 0.02 \\
    11 & 3064.97 & ~0.05
    \label{tab: CH2Br2}
    \end{tabular}
\end{table}

\begin{table}[H]
    \centering
    \caption{Listed are the simulated wavenumbers and intensities of the vibrational modes of \ch{CH2Br2}.}
    \begin{tabular}{c|c|c}
    Mode & Wavenumber $\tilde{\nu}$ / cm$^{-1}$ & Intensity / KM$\cdot\mathrm{mol^{-1}}$ \\
    \hline
    1 & 168.72 & 0.08 \\
    2 & 573.58 & 4.08 \\
    3 & 628.31 & 98.95 \\
    4 & 806.07 & 4.64 \\
    5 & 1101.92 & 0.00 \\
    6 & 1205.80 & 65.32 \\
    7 & 1412.95 & 0.00 \\
    8 & 3126.16 & 1.92 \\
    9 & 3221.84 & 1.28 
    \label{tab: CH2Br2_sim}
    \end{tabular}
\end{table}

\subsubsection{Raman}

\begin{figure}[H]
    \centering
    \includegraphics[width=0.8\textwidth]{Raman/CH2Br2_raman_spectrum.pdf}
    \caption{Measured raman spectrum of dibromomethane.} 
    \label{fig: CH2Br2_Ram}
\end{figure}

\begin{table}[H]
    \centering
    \caption{Listed are the measured Raman shifts and intensities of the signals of \ch{CH2Br2}.}
    \begin{tabular}{c|c|c}
    signal & Raman Shift $\Delta \tilde{\nu}$ / cm$^{-1}$ & intensity / a.u. \\
    \hline
    1 & 169.61 & 4410.67 \\
    2 & 574.79 & 6711.00 \\
    3 & 634.60 & 2364.33 \\
    4 & 1387.16 & 1889.33 \\
    5 & 2432.07 & 1409.00 \\
    6 & 2984.85 & 5162.00 \\
    7 & 3062.11 & ~1965.67
    \label{tab: raman_CH2Br2}
    \end{tabular}
\end{table}

\begin{table}[H]
    \centering
    \caption{Listed are the simulated wavenumbers and raman intensities of the vibrational modes of \ch{CH2Br2}.}
    \begin{tabular}{c|c|c}
    Mode & Wavenumber $\tilde{\nu}$ / cm$^{-1}$ & Raman Intensity / \AA$^4$ amu$^{-1}$ \\
    \hline
    1 & 168.56 & 5.37 \\
    2 & 574.64 & 13.43 \\
    3 & 629.71 & 5.36 \\
    4 & 806.60 & 2.41 \\
    5 & 1102.17 & 8.43 \\
    6 & 1205.71 & 0.74 \\
    7 & 1413.22 & 13.77 \\
    8 & 3125.63 & 97.24 \\
    9 & 3221.31 & 58.05 
    \label{tab: raman_CH2Br2_raman}
    \end{tabular}
\end{table}

\subsection{Chloroform}

\subsubsection{IR}
The measured IR spectrum of chloroform is shown in \autoref{fig: CHCl3_IR}, plotting the intensity of the absorption against the wavenumber $\tilde{\nu}$.


\begin{figure}[H]
    \centering
    \includegraphics[width=0.8\textwidth]{IR/CHCl3_spectrum.pdf}
    \caption{Measured IR spectrum of chloroform.} 
    \label{fig: CHCl3_IR}
\end{figure}

\begin{table}[H]
    \centering
    \caption{Listed are the measured wavenumbers and intensities of the IR signals of \ch{CHCl3}.}
    \begin{tabular}{c|c|c}
    signal & wavenumber $\tilde{\nu}$ / cm$^{-1}$ & intensity / a.u. \\
    \hline
    1 & 626.46 & 0.11 \\
    2 & 667.27 & 0.27 \\
    3 & 742.78 & 1.58 \\
    4 & 910.10 & 0.03 \\
    5 & 928.47 & 0.02 \\
    6 & 1214.15 & 0.30 \\
    7 & 3020.07 & ~0.02
    \label{tab: CHCl3}
    \end{tabular}
\end{table}

\begin{table}[H]
    \centering
    \caption{Listed are the simulated wavenumbers and intensities of the vibrational modes of \ch{CHCl3}.}
    \begin{tabular}{c|c|c}
    Mode & Wavenumber $\tilde{\nu}$ / cm$^{-1}$ & Intensity / KM$\cdot\mathrm{mol^{-1}}$ \\
    \hline
    1 & 254.78 & 0.06 \\
    2 & 254.97 & 0.06 \\
    3 & 362.33 & 0.46 \\
    4 & 665.85 & 7.26 \\
    5 & 741.92 & 167.74 \\
    6 & 742.13 & 167.67 \\
    7 & 1220.08 & 22.80 \\
    8 & 1220.17 & 22.76 \\
    9 & 3169.43 & 0.22
    \label{tab: CHCl3_sim}
    \end{tabular}
\end{table}

\subsubsection{Raman}

\begin{figure}[H]
    \centering
    \includegraphics[width=0.8\textwidth]{Raman/CHCl3_raman_spectrum.pdf}
    \caption{Measured raman spectrum of chloroform.} 
    \label{fig: CHCl3_Ram}
\end{figure}

\begin{table}[H]
    \centering
    \caption{Listed are the measured Raman shifts and intensities of the signals of \ch{CHCl3}.}
    \begin{tabular}{c|c|c}
    signal & Raman Shift $\Delta \tilde{\nu}$ / cm$^{-1}$ & intensity / a.u. \\
    \hline
    1 & 258.84 & 2927.00 \\
    2 & 362.65 & 3127.33 \\
    3 & 664.38 & 3645.00 \\
    4 & 756.86 & 1811.33 \\
    5 & 1213.44 & 1437.33 \\
    6 & 3015.32 & ~3161.67
    \label{tab: raman_CHCl3}
    \end{tabular}
\end{table}

\begin{table}[H]
    \centering
    \caption{Listed are the simulated wavenumbers and raman intensities of the vibrational modes of \ch{CHCl3}.}
    \begin{tabular}{c|c|c}
    Mode & Wavenumber $\tilde{\nu}$ / cm$^{-1}$ & Raman Intensity / \AA$^4$ amu$^{-1}$ \\
    \hline
    1 & 254.60 & 5.14 \\
    2 & 255.07 & 5.13 \\
    3 & 362.22 & 8.69 \\
    4 & 665.69 & 9.80 \\
    5 & 740.88 & 3.08 \\
    6 & 741.45 & 3.07 \\
    7 & 1220.22 & 6.04 \\
    8 & 1220.68 & 6.05 \\
    9 & 3168.77 & 77.28 
    \label{tab: raman_CHCl3_sim}
    \end{tabular}
\end{table}

\subsection{Deuterated Chloroform}

\subsubsection{IR}
\begin{table}[H]
    \centering
    \caption{Listed are the simulated wavenumbers and intensities of the vibrational modes of \ch{CDCl3}.}
    \begin{tabular}{c|c|c}
    Mode & Wavenumber $\tilde{\nu}$ / cm$^{-1}$ & Intensity / KM$\cdot\mathrm{mol^{-1}}$ \\
    \hline
    1 & 253.69 & 0.06 \\
    2 & 253.88 & 0.06 \\
    3 & 360.12 & 0.50 \\
    4 & 646.14 & 6.66 \\
    5 & 717.83 & 125.46 \\
    6 & 717.98 & 125.28 \\
    7 & 909.65 & 63.16 \\
    8 & 909.66 & 63.19 \\
    9 & 2342.61 & 0.74 
    \label{tab: CDCl3_sim}
    \end{tabular}
\end{table}

\subsubsection{Raman}

\begin{figure}[H]
    \centering
    \includegraphics[width=0.8\textwidth]{Raman/CDCl3_raman_spectrum.pdf}
    \caption{Measured raman spectrum of deuterated chloroform.} 
    \label{fig: CDCl3_Ram}
\end{figure}

\begin{table}[H]
    \centering
    \caption{Listed are the measured Raman shifts and intensities of the signals of \ch{CDCl3}.}
    \begin{tabular}{c|c|c}
    signal & Raman Shift $\Delta \tilde{\nu}$ / cm$^{-1}$ & intensity / a.u. \\
    \hline
    1 & 254.97 & 2866.67 \\
    2 & 358.83 & 3009.00 \\
    3 & 645.78 & 3616.67 \\
    4 & 731.05 & 1834.67 \\
    5 & 903.05 & 1434.67 \\
    6 & 2248.93 & ~2796.00
    \label{tab: raman_CDCl3}
    \end{tabular}
\end{table}

\begin{table}[H]
    \centering
    \caption{Listed are the simulated wavenumbers and raman intensities of the vibrational modes of \ch{CDCl3}.}
    \begin{tabular}{c|c|c}
    Mode & Wavenumber $\tilde{\nu}$ / cm$^{-1}$ & Raman Intensity / \AA$^4$ amu$^{-1}$ \\
    \hline
    1 & 254.60 & 5.14 \\
    2 & 255.07 & 5.13 \\
    3 & 362.22 & 8.69 \\
    4 & 665.69 & 9.80 \\
    5 & 740.88 & 3.08 \\
    6 & 741.45 & 3.07 \\
    7 & 1220.22 & 6.04 \\
    8 & 1220.68 & 6.05 \\
    9 & 3168.77 & 77.28 
    \label{tab: raman_CDCl3_sim}
    \end{tabular}
\end{table}

\subsection{Tetrachloromethane}

\subsubsection{IR}
\begin{table}[H]
    \centering
    \caption{Listed are the simulated wavenumbers and intensities of the vibrational modes of \ch{CCl4}.}
    \begin{tabular}{c|c|c}
    Mode & Wavenumber $\tilde{\nu}$ / cm$^{-1}$ & Intensity / KM$\cdot\mathrm{mol^{-1}}$ \\
    \hline
    1 & 212.71 & 0.00 \\
    2 & 212.89 & 0.00 \\
    3 & 310.72 & 0.06 \\
    4 & 310.82 & 0.06 \\
    5 & 310.88 & 0.06 \\
    6 & 451.20 & 0.00 \\
    7 & 754.80 & 185.52 \\
    8 & 755.02 & 185.61 \\
    9 & 755.51 & 185.58 
    \label{tab: CCl4_sim}
    \end{tabular}
\end{table}

\subsubsection{Raman}

\begin{figure}[H]
    \centering
    \includegraphics[width=0.8\textwidth]{Raman/CCl4_raman_spectrum.pdf}
    \caption{Measured raman spectrum of tetrachloromethane.} 
    \label{fig: CCl4_Ram}
\end{figure}

\begin{table}[H]
    \centering
    \caption{Listed are the measured Raman shifts and intensities of the signals of \ch{CCl4}.}
    \begin{tabular}{c|c|c}
    signal & Raman Shift $\Delta \tilde{\nu}$ / cm$^{-1}$ & intensity / a.u. \\
    \hline
    1 & 216.25 & 2831.33 \\
    2 & 308.95 & 3211.67 \\
    3 & 454.10 & 4535.33 \\
    4 & 756.86 & ~1849.33
    \label{tab: raman_CCl4}
    \end{tabular}
\end{table}

\begin{table}[H]
    \centering
    \caption{Listed are the simulated wavenumbers and Raman intensities of the vibrational modes of \ch{CCl4}.}
    \begin{tabular}{c|c|c}
    Mode & Wavenumber $\tilde{\nu}$ / cm$^{-1}$ & Raman Intensity / \AA$^4$ amu$^{-1}$ \\
    \hline
    1 & 212.71 & 4.18 \\
    2 & 212.91 & 4.17 \\
    3 & 310.52 & 5.39 \\
    4 & 310.84 & 5.40 \\
    5 & 311.16 & 5.40 \\
    6 & 451.08 & 16.40 \\
    7 & 753.44 & 1.55 \\
    8 & 754.26 & 1.56 \\
    9 & 754.60 & 1.54 
    \label{tab: raman_CCl4_sim}
    \end{tabular}
\end{table}

\subsection{Tetrachloroethylene}

\subsubsection{IR}
The measured IR spectrum of tetrachloroethylene is shown in \autoref{fig: C2Cl4_IR}, plotting the intensity of the absorption against the wavenumber $\tilde{\nu}$.

\begin{figure}[H]
    \centering
    \includegraphics[width=0.8\textwidth]{IR/C2Cl4_spectrum.pdf}
    \caption{Measured IR spectrum of tetrachloroethylene.} 
    \label{fig: C2Cl4_IR}
\end{figure}
By visual inspection of the IR spectrum in \autoref{fig: C2Cl4_IR}, six absorption signals can be identified, which are listed with their corresponding wavenumbers and intensities in \autoref{tab: C2Cl4}.
% --- LaTeX Tabelle für C2Cl4 ---
\begin{table}[H]
    \centering
    \caption{Listed are the measured wavenumbers and intensities of the IR signals of \ch{C2Cl4} with the corresponding type of the vibrational mode.}
    \begin{tabular}{c|c|c|c}
    signal & wavenumber $\tilde{\nu}$ / cm$^{-1}$ & intensity / a.u. & vibration type \\
    \hline
    1 & 755.02 & 0.18 & \\
    2 & 775.42 & 0.51 & asym. C-Cl stretching \\
    3 & 799.91 & 0.18 & \\
    4 & 903.98 & 0.88 & asym. C-Cl stretching \\
    5 & 1122.32 & 0.03 & \\
    6 & 1354.95 & ~0.01 & 
    \end{tabular}
    \label{tab: C2Cl4}
\end{table}



The calculated vibrational modes of tetrachloroethylene are summarized in \autoref{tab: C2Cl4_sim} with the corresponding wavenumber, intensitiy and vibration type of each mode. 
\begin{table}[H]
    \centering
    \caption{Listed are the simulated wavenumbers and intensities of the vibrational modes of \ch{C2Cl4}.}
    \begin{tabular}{c|c|c|c}
    Mode & Wavenumber $\tilde{\nu}$ / cm$^{-1}$ & Intensity / KM$\cdot\mathrm{mol^{-1}}$ & vibration type \\
    \hline
    1 & 97.18 & 0.00 & \\
    2 & 174.89 & 0.96 & \\
    3 & 234.77 & 0.00 & \\
    4 & 286.90 & 0.51 & \\
    5 & 310.42 & 0.03 & \\
    6 & 342.99 & 0.00 & \\
    7 & 447.02 & 0.00 & \\
    8 & 514.19 & 0.00 & \\
    9 & 774.46 & 82.16 & asym. C-Cl stretching \\
    10 & 895.52 & 202.05 & asym. C-Cl stretching \\
    11 & 978.50 & 0.00 & 
    \label{tab: C2Cl4_sim}
    \end{tabular}
\end{table}
As can be seen in \autoref{tab: C2Cl4_sim}, the most intense IR-active modes are found at wavenumbers of 774.46 cm$^{-1}$ and 895.52 cm$^{-1}$, which correspond well to the measured signals at 755.02 cm$^{-1}$ and 903.98 cm$^{-1}$ in \autoref{fig: C2Cl4_IR}.

In comparison to the simulated IR spectrum of chloroform with the wavenumbers and vibrational modes in \autoref{tab: CHCl3}, tetrachloroethylene shows additional IR-active modes in the low wavenumber region below 500 cm$^{-1}$, which can be attributed to the increased number of atoms in the molecule leading to more vibrational modes.

\subsubsection{Raman}

\begin{figure}[H]
    \centering
    \includegraphics[width=0.8\textwidth]{Raman/C2Cl4_raman_spectrum.pdf}
    \caption{Measured raman spectrum of tetrachloroethylene.} 
    \label{fig: C2Cl4_Ram}
\end{figure}

\begin{table}[H]
    \centering
    \caption{Listed are the measured Raman shifts and intensities of the signals of \ch{C2Cl4}.}
    \begin{tabular}{c|c|c|c}
    signal & Raman Shift $\Delta \tilde{\nu}$ / cm$^{-1}$ & intensity / a.u. & vibration type \\
    \hline
    1 & 231.76 & 2776.67 & \\
    2 & 339.67 & 1879.00 & \\
    3 & 442.71 & 4210.33 & \\
    4 & 1567.61 & 4313.00 & \\
    5 & 2432.07 & ~2068.33 & 
    \label{tab: raman_C2Cl4}
    \end{tabular}
\end{table}

\begin{table}[H]
    \centering
    \caption{Listed are the simulated wavenumbers and Raman intensities of the vibrational modes of \ch{C2Cl4}.}
    \begin{tabular}{c|c|c|c}
    Mode & Wavenumber $\tilde{\nu}$ / cm$^{-1}$ & Raman Intensity / \AA$^4$ amu$^{-1}$ & vibration type \\
    \hline
    1 & 97.98 & 0.00 & \\
    2 & 174.79 & 0.00 & \\
    3 & 234.62 & 5.56 & \\
    4 & 289.22 & 0.00 & \\
    5 & 310.02 & 0.00 & \\
    6 & 342.83 & 4.61 & \\
    7 & 446.81 & 15.61 & \\
    8 & 517.38 & 3.22 & \\
    9 & 774.24 & 0.00 & asym. C-Cl stretching \\
    10 & 895.62 & 0.00 & asym. C-Cl stretching \\
    11 & 978.54 & 0.44 & \\
    12 & 1623.90 & 48.63 & C=C stretching
    \label{tab: raman_C2Cl4_sim}
    \end{tabular}
\end{table}

In comparison to the calculated IR-active modes in \autoref{tab: C2Cl4_sim}, \autoref{tab: raman_C2Cl4_sim} shows that the most intense Raman-active mode is found at a wavenumber of 1623.90 cm$^{-1}$, which can be assigned to the $\mathrm{C=C}$ stretching mode of tetrachloroethylene, which is not present in the calculated IR spectrum. The cause for this lies in the rule of mutual exclusion, which applies to molecules with a center of symmetry, such as the inversion center of tetrachloroethylene. According to this rule, vibrational modes that are Raman-active are IR-inactive and vice versa, explaining the absence of the 1623.90 cm$^{-1}$ mode in the IR spectrum. By comparing the calculated Raman modes of \ch{C2Cl4} from \autoref{tab: raman_C2Cl4_sim} with the calculated Raman modes of \ch{CHCl3} in \autoref{tab: raman_CHCl3}, the structural differences between the two molecules can be explained. While both molecules show Raman-active modes in the low wavenumber region below 500 cm$^{-1}$, tetrachloroethylene exhibits an additional strong Raman-active mode at 1623.90 cm$^{-1}$, which can be attributed to the presence of the $\mathrm{C=C}$ double bond in \ch{C2Cl4} that is absent in \ch{CHCl3}. Additionally, the C-H stretching mode at 3168.77 cm$^{-1}$ present in chloroform is not observed in tetrachloroethylene due to the lack of hydrogen atoms in its structure.

\section{Discussion}


\section{Conclusion}


\printbibliography[title={References}]

\end{document}
