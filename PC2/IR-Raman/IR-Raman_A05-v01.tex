\documentclass[a4paper,12pt,bibliography=totocnumbered]{scrartcl}

\usepackage[utf8]{inputenc} 
\usepackage[T1]{fontenc}
\usepackage[english]{babel}
\usepackage{amsmath, amssymb,amsfonts}
\usepackage{graphicx}
\usepackage{csquotes}
\usepackage[bookmarks,colorlinks=true]{hyperref}
\usepackage{geometry}
\usepackage{float}
\usepackage[final]{pdfpages}
\usepackage{framed, color} 
\usepackage{scrlayer-scrpage}
\usepackage{siunitx}
\usepackage{subcaption}
%\renewcaptionname{ngerman}{\figurename}{Fig.}
%\renewcaptionname{ngerman}{\tablename}{Table}
\sisetup{
    detect-weight=true, 
    detect-family=true,
    locale=UK,
    exponent-product = \cdot,
    range-phrase={\,bis\,},
    list-final-separator ={\,\linebreak[0] \text{and}\,},
    separate-uncertainty=true,
    per-mode = symbol-or-fraction
}
%macht komata anstatt kreuze bei Zehnerpotenzen
\usepackage[backend=biber, style=chem-angew]{biblatex} 
\addbibresource{lit.bib} 

\usepackage{chemgreek}
\usepackage{chemformula}
\geometry{left = 2.5cm} \geometry{top = 3cm}

\urlstyle{same}
%Hyperlinks-Setup
\hypersetup{
	colorlinks,
	linktocpage,
	citecolor=black,
	filecolor=black,
	linkcolor=black,
	urlcolor=black
}

%\numberwithin{equation}{section}

\setlength{\parindent}{0 mm}
\setlength{\parskip}{2 mm} 



\pagestyle{scrheadings}
%Header oben links auf linker Seite (ungerade Seitenzahl) und oben rechts auf rechter Seite (gerade Seitenzahl), beinhaltet gruppennummer und Versuchskürzel. Im Fall eine einseitigen Dokuments: Header oben rechts
\ihead{\VERSUCHSNR} %Header oben rechts auf linker Seite und oben links auf rechter Seite. Beinhaltet die Namen der Verfasser. Im Fall eine einseitigen Dokuments: Header oben links!
\ohead{\GRUPPENNR}
\ofoot{\thepage} 
\cfoot{\empty}  
\ifoot{\empty} 


\newcommand{\VERSUCHSDATUM}{21.01.2026}
\newcommand{\PROTOKOLLDATUM}{\today}

\newcommand{\VerfasserEINS}{Vincent Kümmerle}
\newcommand{\MatNoEINS}{3712667}
\newcommand{\EmailEINS}{st187541@stud.uni-stuttgart.de}
\newcommand{\StudiengangEINS}{B.Sc. Chemie}

\newcommand{\VerfasserZWEI}{Elvis Gnaglo}
\newcommand{\MatNoZWEI}{3710504}
\newcommand{\EmailZWEI}{st189318@stud.uni-stuttgart.de}
\newcommand{\StudiengangZWEI}{B.Sc. Chemie}

\newcommand{\VerfasserDREI}{Julian Brügger}
\newcommand{\MatNoDREI}{3715444}
\newcommand{\EmailDREI}{st190050@stud.uni-stuttgart.de}
\newcommand{\StudiengangDREI}{B.Sc. Chemie}

\newcommand{\BETREUER}{Mansha Shafquath}
\newcommand{\GRUPPENNR}{A05}

\newcommand{\VERSUCHSNR}{IR + Raman}
\newcommand{\VERSUCHSNAME}{IR- and Raman-Spectroscopy}


\begin{document}
\thispagestyle{empty}


\begin{titlepage}

\begin{center}
\Huge{\textbf{\VERSUCHSNR\ - \VERSUCHSNAME}}\\
\vspace{10mm}% Abstand
\Large{Protocol for the PC 2 lab course by \\ \textbf{\VerfasserEINS\;\& \VerfasserZWEI\;\& \VerfasserDREI}}\\
\vspace{10mm} 
\Large{University of Stuttgart}\\
\end{center}
\vspace{0cm}
\begin{center}
\begin{tabular}{ll}
\large{authors:}		& \large{\VerfasserEINS,} \large{\MatNoEINS} \\
 						& \large{\EmailEINS} \\
						\vspace{0cm}\\
						& \large{\VerfasserZWEI,} \large{\MatNoZWEI} \\
                        & \large{\EmailZWEI} \\
						\vspace{0cm}\\
						& \large{\VerfasserDREI,} \large{\MatNoDREI} \\
                        & \large{\EmailDREI} \\
						\vspace{0cm}\\
\large{group number:}	& \large{\GRUPPENNR} \\
\vspace{0cm}\\
\large{date of experiment:}	& \large{\VERSUCHSDATUM} \\
\vspace{0cm}\\
\large{supervisor:}		& \large{\BETREUER} \\
\vspace{0cm}\\
\large{submission date:} & \large{\PROTOKOLLDATUM}
\end{tabular}
\end{center}

\vspace{1cm}


\textbf{Abstract:}

\end{titlepage}


\thispagestyle{empty}

\tableofcontents 

\clearpage

\renewcommand{\thepage}{\arabic{page}}
\setcounter{page}{1}


\section{Theory}


\supercite{Skript}






\newpage
\section{Procedure}
To simulate and calculate the vibrational normal modes, the program \texttt{Avogadro2} was used. The structures of the molecules methane, chloromethane, dichloromethane, dibromomethane, chloroform, deuterated chloroform, tetrachloromethane and tetrachloroethylene were built, their geometry was optimized and the optimized coordinates were used to calculate the vibrational modes with the \texttt{ORCA} software, resulting in a list of IR and Raman frequencies and intensities for each molecule. \\
In the experimental part, the Raman spectra of dichloromethane, dibromomethane, chloroform, deuterated chloroform, tetrachloromethane and tetrachloroethylene were measured and analysized with the \texttt{WPenlighten} software. The IR spectra of dichloro-\\methane, dibromomethane, chloroform and tetrachloroethylene were measured using an ATR spectrometer and analysized with the \texttt{Opus} software.


\section{Results and Analysis}

\subsection{Methane}

\subsubsection{IR}



\subsubsection{Raman}



\newpage
\subsection{Chloromethane}

\subsubsection{IR}


\subsubsection{Raman}


\subsection{Dichloromethane}

\subsubsection{IR}



\begin{figure}[H]
    \centering
    \includegraphics[width=1\textwidth]{IR/CH2Cl2_spectrum.pdf}
    \caption{Measured IR spectrum of dichloromethane.} 
    \label{fig: CH2Cl2_IR}
\end{figure}



% --- LaTeX Tabelle für CH2Cl2 ---
\begin{table}[H]
    \centering
    \caption{Listed are the measured wavenumbers and intensities of the IR signals of \ch{CH2Cl2}.}
    \begin{tabular}{c|c|c}
    signal & wavenumber $\tilde{\nu}$ / cm$^{-1}$ & intensity / a.u. \\
    \hline
    1 & 704.00 & 0.45 \\
    2 & 730.53 & 1.01 \\
    3 & 895.82 & 0.04 \\
    4 & 1265.17 & 0.29 \\
    5 & 1422.29 & ~0.02
    \label{tab: CH2Cl2}
    \end{tabular}
\end{table}







\subsubsection{Raman}


\subsection{Dibromomethane}

\subsubsection{IR}



\begin{figure}[H]
    \centering
    \includegraphics[width=1\textwidth]{IR/CH2Br2_spectrum.pdf}
    \caption{Measured IR spectrum of dibromomethane.} 
    \label{fig: CH2Br2_IR}
\end{figure}


% --- LaTeX Tabelle für CH2Br2 ---
\begin{table}[H]
    \centering
    \caption{Listed are the measured wavenumbers and intensities of the IR signals of \ch{CH2Br2}.}
    \begin{tabular}{c|c|c}
    signal & wavenumber $\tilde{\nu}$ / cm$^{-1}$ & intensity / a.u. \\
    \hline
    1 & 455.05 & 0.52 \\
    2 & 577.49 & 0.49 \\
    3 & 632.58 & 0.95 \\
    4 & 677.48 & 0.23 \\
    5 & 812.16 & 0.23 \\
    6 & 1095.80 & 0.08 \\
    7 & 1189.66 & 0.53 \\
    8 & 1389.64 & 0.02 \\
    11 & 3064.97 & ~0.05
    \label{tab: CH2Br2}
    \end{tabular}
\end{table}


\subsubsection{Raman}


\subsection{Chloroform}

\subsubsection{IR}



\begin{figure}[H]
    \centering
    \includegraphics[width=1\textwidth]{IR/CHCl3_spectrum.pdf}
    \caption{Measured IR spectrum of chloroform.} 
    \label{fig: CHCl3_IR}
\end{figure}

\begin{table}[H]
    \centering
    \caption{Listed are the measured wavenumbers and intensities of the IR signals of \ch{CHCl3}.}
    \begin{tabular}{c|c|c}
    signal & wavenumber $\tilde{\nu}$ / cm$^{-1}$ & intensity / a.u. \\
    \hline
    1 & 626.46 & 0.11 \\
    2 & 667.27 & 0.27 \\
    3 & 742.78 & 1.58 \\
    4 & 910.10 & 0.03 \\
    5 & 928.47 & 0.02 \\
    6 & 1214.15 & 0.30 \\
    7 & 3020.07 & ~0.02
    \label{tab: CHCl3}
    \end{tabular}
\end{table}



\subsubsection{Raman}


\subsection{Deuterated Chloroform}

\subsubsection{IR}


\subsubsection{Raman}


\subsection{Tetrachloromethane}

\subsubsection{IR}


\subsubsection{Raman}


\subsection{Tetrachloroethylene}

\subsubsection{IR}




\begin{figure}[H]
    \centering
    \includegraphics[width=1\textwidth]{IR/C2Cl4_spectrum.pdf}
    \caption{Measured IR spectrum of tetrachloroethylene.} 
    \label{fig: C2Cl4_IR}
\end{figure}

% --- LaTeX Tabelle für C2Cl4 ---
\begin{table}[H]
    \centering
    \caption{Listed are the measured wavenumbers and intensities of the IR signals of \ch{C2Cl4}.}
    \begin{tabular}{c|c|c}
    signal & wavenumber $\tilde{\nu}$ / cm$^{-1}$ & intensity / a.u. \\
    \hline
    1 & 755.02 & 0.18 \\
    2 & 775.42 & 0.51 \\
    3 & 799.91 & 0.18 \\
    4 & 903.98 & 0.88 \\
    5 & 1122.32 & 0.03 \\
    6 & 1354.95 & ~0.01
    \label{tab: C2Cl4}
    \end{tabular}
\end{table}


\subsubsection{Raman}



\section{Discussion}


\section{Conclusion}


\printbibliography[title={References}]

\end{document}
