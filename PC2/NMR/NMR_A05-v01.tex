\documentclass[a4paper,12pt,bibliography=totocnumbered]{scrartcl}

\usepackage[utf8]{inputenc} 
\usepackage[T1]{fontenc}
\usepackage[ngerman]{babel} 
\usepackage{amsmath, amssymb,amsfonts}
\usepackage{graphicx}
\usepackage{csquotes}
\usepackage[bookmarks,colorlinks=true]{hyperref}
\usepackage{geometry}
\usepackage{float}
\usepackage[final]{pdfpages}
\usepackage{framed, color} 
\usepackage{scrlayer-scrpage}
\usepackage{siunitx}
\usepackage{subfigure}
\renewcaptionname{ngerman}{\figurename}{Abb.}
\renewcaptionname{ngerman}{\tablename}{Tab.}
\sisetup{detect-weight=true, detect-family=true,locale=DE,range-phrase={\,bis\,},list-final-separator ={\,\linebreak[0] \text{und}\,},separate-uncertainty=true,per-mode = symbol-or-fraction}
\DeclareSIUnit\curie{Ci}
\usepackage[backend=biber, style=chem-angew]{biblatex} 
\addbibresource{lit.bib} 

\usepackage{chemgreek}
\usepackage{chemformula}
\geometry{left = 2.5cm} \geometry{top = 3cm}

\urlstyle{same}
%Hyperlinks-Setup
\hypersetup{
	colorlinks,
	linktocpage,
	citecolor=black,
	filecolor=black,
	linkcolor=black,
	urlcolor=black
}

%\numberwithin{equation}{section}

\setlength{\parindent}{0 mm}
\setlength{\parskip}{2 mm} 



\pagestyle{scrheadings}
%Header oben links auf linker Seite (ungerade Seitenzahl) und oben rechts auf rechter Seite (gerade Seitenzahl), beinhaltet gruppennummer und Versuchskürzel. Im Fall eine einseitigen Dokuments: Header oben rechts
\ihead{\VERSUCHSABK} %Header oben rechts auf linker Seite und oben links auf rechter Seite. Beinhaltet die Namen der Verfasser. Im Fall eine einseitigen Dokuments: Header oben links!
\ohead{\GRUPPENNR}
\ofoot{\thepage} 
\cfoot{\empty}  
\ifoot{\empty} 


\newcommand{\VERSUCHSDATUM}{07.01.2025}
\newcommand{\PROTOKOLLDATUM}{\today}

\newcommand{\VerfasserEINS}{Vincent Kümmerle}
\newcommand{\MatNoEINS}{3712667}
\newcommand{\EmailEINS}{st187541@stud.uni-stuttgart.de}
\newcommand{\StudiengangEINS}{B.Sc. Chemie}

\newcommand{\VerfasserZWEI}{Elvis Gnaglo}
\newcommand{\MatNoZWEI}{3710504}
\newcommand{\EmailZWEI}{st189318@stud.uni-stuttgart.de}
\newcommand{\StudiengangZWEI}{B.Sc. Chemie}

\newcommand{\VerfasserDREI}{Julian Brügger}
\newcommand{\MatNoDREI}{3715444}
\newcommand{\EmailDREI}{st190050@stud.uni-stuttgart.de}
\newcommand{\StudiengangDREI}{B.Sc. Chemie}

\newcommand{\BETREUER}{Valentin Bayer}
\newcommand{\GRUPPENNR}{A05}

\newcommand{\VERSUCHSABK}{NMR}
\newcommand{\VERSUCHSNAME}{Kernmagnetische Resonanzspektroskopie}


\begin{document}
\thispagestyle{empty}


\begin{titlepage}

\begin{center}
\Huge{\textbf{\VERSUCHSABK\ - \VERSUCHSNAME}}\\
\vspace{10mm}% Abstand
\Large{Protokoll zum Versuch des PC 2 Praktikums von \\ \textbf{\VerfasserEINS\;\& \VerfasserZWEI\;\& \VerfasserDREI}}\\
\vspace{10mm} 
\Large{Universität Stuttgart}\\
\end{center}
\vspace{1cm}
\begin{center}
\begin{tabular}{ll}
\large{Autoren:}		& \large{\VerfasserEINS,} \large{\MatNoEINS} \\
 						& \large{\EmailEINS} \\
 						\vspace{0cm}\\
						& \large{\VerfasserZWEI,} \large{\MatNoZWEI} \\
                        & \large{\EmailZWEI} \\
						\vspace{0cm}\\
						& \large{\VerfasserDREI,} \large{\MatNoDREI} \\
                        & \large{\EmailDREI} \\
						\vspace{0cm}\\
\large{Gruppennummer:}	& \large{\GRUPPENNR} \\
\vspace{0cm}\\
\large{Versuchsdatum:}	& \large{\VERSUCHSDATUM} \\
\vspace{0cm}\\
\large{Betreuer:}		& \large{\BETREUER} \\
\vspace{0cm}\\
\large{Erstabgabedatum:} & \large{\PROTOKOLLDATUM}
\end{tabular}
\end{center}
\vspace{1cm}
\textbf{Abstract:}
In diesem Versuch wurde anhand verschiedener Proben die NMR Messtechnik untersucht. Dazu wurden im ersten Versuchsteil die Spin-Gitter-Relaxationszeit $T_1 = \SI{64,59}{ms}$ und die Spin-Spin-Relaxationszeit $T_2 = \SI{78,60}{ms}$ einer Kupfersulfat-Lösung über Messung des FIDs bestimmt.
Anschließend wurde die NMR-Bildgebung durch die Messung von zwei Wasserproben, sowie einer 2D Messung eines Buschstabens untersucht.
Zum Schluss wurde die chemische Verschiebung über die Messung verschiedener fluorhaltiger Proben gegen \ch{CFCl3} als Standard gemessen.
\end{titlepage}


\thispagestyle{empty}

\tableofcontents 

\clearpage

\renewcommand{\thepage}{\arabic{page}}
\setcounter{page}{1}


\section{Theorie}

Die Kernmagnetische Resonanzspektroskopie (nuclear magnetic resonance spectroscopy kurz NMR) ist eine Spektroskopieart, bei der Atome mit einem von null verschiedenen Kernspin detektiert werden können. Im Rahmen dieses Versuches wird der Fokus auf die Untersuchung von Spinrelaxationszeiten und der NMR-Bildgebung gelegt.

\subsection{Grundlagen der Kernmagnetischen Resonanzspektroskopie}

In diesem Versuch werden NMR-Spektren der Kerne \ch{^{1}H} und \ch{^{19}F} aufgenommen. Beide besitzen eine Kernspinquantenzahl von $ I = \frac{1}{2} $. Der Kernspin $\vec{I}$ der beiden Kerne mit $I = \frac{1}{2}$ hat den Betrag $| \vec{I} | = \sqrt{I(I+1)} \hbar$.\supercite{Skript} Der Kernspin $\vec{I}$ ist proportional zum magnetischen Moment $\vec{\mu}$ :
\begin{equation}
	\vec{\mu} = \gamma \cdot \vec{I} 
\end{equation}
Für die potentielle Energie des Kernspins ergibt sich 
\begin{equation}
	E = -\vec{\mu} \cdot \vec{B_0} = - \mu_z \cdot B_0 = \hbar \cdot \gamma \cdot B_0 \cdot m_I
	\label{eqE}
\end{equation}
$m_I$ ist hierbei die magnetische Kernspinquantenzahl, die die folgenden Werte annehmen kann $m_I = -I, -I+1, ...., I$, $\gamma$ beschreibt das gyromagnetische Verhältnis.
Als Auswahlregel für den Energieniveau-Übergang gilt $\Delta m_I = \pm 1$.

Bei den verwendeten Magnetfeldern im Bereich von $B_0 \approx 1 - 22$ T liegen die Einstrahlfrequenzen im Radiowellenbereich.

\subsection{Magnetische Wechselwirkungen}

Wechselwirkungen und elektronische Effekte in chemischen Verbindungeb können zu Aufspaltungen oder Verschiebungen führen. Zu nennen sind hier vor allem die chemische Verschiebung $\delta$, welche von der elektronischen (=chemischen) Umgebung des betreffenden Kernes abhängt und zu einer Verschiebung dessen Frequenz $\nu$ führt. Sie berechnet sich nach \autoref{eq: verschiebung}
\begin{equation}
\delta = \frac{\nu_{\text{Probe}} - \nu_{\text{Ref}}}{\nu_{\text{Ref}}}
\label{eq: verschiebung}
\end{equation}
Da $\delta$ sehr kleine Werte ergibt, werden diese nach Konvention mit $10^6$ multipliziert und in ppm angegeben.
Anhand dieser chemischen Verschiebung ist es möglich bei \ch{^{13}C} und \ch{^1H} Spektren Rückschlüsse auf die chemische Struktur der betreffenden Substanz zu ziehen. Weiterhin können Kernspins mit anderen Kernspins kopppeln, was zu Aufspaltungen im Spektrum führt, dieser Effekt ist als skalare Wechselwirkung bekannt. Außerdem können die Kernspins noch mit ungepaarten Elektronen, bei Anwesenheit paramgagnetischer Substanzen, koppeln.


\subsection{Spinrelaxation}

Die makroskopische Kernmagnetisierung $M$ ergibt sich aus der Vektorsumme der magnetischen Momente aller Kerne innerhalb der Probe. Im thermischen Gleichgewicht resultiert daraus aufgrund der energetischen Aufspaltung (Zeeman-Effekt) und der daraus folgenden Besetzungsunterschiede gemäß der Boltzmann-Statistik die Gleichgewichtsmagnetisierung $M_0$ in Richtung des äußeren Feldes. Da wir nur Kerne mit einem Kernspin von $I = 1/2$ betrachten, berechnet sich der Betrag der Magnetisierung wie folgt:
\begin{equation}
	M_0 = N \frac{\gamma^2 \hbar^2 B_0}{4kT} 
\end{equation}

Beim FT-Verfahren wird das System einem kurzen Radiofrequenzimpuls (wenige \textmu s) ausgesetzt, der das thermische Gleichgewicht stört. Dabei wird der longitudinale Magnetisierungsvektor aus der $z$-Richtung in die transversale Ebene ($x,y$-Ebene) geklappt. Nach dem RF-Impuls wird die Präzession dieser Quermagnetisierung als Induktionsspannung detektiert. Dieses Signal, der Free Induction Decay (FID), klingt mit der effektiven Relaxationszeit $T_2^*$ ab, welche sowohl die molekulare Spin-Spin-Relaxation ($T_2$) als auch Inhomogenitäten des statischen Magnetfeldes berücksichtigt.
Nach Bloch stellt sich diese Gleichgewichtsmagnetisierung nach einem Geschwindigkeitsgesetz erster Ordnung ein, was für die Magnetisierung in $z$-Richtung folgende Gleichung ergibt:
\begin{equation}
	\frac{\partial M_z}{\partial t} = - \frac{(M_z-M_0)}{T_1}
    \label{eq:blo}
\end{equation}

Die Relaxationszeit $T_1$ ist hierbei die Spin-Gitter-Relaxationszeit, sie beschreibt den Energieaustausch zwischen Spinsystem und den Freiheitsgraden der benachbarten Atome und Moleküle, welche als Gitter aufgefasst werden.

Mittels Fouriertransformation wird das Signal der Präzessionsbewegung als klassische Lorentzkurve dargestellt.

\subsection{Relaxationszeiten}

Um die Spin-Gitter-Relaxationszeit $ T_1 $ zu messen wird eine Inversion-Recovery- Sequenz, bestehend aus einer $ 180^{\circ} -  \tau - 90^{\circ} -$ Impulsfolge, eingesetzt. Durch Einsetzen von $M_Z (\tau = 0) = -M_0 $ in \autoref{eq:blo} lässt sich \autoref{eq: T1} herleiten.
\begin{equation}
    M_z(\tau) = M_0 \left( 1 - 2 e^{-\frac{\tau}{T_1}} \right)
    \label{eq: T1}
\end{equation}

Die Bestimmung der Spin-Spin-Relaxationszeit $T_2$ erfolgt mittels der Carr-Purcell-Sequenz. Dabei wird nach einem anfänglichen $90^\circ$-Puls eine Serie von $180^\circ$-Refokus\- sierungsimpulsen eingestrahlt, um die durch Feldinhomogenitäten verursachte Dephasierung umzukehren und eine Folge von Spinechos zu erzeugen. Die Abnahme der Amplitude der Echos hängt dabei nur von der Spin-Spin-Relaxationszeit $T_2$ ab. Die Amplitude des Spinechosignals zum Zeitpunkt $t = 2n\tau$ ergibt sich nach \autoref{eq: T2}:
\begin{equation}
    M(t) = M_0 \cdot e^{-\frac{t}{T_2}}
    \label{eq: T2}
\end{equation}


\subsection{Bildgebung}

Bei der NMR-Bildgebung werden zwei- oder dreidimensionale Bilder von Objekten durch eine ortsaufgelöste Erfassung der Probe erstellt. Wirkt ein homogenes Magnetfeld $B_0$, welches von einem linearen Gradienten $G_x = \partial B / \partial x$ in $x$-Richtung überlagert wird, so ist ein linearer Zusammenhang zwischen der Frequenz und der $x$-Position beobachtbar. Bei modernen Methoden werden zudem Gradienten in $y$-Richtung verwendet, sowie in Abhängigkeit der Zeit gemessen. Zur Aufnahme eines zweidimensionalen Schnittbildes wird ein Gradient in z-Richtung während des Anregungsimpulses angelegt (Schichtselektion), sodass nur Kerne in einer bestimmten Ebene die Resonanzbedingung erfüllen. Die Ortskodierung innerhalb dieser Schicht erfolgt anschließend durch Gradienten in x- und y-Richtung (Frequenz- und Phasenkodierung).“ Mit ihr ist es möglich die Resonanzbedingung in abhängigkeit des Gradienten zu bestimmen. Ein frequenzselektiver RF-Impuls der Form einer sin(x)/x-Funktion wird verwendet um eine gute Auflösung zu erhalten.


\section{Versuchsdurchführung}
Vor Beginn der Messungen wurde der ''Daily-Chek'' durchgeführt. Dafür wurde die Probe Daily-Check Sample in das Spektrometer gestellt und die Messung gestartet.
Als nächstes wurde mit derselben Probe ein FID-Signal aufgenommen, indem die Parameter auf NS = 10, RD = 1 s, Receiver Gain = 66 dB, detection mode = magnitude eingestellt wurden.

\subsection{Kernspinrelaxation von Lösungen paramagnetischer Ionen}
Die Probe der \ch{CuSO4}-Lösung (0,01 M) wurde ins Spektrometer eingesetzt und das FID-Signal in gleicher Weise wie bei der Daily-Check-Probe aufgenommen.
\subsubsection{Inversion-Recovery-Experiment}
Für die Messung der Spin-Gitter-Relaxationszeit $T_1$ der \ch{CuSO4}-Probe wurde die Messapplikation ''t1\_pcII'' verwendet. Vom Startwert  $\tau = $ 0,5 ms wurden insgesamt 18 Messungen mit $\tau$-Werten von 0,5 ms bis 180,5 ms durch Erhöhung in 10 ms Schritten durchgeführt und die Signalintensität aus den aufgenommenen Spektren abgelesen.
\subsubsection{Carr-Purcell-Experiment}
Für die Messung der Spin-Spin-Relaxationszeit $T_2$ der \ch{CuSO4}-Probe wurde die Messapplikation ''cpmg\_pcII'' verwendet. Für Carr-Purcell-Experiment wurden 20 Echos mit $\tau = 20$ ms verwendet, damit die Amplitude des letzten Echos ungefähr ein Zehntel der Amplitude des ersten Echos entspricht. Das Spektrum wurde gespeichert und die Echoamplitude jedes Echos nach dem Versuch abgelesen.

\subsection{NMR-Bildgebung}


\subsubsection{Wasserproben}
Bei diesem Versuchsteil wurden zwei Wasserproben mit der Messapplikation ''imaging1\_pcII'' im Spektrometer gemessen. Dafür wurden die Proben zuerst mit einer Ausrichtung von $0^\circ $ und anschließend mit einer Ausrichtung von $90^\circ $ gemessen.

\subsubsection{2D-NMR}
Im zweiten Teil dises Versuchs wurde eine andere Probe mit der Messapplikation ''imaging2\_pcII'' gemessen. Dazu wurde die Probe mit einer Ausrichtung von $0^\circ$ im Spektrometer gemessen. Anschließend wurde die Probe um $10^\circ$ gedreht und erneut gemessen. Dies wurde so lange wiederholt, bis eine Drehung um $180^\circ$ erreicht wurde.

\subsection{Chemische Verschiebung}

Im dritten Versuchsteil wurden drei Proben (\ch{CFCl3}, eine \ch{CFCl3}/\ch{C6H11F}-Mischung sowie PFTBA) mit der Messapplikation \texttt{fluor\_ft\_pcII} vermessen. Das für jede Probe aufgenommene FID-Signal wurde anschließend fourier-transformiert, um die entsprechenden \ch{^{19}F}-NMR-Spektren zu erhalten und die chemische Verschiebung der Signale zu bestimmen.

\section{Messwerte}

\subsection{Inversion-Recovery-Experiment}
Die Signalintensitäten in \autoref{tab: T1} entsprechen den Startintensitäten der aufgenommenen Spektren des Inversion-Recovery-Experiments für die \ch{CuSO4}-Probe bei den jeweiligen $\tau$-Werten.
\begin{table}[H]
    \centering
    \caption{Signalintensitäten des Inversion-Recovery-Experiments der \ch{CuSO4}-Probe.\\}
    \label{tab: T1}
    \begin{tabular}{c|c}
        $\tau$ / ms & Intensität / \% \\ 
		\hline
        0.5 & -79.83638584 \\ 
        10.5 & -61.32112332 \\ 
        20.5 & -43.38705739 \\ 
        30.5 & -27.67277167 \\ 
        40.5 & -13.46031746 \\
        50.5 & -1.245421245 \\
        60.5 & 10.12454212 \\ 
        70.5 & 19.71184371 \\ 
        80.5 & 28.42735043 \\ 
        90.5 & 35.79975580 \\ 
        100.5 & 42.53235653 \\ 
        110.5 & 48.29792430 \\ 
        120.5 & 53.46275946 \\ 
        130.5 & 57.87545788 \\ 
        140.5 & 61.89010989 \\ 
        150.5 & 65.34798535 \\ 
        160.5 & 68.29548230 \\ 
        170.5 & 71.12332112 \\ 
        180.5 & 73.78021978
    \end{tabular}
\end{table}


\newpage

\section{Auswertung}

\subsection{Kernspinrelaxation von Lösungen paramagnetischer Ionen}

\subsubsection{FID-Messungen}
%Hier FEHLT noch Text \& Erklärungen zu den FID-Signalen.
Die in diesem Versuch verwendete Messtechnik wird FT-Technik genannt. Dabei werden Radiowellen mit hoher Intensität eingestrahlt, bis die Magnetisierung die x,y-Ebene erreicht. Die Einstrahlung wird daraufhin gestoppt, wodurch die Magnetisierung sich in der x,y-Ebene frei dreht, bis die Ausgangsmagnetisierung wieder erreicht wird. Dadurch entsteht ein abklingendes FID-Signal, welches über eine Fouriertransformation anhand der Frequenz des abklingenden Signals zu einem typischen Absorptionssignal umgewandelt wird. Der Verlauf einer Messung ist in \autoref{fig: FID} dargestellt. 
\begin{figure}[H]
    \centering
    \includegraphics[width=0.8\textwidth]{A05/FID.jpeg}
    \caption{Darstellung der FT-Messtechnik zur Bestimmung des FID-Signals (unterer Pfad).\supercite{Skript}} 
    \label{fig: FID}
\end{figure}

% Die braucht ihr ja auch um eure Vorgehensweise zu erklären. Ihr könnt ja auch z.B. den Verlauf der beiden Signale miteinander vergleichen.
\autoref{fig: FID_Check} zeigt das aufgenommene FID-Signal der ''Daily-Check-Probe'' und \autoref{fig: FID_Cu} das der Kupfersulfat-Lösung.
\begin{figure}[H]
    \centering
    \includegraphics[width=0.8\textwidth]{A05/Check/fid_pcii.Daylicheck.pdf}
    \caption{FID-Signal der ''Daily-Check-Probe''.} 
    \label{fig: FID_Check}
\end{figure}

\begin{figure}[H]
    \centering
    \includegraphics[width=0.8\textwidth]{A05/Paramagnetisch/FID/fid_pcii_FID.pdf}
    \caption{FID-Signal der \ch{CuSO4}-Probe.} 
    \label{fig: FID_Cu}
\end{figure}
Der Vergleich der beiden Messungen zeigt, dass der FID der beiden Proben sehr änlich verläuft. Der einzige sichtbare Unterschied ist die gemessene Intensität beim Start der Messung, da die der ''Daily-Check-Probe'' bei ungefähr 85 \% und die der Kupfersulfat-Lösung bei ungefähr 105 \% liegt. 

\subsubsection{Relaxationszeiten}
Die gemessenen Signalintensitäten aus \autoref{tab: T1} lassen sich gegen die zugehörigen $\tau$-Werte auftragen, wie in \autoref{fig: T1_Mess} dargestellt ist.

\begin{figure}[H]
    \centering
    \includegraphics[width=0.9\textwidth]{A05/Paramagnetisch/T1 Inversion/NMR_data_T1.pdf}
    \caption{Signalintensitäten für verschiedene $\tau$-Werte der \ch{CuSO4}-Probe.} 
    \label{fig: T1_Mess}
\end{figure}
Das aufgezeichnete Signal in \autoref{fig: T1_Mess} ist aufgrund des 180°-Pulses negativ für kleine $\tau$-Werte, da die Magnetisierung $M_z(\tau)$ entlang der z-Achse invertiert wird und für kleine $\tau$-Werte der Exponentialterm in \autoref{eq: T1} größer als 1 ist. Bei zunehmendem $\tau$-Wert wird der Exponentialterm immer kleiner, wird bei $\tau = T_1 \cdot \ln(2)$ zu $\frac{1}{2}$ und somit die Magnetisierung $M_z(\tau)$ null. Für größere $\tau$-Werte wird die Magnetisierung $M_z(\tau)$ positiv, da der Exponentialterm in \autoref{eq: T1} kleiner als $\frac{1}{2}$ ist und die Magnetisierung sich dem Gleichgewichtswert $M_0$ annähert. Der 90°-Puls überführt die Magnetisierung $M_z(\tau)$ in die Quermagnetisierung in der $xy$-Ebene, welche detektiert werden kann. Da die Signalintensitäten aus \autoref{fig: T1_Mess} folglich proportional zu der Magnetisierung $M_z(\tau)$ sind, kann die Relaxationszeit $T_1$ durch \autoref{eq: T1} bestimmt werden.
Dazu wird die Gleichung logarithmiert, wodurch sich \autoref{eq: ln_T1} ergibt.
\begin{equation}
    \ln \left(\frac{M_0 - M_z(\tau)}{2 M_0} \right) = - \frac{\tau}{T_1}
    \label{eq: ln_T1}   
\end{equation}
Als $M_0$ wird der Betrag des ersten gemessenen Intensitätswertes bei $\tau = 0,5$ ms verwendet, da dieser der negativen Gleichgewichtsmagnetisierung entspricht.
Durch Auftragung von $\ln \left(\frac{M_0 - M_z(\tau)}{2 M_0} \right)$ gegen $\tau$ lässt sich \autoref{fig: ln_T1} erstellen. Zusätzlich ist in \autoref{fig: ln_T1} eine Fitgerade im Bereich bis $\tau = 140,5$ ms eingetragen, da die Messwerte ab diesem Wert nicht mehr linear verlaufen.

\begin{figure}[H]
    \centering
    \includegraphics[width=0.9\textwidth]{A05/Paramagnetisch/T1 Inversion/NMR_data_ln_Fit.pdf}
    \caption{Auftragung von $\ln \left(\frac{M_0 - M_z(\tau)}{2 M_0} \right)$ gegen die Zeit $\tau$.} 
    \label{fig: ln_T1}
\end{figure}
Aus der Geradengleichung $y = m \cdot \tau + c$ des Fits mit $m = -0,015483$ und $c = 0,0728$ lässt sich die Spin-Gitter-Relaxationszeit $T_1$ nach \autoref{eq: ln_T1} berechnen, da die Steigung $m$ dem Wert $-\frac{1}{T_1}$ entspricht.
\begin{equation}
    T_1 = - \frac{1}{m} = - \frac{1}{-0,015483} = \SI{64,59}{ms}
\end{equation}
% T1 = 64.59 ms

Die im Carr-Purcell-Experiment gemessenen Signalintensitäten der 20 Spinechos lassen sich gegen die Zeit $\tau$ auftragen, wie in \autoref{fig: T2_Mess} dargestellt ist.
\begin{figure}[H]
    \centering
    \includegraphics[width=0.9\textwidth]{A05/Paramagnetisch/Carr-Purcell/cpmg_pci_20Loops_20msi_T2.pdf}
    \caption{Auftragung der Signalintensitäten der gemessenen 20 Spinechos gegen die Zeit $\tau$.} 
    \label{fig: T2_Mess}
\end{figure}
Die Höhen der Spinechos aus \autoref{fig: T2_Mess} werden mit den zugehörigen Werten für $2n\tau$ in \autoref{tab: T2} dargestellt und entsprechen den Werten der Magnetisierung $M(2n\tau)$, die mit der Zeit $\tau$ exponentiell abnimmt.
Diese Abnahme der Magnetisierung wird durch \autoref{eq: T2} beschrieben.
\begin{table}[H]
    \centering
    \caption{Signalintensitäten des Carr-Purcell-Experiments der \ch{CuSO4}-Probe.\\}
    \label{tab: T2}
    \begin{tabular}{c|c}
        $2n\tau$ / ms & $M(2n\tau)$ \\
        \hline
20.2 & 118.095 \\
40.0 & 118.617 \\
60.0 & 87.741 \\
80.0 & 66.868 \\
100.1 & 49.621 \\
120.0 & 37.500 \\
139.9 & 28.181 \\
160.1 & 21.224 \\
180.0 & 15.733 \\
200.2 & 11.963 \\
220.2 & 9.264 \\
240.0 & 7.085 \\
260.0 & 5.379 \\
280.0 & 4.170 \\
300.2 & 3.358 \\
319.8 & 2.592 \\
340.1 & 2.131 \\
360.4 & 1.990 \\
380.4 & 1.447 \\
400.0 & 1.349
    \end{tabular}
\end{table}
Um die Spin-Spin-Relaxationszeit $T_2$ zu bestimmen, wird zuerst \autoref{eq: T2} logarithmiert, wodurch sich \autoref{eq: ln_T2} ergibt.
\begin{equation}
    \ln M (t = 2n \tau) = - \frac{t}{T_2}
    \label{eq: ln_T2}   
\end{equation}
Dann werden die Werte aus \autoref{tab: T2} verwendet, um mit \autoref{eq: ln_T2} alle logarithmierten Werte der Magnetisierung zu berechnen.
Durch Auftragung von $\ln M (2n \tau)$ gegen $\tau$ lässt sich \autoref{fig: ln_T2} erstellen. Zusätzlich ist in \autoref{fig: ln_T2} eine Fitgerade eingetragen, um aus der Steigung die Relaxationszeit $T_2$ zu bestimmen.
\begin{figure}[H]
    \centering
    \includegraphics[width=0.9\textwidth]{A05/Paramagnetisch/Carr-Purcell/cpmg_pci_20Loops_20msi_T2_ln_Fit.pdf}
    \caption{Auftragung der logarithmierten Magnetisierung der gemessenen Spinechos gegen die Zeit $\tau$.} 
    \label{fig: ln_T2}
\end{figure}
Aus der Geradengleichung $y = m \cdot t + c$ des Fits mit $m = -0,012723$ und $c = 5,1207$ lässt sich die Spin-Spin-Relaxationszeit $T_2$ nach \autoref{eq: ln_T2} berechnen, da die Steigung $m$ dem Wert $-\frac{1}{T_2}$ entspricht.
\begin{equation}
    T_2 = - \frac{1}{m} = - \frac{1}{-0,012723} = \SI{78,60}{ms}
\end{equation}
%T2 = 78.60 ms
Somit ist die Spin-Spin-Relaxationszeit $T_2 = \SI{78,60}{ms}$ der \ch{CuSO4}-Probe größer als die Spin-Gitter-Relaxationszeit $T_1 = \SI{64,59}{ms}$.
Dies widerspricht der Erwartung, da die Energieabgabe an das Gitter, die durch $T_1$ charakterisiert wird, länger sein sollte als der Verlust der Phasenbeziehung der Spins in der Relaxationszeit $T_2$.

\subsection{NMR-Bildgebung}
\subsubsection{Wasserproben}
\autoref{wasser0} zeigt das aufgenommene Spektrum der Wasserproben bei einer Ausrichtung von $0^\circ$ also orthogonal zum Gradienten.
\begin{figure}[H]
    \centering
    \includegraphics[width=0.9\textwidth]{A05/Bildgebung/Probe 1/Wasser0.pdf}
    \caption{Gemessenes NMR-Spektrum der beiden Wasserproben bei einer orthogonalen Ausrichtung zum Gradienten des Spektrometers.} 
    \label{wasser0}
\end{figure}
Im Spektrum erkennt man einen großen, sehr breiten Peak. Dieser kommt durch die Positionierung der beiden Proben zustande, da diese bei einer Ausrichtung von $0^\circ$ den gleichen Abstand zur Gradientenquelle besitzen und somit das gleiche Magnetfeld an allen Stellen in der Probe erfahren. Die Absenkung an der Seite kommt durch eine Messungenauigkeit zustande, bei der eine Probe einen leicht veränderten Abstand zur Gradientenquelle besitzt.  
\autoref{wasser90} zeigt das aufgenommene Spektrum bei einer Ausrichtung von $90^\circ$.
\begin{figure}[H]
    \centering
    \includegraphics[width=0.9\textwidth]{A05/Bildgebung/Probe 1/Wasser90.pdf}
    \caption{Gemessenes NMR-Spektrum der beiden Wasserproben bei einer parallelen Ausrichtung zum Gradienten des Spektrometers.} 
    \label{wasser90}
\end{figure}
Im Spektrum sind nun zwei Signale zu sehen. Das kommt daher, dass die Proben nach der Drehung hintereinander in der Richtung des Gradienten liegen und deshalb eine unterschiedliche Magnetfeldstärke erfahren. 
\subsubsection{2D-NMR}
\autoref{h_seit} zeigt das aufgenommene 2D-NMR der Probe von der Seite.
\begin{figure}[H]
    \centering
    \includegraphics[width=0.8\textwidth]{A05/Bildgebung/Probe 2/Probe2_seite2.png}
    \caption{Gemessenes 2D NMR-Spektrum der Probe von der Seite.} 
    \label{h_seit}
\end{figure}
Im Spektrum sind die einzelnen Schichten in denen Wasserstoffatome detektiert wurden, farblich ihrer Intensität und damit ihrer Spindichte nach markiert. Es wird ersichtlich, dass die Probe in der Mitte die höchste Spindichte hat, was darauf deuten lässt, dass sich dort das Wasser im Probenröhrchen befindet. \autoref{h_ob} zeigt das gleiche Spektrum von oben. 
\begin{figure}[H]
    \centering
    \includegraphics[width=0.8\textwidth]{A05/Bildgebung/Probe 2/Probe2_oben.png}
    \caption{Gemessenes 2D NMR-Spektrum der Probe von oben.} 
    \label{h_ob}
\end{figure}
Durch die Ansicht von oben, wird erkenntlich, dass es sich bei der Probe um den Buchstaben H handelt der mit Wasser gefüllt ist. Man erkennt außerdem das Probenröhrchen, in dem sich die Probe befindet, an der türkisen umrandung, sowie die Probenkammer des NMR-Spektrometers an der dunkelblauen Fläche. Die Enden der beiden parallelen Striche des H's sind aufgrund von mangelnder Genauigkeit bei der Ausrichtung der Probe nicht so intensiv wie der Rest des Buchstabens.
\subsection{Chemische Verschiebung}

%\subsection{Carr-Purcell-Experiment}
 
In \autoref{fig: cv1} ist das fouriertransformierte \ch{^{19}F}-Spektrum der \ch{CFCl3}-Probe abgebildet.

\begin{figure}[H]
    \centering
    \includegraphics[width=1\textwidth]{A05/Chemische Verschiebung/CFCl3.pdf}
    \caption{Das \ch{^{19}F}-Spektrum von Trichlorfluormethan.} 
    \label{fig: cv1}
\end{figure}

Das Maximum liegt bei einer Frequenz zwischen 1,2 kHz und 1,4 kHz und ist dem Fluoratom im Trichlorfluormethan zuzuordnen.

\autoref{fig: cv2} zeigt das fouriertransformierte \ch{^{19}F}-Spektrum von Fluorcyclohexan mit Trichlorfluormethan als internem Standard. 

\begin{figure}[H]
    \centering
    \includegraphics[width=1\textwidth]{A05/Chemische Verschiebung/CFCl3_C6H11F.pdf}
    \caption{Das \ch{^{19}F}-Spektrum von Fluorcyclohexan mit Trichlorfluormethan als internem Standard.} 
    \label{fig: cv2}
\end{figure}

Der rechte Peak, dessen Maximum bei $1,36$ kHz liegt ist aufgrund des ähnlichen Frequenzwertes dem internen Standard Trichlorfluormethan zuzuordnen. Der linke Peak, der die höchste Intensität besitzt und dessen Frequenz bei $-0,21$ kHz liegt ist somit dem Fluor-Atom im Fluorcyclohexan zuzuordnen. Die Positionen der Peaks lassen sich dadurch erklären, dass die elektronegativen Chlorsubstituenten, das Fluoratom im \ch{CFCl3} entschirmen, was zu einer höheren Frequenz führt, wobei der organische Cyclohexylrest, ledeglich einen schwachen +I-Effekt ausübt, was zu einer schwachen Abschirmung führt.

Um nun nach \autoref{eq: verschiebung} die chemische Verschiebung von Fluorcyclohexan zu berechnen muss zunächst die Messfrequenz von $18,76$ MHz zu den Frequenzwerten addiert werden, ehe diese in die Formel eingesetzt werden können.

\begin{equation}
    \delta_{\text{C}_6\text{H}_{11}\text{F}} = \frac{\nu_{\text{C}_6\text{H}_{11}\text{F}} - \nu_{\text{CFCl}_3}}{\nu_{\text{CFCl}_3}} = \frac{18,7598 \text{ MHz} - 18,76136 \text{ MHz}}{18,76136 \text{ MHz}} \cdot 10^6 = -83,8097 \text{ ppm}
\end{equation}

Um die Linienbreiten zu bestimmen wird die Halbwertsbreite FWHH (Full Width Half Height) abgelesen. Bei Fluorcyclohexan ist eine Breite von ca $0,1$ kHz bei einer halben Höhe von ca $700$ abzulesen, was einer Linienbreite von $53,3$ ppm entspricht. Bei Trichlorfluormethan eine Linienbreite von ca $0,3$ kHz bei einer ungefähren halben Höhe von 380, was einer Linienbreite von ca $16,0$ ppm entspricht. Da die Peakbreite umgekehrt proportional zur Relaxationszeit ist, legt dies den Schluss nahe, dass die Relaxationszeit von Fluorcyclohexan kleiner als die des Trichlorfluormethan ist.

In \autoref{fig: cv3} ist das fouriertransformierte \ch{^{19}F}-Spektrum der PFTBA-Probe dargestellt.


\begin{figure}[H]
    \centering
    \includegraphics[width=1\textwidth]{A05/Chemische Verschiebung/PFTBA.pdf}
    \caption{Das \ch{^{19}F}-Spektrum von PFTBA.} 
    \label{fig: cv3}
\end{figure}

In diesem Spektrum lassen sich drei Peaks bei den Frequenzen $-1,221$~kHz, $-1,101$~kHz und $-0,240$~kHz identifizieren. Unter Berücksichtigung der Betriebsfrequenz des Spektrometers ergeben sich daraus die absoluten Resonanzfrequenzen von $18,7588$~MHz, $18,7589$~MHz und $18,7598$~MHz. Durch die Berechnung nach \autoref{eq: verschiebung} resultieren daraus die Werte $-135,3912$~ppm, $-122,0823$ ppm und $-26,6081$~ppm.

Die Zuordnung dieser drei Peaks lässt sich durch die elektronische Struktur des Perfluorotributylamin-Moleküls (FTBA) erklären, wie sie auch in \supercite{paper} hinlänglich erklärt wurde. Das Signal bei $-26,6081$~ppm weist den höchsten ppm-Wert auf und gehört zu den Fluoratomen am Kohlenstoff in $\alpha$-Position zum Stickstoffatom. Da Stickstoff als elektronegatives Zentrum wirkt, entzieht er der unmittelbaren Umgebung Elektronendichte, was zu einer starken Entschirmung dieser Kerne führt. Der Peak bei $-122,0823$~ppm wird den terminalen \ch{CF3}-Gruppen zugeordnet, die am Ende der fluorierten Butylketten eine charakteristische Abschirmung erfahren.

Dass im Spektrum insgesamt nur drei Signale, trotz vier chemisch verschiedener Fluoratome für die Butylkette zu beobachten sind, liegt an der elektronischen / chemischen Änlichkeit der beiden  mittleren Fluoratome. Wie im beigefügten Paper erläutert, weisen die Fluoratome in den mittleren $\beta$- und $\gamma$-Positionen nahezu identische chemische Umgebungen auf. Infolgedessen überlagern sich ihre Resonanzen zu einem gemeinsamen, intensiven Peak bei $-135,3912$~ppm, was die geringere Anzahl an beobachteten Signalen im Vergleich zur Anzahl der chemisch unterschiedlichen Kohlenstoffpositionen erklärt. Die grafisch ermittelten Linienbreiten der Signale in diesem Datensatz betragen circa $0,15$~kHz, $0,21$~kHz und $0,50$~kHz, was auf der ppm-Skala Werten von $7,99$~ppm, $11,19$~ppm und $26,65$~ppm entspricht.


\section{Fehlerbetrachtung} 
Ein möglicher Fehler bei der Bestimmung der Spin-Spin-Relaxationszeit $T_2$ könnte durch die Wahl der Anzahl der Echos im Carr-Purcell-Experiment entstanden sein, da die Amplitude des letzten Echos nur ungefähr ein Zehntel der Amplitude des ersten Echos betragen sollte, was den Messwerten zufolge nicht zutrifft. Die geringen Echomplituden der letzten fünf Echos könnten zu einer ungenauen Bestimmung der Relaxationszeit $T_2$ geführt haben, da die Werte für die Magnetisierung in diesem Bereich sehr klein sind und somit stärker vom expontiellen Abklingverhalten abweichen können.
\newpage
\section{Zusammenfassung}
Im ersten Versuchsteil wurde die Spin-Gitter-Relaxationszeit einer Kupfersulfat- Lösung über das Inversion-Recovery-Experiment als $T_1 = \SI{64,59}{ms}$ bestimmt. Anschließend wurde die Spin-Spin-Relaxationszeit derselben Lösung über das Carr-Purcell- Experiment als $T_2 = \SI{78,60}{ms}$ bestimmt.\\
Im zweiten Versuchsteil wurde die Bildgebung des NMR-Spektrometers untersucht. Dazu wurde zuerst eine Probe mit Wasser bei einer Ausrichtung von $0^\circ$ gemessen, wobei ein Spektrum mit einem Peak entstand. Anschließend wurde die Probe um $90^\circ$ gedreht und erneut gemessen, wobei ein Spektrum mit zwei Peaks entstand. Zum Schluss wurde ein 2D-NMR eines mit Wasser gefüllten Buchstaben aufgenommen.\\
Im letzten Versuchsteil wurde zunächst ein \ch{^19F}-Spektrum von Fluorcyclohexan mit Trichlorfluormethan als internem Standard aufgenommen. Der Trichlorfluormethan-Peak ist aufgrund der entschirmenden Chlorsubstituenten gegenüber dem Fluorcyclohexan hochfeldverschoben. Es konnte eine chemische verschiebung von $ \delta_{\ch{C6H11F} = -83,8097}$ ppm für Fluorcyclohexan bestimmt werden. Die Linienbreite konnte zu $53,3$ ppm bestimmt werden. Im Perfluorotributylamin-Spektrum waren drei Peaks zu beobachten. Die chemischen Verschiebungen dieser liegen bei $-135,3912$~ppm, $-122,0823$ ppm und $-26,6081$~ppm, in der Reihenfolge sind sie den $\beta$- und $\gamma$-Fluoaratomen, den endständigen \ch{CF3}-Gruppen und schließlich den $\alpha$-Fluoratomen zugeordnet. Aufgrund der sehr ähnlichen chemischen Umgebung der $\beta$- und $\gamma$-Fluorsubstituenten überlagern sich deren Peaks.

\printbibliography[title={Literatur}]


\end{document}
