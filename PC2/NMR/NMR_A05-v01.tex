\documentclass[a4paper,12pt,bibliography=totocnumbered]{scrartcl}

\usepackage[utf8]{inputenc} 
\usepackage[T1]{fontenc}
\usepackage[ngerman]{babel} 
\usepackage{amsmath, amssymb,amsfonts}
\usepackage{graphicx}
\usepackage{csquotes}
\usepackage[bookmarks,colorlinks=true]{hyperref}
\usepackage{geometry}
\usepackage{float}
\usepackage[final]{pdfpages}
\usepackage{framed, color} 
\usepackage{scrlayer-scrpage}
\usepackage{siunitx}
\usepackage{subfigure}
\renewcaptionname{ngerman}{\figurename}{Abb.}
\sisetup{detect-weight=true, detect-family=true,locale=DE,range-phrase={\,bis\,},list-final-separator ={\,\linebreak[0] \text{und}\,},separate-uncertainty=true,per-mode = symbol-or-fraction}
\DeclareSIUnit\curie{Ci}
\usepackage[backend=biber, style=chem-angew]{biblatex} 
\addbibresource{lit.bib} 

\usepackage{chemgreek}
\usepackage{chemformula}
\geometry{left = 2.5cm} \geometry{top = 3cm}

\urlstyle{same}
%Hyperlinks-Setup
\hypersetup{
	colorlinks,
	linktocpage,
	citecolor=black,
	filecolor=black,
	linkcolor=black,
	urlcolor=black
}

%\numberwithin{equation}{section}

\setlength{\parindent}{0 mm}
\setlength{\parskip}{2 mm} 



\pagestyle{scrheadings}
%Header oben links auf linker Seite (ungerade Seitenzahl) und oben rechts auf rechter Seite (gerade Seitenzahl), beinhaltet gruppennummer und Versuchskürzel. Im Fall eine einseitigen Dokuments: Header oben rechts
\ihead{\VERSUCHSABK} %Header oben rechts auf linker Seite und oben links auf rechter Seite. Beinhaltet die Namen der Verfasser. Im Fall eine einseitigen Dokuments: Header oben links!
\ohead{\GRUPPENNR}
\ofoot{\thepage} 
\cfoot{\empty}  
\ifoot{\empty} 


\newcommand{\VERSUCHSDATUM}{07.01.2025}
\newcommand{\PROTOKOLLDATUM}{\today}

\newcommand{\VerfasserEINS}{Vincent Kümmerle}
\newcommand{\MatNoEINS}{3712667}
\newcommand{\EmailEINS}{st187541@stud.uni-stuttgart.de}
\newcommand{\StudiengangEINS}{B.Sc. Chemie}

\newcommand{\VerfasserZWEI}{Elvis Gnaglo}
\newcommand{\MatNoZWEI}{3710504}
\newcommand{\EmailZWEI}{st189318@stud.uni-stuttgart.de}
\newcommand{\StudiengangZWEI}{B.Sc. Chemie}

\newcommand{\VerfasserDREI}{Julian Brügger}
\newcommand{\MatNoDREI}{3715444}
\newcommand{\EmailDREI}{st190050@stud.uni-stuttgart.de}
\newcommand{\StudiengangDREI}{B.Sc. Chemie}

\newcommand{\BETREUER}{Valentin Bayer}
\newcommand{\GRUPPENNR}{A05}

\newcommand{\VERSUCHSABK}{NMR}
\newcommand{\VERSUCHSNAME}{Kernmagnetische Resonanzspektroskopie}


\begin{document}
\thispagestyle{empty}


\begin{titlepage}

\begin{center}
\Huge{\textbf{\VERSUCHSABK\ - \VERSUCHSNAME}}\\
\vspace{10mm}% Abstand
\Large{Protokoll zum Versuch des PC 2 Praktikums von \\ \textbf{\VerfasserEINS\;\& \VerfasserZWEI\;\& \VerfasserDREI}}\\
\vspace{10mm} 
\Large{Universität Stuttgart}\\
\end{center}
\vspace{1cm}
\begin{center}
\begin{tabular}{ll}
\large{Autoren:}		& \large{\VerfasserEINS,} \large{\MatNoEINS} \\
 						& \large{\EmailEINS} \\
 						\vspace{0cm}\\
						& \large{\VerfasserZWEI,} \large{\MatNoZWEI} \\
                        & \large{\EmailZWEI} \\
						\vspace{0cm}\\
						& \large{\VerfasserDREI,} \large{\MatNoDREI} \\
                        & \large{\EmailDREI} \\
						\vspace{0cm}\\
\large{Gruppennummer:}	& \large{\GRUPPENNR} \\
\vspace{0cm}\\
\large{Versuchsdatum:}	& \large{\VERSUCHSDATUM} \\
\vspace{0cm}\\
\large{Betreuer:}		& \large{\BETREUER} \\
\vspace{0cm}\\
\large{Erstabgabedatum:} & \large{\PROTOKOLLDATUM}
\end{tabular}
\end{center}
\vspace{1cm}
\textbf{Abstract:}
In diesem Versuch wurde 
\end{titlepage}


\thispagestyle{empty}

\tableofcontents 

\clearpage

\renewcommand{\thepage}{\arabic{page}}
\setcounter{page}{1}


\section{Theorie}

Die Kernmagnetische Resonanzspektroskopie (nuclear magnetic resonance spectroscopy kurz NMR) ist eine Spektroskopieart, bei der Atome mit einem von null verschiedenen Kernspin detektiert werden können. Im Rahmen dieses Versuches wird der Fokus auf die Untersuchung von Spinrelaxationszeiten und der NMR-Bildgebung gelegt.

\subsection{Grundlagen der Kernmagnetischen Resonanzspektroskopie}

In diesem Versuch werden NMR-Spektren der Kerne \ch{^{1}H} und \ch{^{19}F} aufgenommen. Beide besitzen eine Kernspinquantenzahl von $ I = \frac{1}{2} $. Der Kernspin $\vec{I}$ der beiden Kerne mit $I = \frac{1}{2}$ hat den Betrag $| \vec{I} | = \sqrt{I(I+1)} \hbar$. Der Kernspin $\vec{I}$ ist proportional zum magnetischen Moment $\vec{\mu}$ :
\begin{equation}
	\vec{\mu} = \gamma \cdot \vec{I} 
\end{equation}
Für die potentielle Energie des Kernspins ergibt sich 
\begin{equation}
	E = -\vec{\mu} \cdot \vec{B_0} = - \mu_z \cdot B_0 = \hbar \cdot \gamma \cdot B_0 \cdot m_I
	\label{eqE}
\end{equation}
$m_I$ ist hierbei die magnetische Kernspinquantenzahl, die die folgenden Werte annehmen kann $m_I = -I, -I+1, ...., I$, $\gamma$ beschreibt das gyromagnetische Verhältnis.
Als Auswahlregel für den Energieniveau-Übergang gilt $\Delta m_I = \pm 1$.

Bei den verwendeten Magnetfeldern im Bereich von $B_0 \approx 1 - 22$ T liegen die Einstrahlfrequenzen im Radiowellenbereich.

\subsection{Magnetische Wechselwirkungen}

Um die Energieeigenwerte der zu untersuchenden Kerne bestimmen zu können muss der Energieausdruck \autoref{eqE} durch den zugehörigen Hamiltonoperator ausgedrückt werden. Den entsprechenden Energieeigenwert erhält man dan durch anwenden dieses Hamiltonoperators auf die beiden Spinfunktionen $ |\alpha \rangle = |+\frac{1}{2} \rangle $ und  $|\beta \rangle = |-\frac{1}{2} \rangle $ erhält man die folgenden Energieeigenwerte:
\begin{equation}
	\begin{split}
		\epsilon_{\frac{1}{2}} = - \frac{1}{2} \gamma \hbar B_0 \\
		\epsilon_{-\frac{1}{2}} = + \frac{1}{2} \gamma \hbar B_0 \\
	\end{split}
\end{equation}

In NMR-Spektren einer chemischen Verbindung gibt es allerdings weit mehr als diese beiden Werte. Wechselwirkungen und elektronische Effekte können zu Aufspaltungen oder Verschiebungen führen. Zu nennen sind hier vor allem die chemische Verschiebung $\delta$, welche von der elektronischen (=chemischen) Umgebung des betreffenden Kernes abhängt und zu einer Verschiebung dessen Frequenz $\nu$ führt. Anhand dieser chemischen Verschiebung ist es möglich bei \ch{^{13}C} und \ch{^1H} Spektren Rückschlüsse auf die chemische Struktur der betreffenden Substanz zu ziehen. Weiterhin können Kernspins mit anderen Kernspins kopppeln, was zu Aufspaltungen im Spektrum führt, dieser Effekt ist als sklarare Wechselwirkung bekannt. Außerdem können die Kernspins noch mit ungepaarten Elektronen, bei Anwesenheit paramgagnetischer Substanzen, koppeln.


\subsection{Spinrelaxation}

Um von der Betrachtung eines einzelnen Kerns auf die Betrachtung makroskopischer Materie überzugehen wird die Kernmagnetisiserung $M_0$ verwendet, welche alle Spins in einer Probe berücksichtigt. Da wir nur Kerne mit einem Kernspin von $I=\frac{1}{2} $ betrachten berechnet sich die Kernmagnetisiserung wie folgt.

\begin{equation}
	M_0 = N \frac{\gamma^2 \hbar^2 B_0}{4kT} 
\end{equation}

Beim FT-Verfahren wird das System einem kurzen Radiofrequenzimpuls (wenige µs) ausgesetzt, der das thermische (magnetische) Gleichgewicht stört. Dabei wird der longitudinale Magnetisierungsvektor aus der $M_z$ in die transversale Ebene ($x,y$-Ebene) geklappt. Nach dem dem RF-Impuls lässt sich die Präzession dieser Quermagnetisierung als Induktionsspannung detektieren. Es wird ein mit der Relaxationszeit $T_2^*$ exponentiell abklingendes Signal detektiert, welches als freier Induktionszerfall, oder Free Induction Decay, kurz FID bezeichnet wird.

Nach Bloch stellt sich diese Gleichgewichtsmagnetisierung nach einem Geschwindigkeitsgesetz erster Ordnung ein, was für die Magnetisierung in $z$-Richtung folgende Gleichung ergibt:
\begin{equation}
	\frac{\partial M_z}{\partial t} = - \frac{(M_z-M_0)}{T_1}
\end{equation}

Die Relaxationszeit $T_1$ ist hierbei die Spin-Gitter-Relaxationszeit, sie beschreibt den Energieaustausch zwischen Spinsystem und den Freiheitsgraden der benachbarten Atome und Moleküle, welche als Gitter aufgefasst werden.

Mittels Fouriertransformation wird das Signal der Präzessionsbewegung als klassische Lorentzkurve dargestellt.

\subsection{Relaxationszeiten}



\begin{equation}
    M_z(\tau) = M_0 \left( 1 - 2 e^{-\frac{\tau}{T_1}} \right)
    \label{eq: T1}
\end{equation}

\subsection{Bildgebung}




\newpage

\section{Versuchsdurchführung}
Vor Beginn der Messungen wurde der ''Daily-Chek'' durchgeführt. Dafür wurde die Probe Daily-Check Sample in das Spektrometer gestellt und die Messung gestartet.
Als nächstes wurde mit derselben Probe ein FID-Signal aufgenommen, indem die Parameter auf NS = 10, RD = 1 s, Receiver Gain = 66 dB, detection mode = magnitude eingestellt wurden.

\subsection{Kernspinrelaxation von Lösungen paramagnetischer Ionen}
Die Probe der \ch{CuSO4}-Lösung (0,01 M) wurde ins Spektrometer eingesetzt und das FID-Signal in gleicher Weise wie bei der Daily-Check-Probe aufgenommen.
\subsubsection{Inversion-Recovery-Experiment}
Für die Messung der Spin-Gitter-Relaxationszeit $T_1$ der \ch{CuSO4}-Probe wurde die Messapplikation ''t1\_pcII'' verwendet. Vom Startwert  $\tau = $ 0,5 ms wurden insgesamt 18 Messungen mit $\tau$-Werten von 0,5 ms bis 180,5 ms durch Erhöhung in 10 ms Schritten durchgeführt und die Signalintensität aus den aufgenommenen Spektren abgelesen.
\subsubsection{Carr-Purcell-Experiment}
Für die Messung der Spin-Spin-Relaxationszeit $T_2$ der \ch{CuSO4}-Probe wurde die Messapplikation ''cpmg\_pcII'' verwendet. Für Carr-Purcell-Experiment wurden 20 Echos mit 20 ms verwendet, damit die Amplitude des letzten Echos ungefähr ein Zehntel der Amplitude des ersten Echos entspricht. Das Spektrum wurde gespeichert und die Echoamplitude jedes Echos nach dem Versuch abgelesen.

\subsection{NMR-Bildgebung}


\subsubsection{Wasserproben}
Bei diesem Versuchsteil wurden zwei Wasserproben mit der Messapplikation ''imaging1\_pcII'' im Spektrometer gemessen. Dafür wurden die Proben zuerst mit einer Ausrichtung von $0^\circ $ und anschließend mit einer Ausrichtung von $90^\circ $ gemessen.

\subsubsection{2D-NMR}
Im zweiten Teil dises Versuchs wurde eine andere Probe mit der Messapplikation ''imaging2\_pcII'' gemessen. Dazu wurde die Probe mit einer Ausrichtung von $0^\circ$ im Spektrometer gemessen. Anschließend wurde die Probe um $10^\circ$ gedreht und erneut gemessen. Dies wurde so lange wiederholt, bis eine Drehung um $180^\circ$ erreicht wurde.

\subsection{Chemische Verschiebung}

\section{Messwerte}

\subsection{Inversion-Recovery-Experiment}
Die Signalintensitäten in \autoref{tab: T1} entsprechen den Startintensitäten der aufgenommenen Spektren des Inversion-Recovery-Experiments für die \ch{CuSO4}-Probe bei den jeweiligen $\tau$-Werten.
\begin{table}[H]
    \centering
    \begin{tabular}{c|c}
        $\tau$ / ms & Intensität / \% \\ 
		\hline
        0.5 & -79.83638584 \\ 
        10.5 & -61.32112332 \\ 
        20.5 & -43.38705739 \\ 
        30.5 & -27.67277167 \\ 
        40.5 & -13.46031746 \\
        50.5 & -1.245421245 \\
        60.5 & 10.12454212 \\ 
        70.5 & 19.71184371 \\ 
        80.5 & 28.42735043 \\ 
        90.5 & 35.79975580 \\ 
        100.5 & 42.53235653 \\ 
        110.5 & 48.29792430 \\ 
        120.5 & 53.46275946 \\ 
        130.5 & 57.87545788 \\ 
        140.5 & 61.89010989 \\ 
        150.5 & 65.34798535 \\ 
        160.5 & 68.29548230 \\ 
        170.5 & 71.12332112 \\ 
        180.5 & 73.78021978
	\label{tab: T1}
    \end{tabular}
\end{table}




\section{Auswertung}

\subsection{Kernspinrelaxation von Lösungen paramagnetischer Ionen}

\subsubsection{FID-Messungen}

% Die braucht ihr ja auch um eure Vorgehensweise zu erklären. Ihr könnt ja auch z.B. den Verlauf der beiden Signale miteinander vergleichen.

\begin{figure}[H]
    \centering
    \includegraphics[width=0.8\textwidth]{A05/Check/fid_pcii.Daylicheck.pdf}
    \caption{FID-Signal der ''Daily-Check-Probe''.} 
    \label{fig: FID_Check}
\end{figure}


\begin{figure}[H]
    \centering
    \includegraphics[width=0.8\textwidth]{A05/Paramagnetisch/FID/fid_pcii_FID.pdf}
    \caption{FID-Signal der \ch{CuSO4}-Probe.} 
    \label{fig: FID_Cu}
\end{figure}

\subsubsection{Relaxationszeiten}
Die gemessenen Signalintensitäten aus \autoref{tab: T1} lassen sich gegen die zugehörigen $\tau$-Werte auftragen, wie in \autoref{fig: T1_Mess} dargestellt ist.

\begin{figure}[H]
    \centering
    \includegraphics[width=0.8\textwidth]{A05/Paramagnetisch/T1 Inversion/NMR_data_T1.pdf}
    \caption{Signalintensitäten für verschiedene $\tau$-Werte der \ch{CuSO4}-Probe.} 
    \label{fig: T1_Mess}
\end{figure}
Da die Signalintensitäten aus \autoref{fig: T1_Mess} mit de Magnetisierung $M_z(\tau)$ proportional sind, kann die Relaxationszeit $T_1$ durch \autoref{eq: T1} bestimmt werden.
Dazu wird die Gleichung logarithmiert, wodurch sich \autoref{eq: ln_T1} ergibt.
\begin{equation}
    \ln \left(\frac{M_0 - M_z(\tau)}{2 M_0} \right) = - \frac{\tau}{T_1}
    \label{eq: ln_T1}   
\end{equation}
Als $M_0$ wird der Betrag des ersten gemessenen Intensitätswertes bei $\tau = 0,5$ ms verwendet, da dieser der negativen Gleichgewichtsmagnetisierung entspricht.
Durch Auftragung von $\ln \left(\frac{M_0 - M_z(\tau)}{2 M_0} \right)$ gegen $\tau$ lässt sich \autoref{fig: ln_T1} erstellen. Zusätzlich ist in \autoref{fig: ln_T1} eine Fitgerade im Bereich bis $\tau = 140,5$ ms eingetragen, da die Messwerte ab diesem Wert nicht mehr linear verlaufen.

\begin{figure}[H]
    \centering
    \includegraphics[width=0.8\textwidth]{A05/Paramagnetisch/T1 Inversion/NMR_data_ln_Fit.pdf}
    \caption{ln-Plot der \ch{CuSO4}-Probe.} 
    \label{fig: ln_T1}
\end{figure}
Aus der Geradengleichung $y = m \cdot \tau + c$ des Fits mit $m = -0,01549$ und $c = 0,0728$ lässt sich die Spin-Gitter-Relaxationszeit $T_1$ nach \autoref{eq: ln_T1} berechnen, da die Steigung $m$ dem Wert $-\frac{1}{T_1}$ entspricht.
\begin{equation}
    T_1 = - \frac{1}{m} = - \frac{1}{-0,01549} = \SI{64,59}{ms}
\end{equation}


%\subsection{Carr-Purcell-Experiment}

\begin{table}[H]
    \centering
    \begin{tabular}{c|c}
        $\tau$ / ms & Echoamplitude / \% \\ 
		\hline
         &  \\ 

	\label{tab: T2}
    \end{tabular}
\end{table}
\newpage

% T1 = 64.59 ms

\section{Fehlerrechnung} 

\section{Zusammenfassung}
Im ersten Versuchsteil wurde 

\printbibliography[title={Literatur}]


\end{document}
