\documentclass[a4paper,12pt,bibliography=totocnumbered]{scrartcl}

\usepackage[utf8]{inputenc} 
\usepackage[T1]{fontenc}
\usepackage[english]{babel}
\usepackage{amsmath, amssymb,amsfonts}
\usepackage{graphicx}
\usepackage{csquotes}
\usepackage[bookmarks,colorlinks=true]{hyperref}
\usepackage{geometry}
\usepackage{float}
\usepackage[final]{pdfpages}
\usepackage{framed, color} 
\usepackage{scrlayer-scrpage}
\usepackage{siunitx}
\usepackage{subfigure}
%\renewcaptionname{ngerman}{\figurename}{Fig.}
%\renewcaptionname{ngerman}{\tablename}{Table}
\sisetup{detect-weight=true, detect-family=true,locale=UK,range-phrase={\.bis\.},list-final-separator ={\,\linebreak[0] \text{und}\,},separate-uncertainty=true,per-mode = symbol-or-fraction}
\DeclareSIUnit\curie{Ci}
\usepackage[backend=biber, style=chem-angew]{biblatex} 
\addbibresource{lit.bib} 

\usepackage{chemgreek}
\usepackage{chemformula}
\sisetup{detect-weight=true, detect-family=true,locale=DE,range-phrase={\,bis\,},list-final-separator ={\,\linebreak[0] \text{und}\,},separate-uncertainty=true,per-mode = symbol-or-fraction}
%macht komata anstatt kreuze bei Zehnerpotenzen
\geometry{left = 2.5cm} \geometry{top = 3cm}

\urlstyle{same}
%Hyperlinks-Setup
\hypersetup{
	colorlinks,
	linktocpage,
	citecolor=black,
	filecolor=black,
	linkcolor=black,
	urlcolor=black
}

%\numberwithin{equation}{section}

\setlength{\parindent}{0 mm}
\setlength{\parskip}{2 mm} 



\pagestyle{scrheadings}
%Header oben links auf linker Seite (ungerade Seitenzahl) und oben rechts auf rechter Seite (gerade Seitenzahl), beinhaltet gruppennummer und Versuchskürzel. Im Fall eine einseitigen Dokuments: Header oben rechts
\ihead{\VERSUCHSNR} %Header oben rechts auf linker Seite und oben links auf rechter Seite. Beinhaltet die Namen der Verfasser. Im Fall eine einseitigen Dokuments: Header oben links!
\ohead{\GRUPPENNR}
\ofoot{\thepage} 
\cfoot{\empty}  
\ifoot{\empty} 


\newcommand{\VERSUCHSDATUM}{14.01.2026}
\newcommand{\PROTOKOLLDATUM}{\today}

\newcommand{\VerfasserEINS}{Vincent Kümmerle}
\newcommand{\MatNoEINS}{3712667}
\newcommand{\EmailEINS}{st187541@stud.uni-stuttgart.de}
\newcommand{\StudiengangEINS}{B.Sc. Chemie}

\newcommand{\VerfasserZWEI}{Elvis Gnaglo}
\newcommand{\MatNoZWEI}{3710504}
\newcommand{\EmailZWEI}{st189318@stud.uni-stuttgart.de}
\newcommand{\StudiengangZWEI}{B.Sc. Chemie}

\newcommand{\VerfasserDREI}{Julian Brügger}
\newcommand{\MatNoDREI}{3715444}
\newcommand{\EmailDREI}{st190050@stud.uni-stuttgart.de}
\newcommand{\StudiengangDREI}{B.Sc. Chemie}

\newcommand{\BETREUER}{Xiangyin Tan}
\newcommand{\GRUPPENNR}{A05}

\newcommand{\VERSUCHSNR}{Diffusion}
\newcommand{\VERSUCHSNAME}{Determination of Diffusion Constants using the Schlieren Method}


\begin{document}
\thispagestyle{empty}


\begin{titlepage}

\begin{center}
\Huge{\textbf{\VERSUCHSNR\ - \VERSUCHSNAME}}\\
\vspace{10mm}% Abstand
\Large{Protocol for the PC 2 lab course by \\ \textbf{\VerfasserEINS\;\& \VerfasserZWEI\;\& \VerfasserDREI}}\\
\vspace{10mm} 
\Large{University of Stuttgart}\\
\end{center}
\vspace{0cm}
\begin{center}
\begin{tabular}{ll}
\large{authors:}		& \large{\VerfasserEINS,} \large{\MatNoEINS} \\
 						& \large{\EmailEINS} \\
						\vspace{0cm}\\
						& \large{\VerfasserZWEI,} \large{\MatNoZWEI} \\
                        & \large{\EmailZWEI} \\
						\vspace{0cm}\\
						& \large{\VerfasserDREI,} \large{\MatNoDREI} \\
                        & \large{\EmailDREI} \\
						\vspace{0cm}\\
\large{group number:}	& \large{\GRUPPENNR} \\
\vspace{0cm}\\
\large{date of experiment:}	& \large{\VERSUCHSDATUM} \\
\vspace{0cm}\\
\large{supervisor:}		& \large{\BETREUER} \\
\vspace{0cm}\\
\large{submission date:} & \large{\PROTOKOLLDATUM}
\end{tabular}
\end{center}

\vspace{1cm}


\textbf{Abstract:}


\end{titlepage}


\thispagestyle{empty}

\tableofcontents 

\clearpage

\renewcommand{\thepage}{\arabic{page}}
\setcounter{page}{1}


\section{Theory}
Molecular diffusion describes the movement of particles from a region of high concentration to a region of low concentration, which leads to an even distribution of particles.\supercite{Skript}
It was first discovered by Robert Brown, who observed the random movement of tiny particles suspended in a liquid, which is called Brownian motion. Then German physicist Adolf Fick lay the theoretical foundation for diffusion by formulating two laws. \\
%\subsection{Fick's Laws}
The first Fick's law describes how the particle flux density $j(z)$ is connected to the Diffusion coefficient $D$ and the concentration gradient $\frac{\partial c(z)}{\partial z}$, which is shown in \autoref{eq:2.8}.
\begin{equation} 
j(z) = -D \cdot \frac{\partial c}{\partial z} 
\label{eq:2.8} 
\end{equation}
The physical meaning of the first Fick's law is that a concentration gradient leads to mutual diffusion of particles, leading to an even distribution.
The diffusion coefficient can also be defined by the ,,Einstein relation'' in \autoref{eq:2.7}.
\begin{equation}
    D = \frac{1}{2}\nu \Delta z^2
    \label{eq:2.7}
\end{equation}
$\nu$ is the jump rate and $\Delta z^2$ the mean square jump distance of particles, assuming that they can only move along the z axis.
Second Fick's law describes the temporal course of concentration as shown in \autoref{eq:2.11}.
\begin{equation} 
\frac{\partial c}{\partial t} = D \cdot \frac{\partial^2 c}{\partial z^2} \label{eq:2.11} 
\end{equation}
\autoref{eq:2.11} states that the concentration equalization proceeds faster the stronger the concentration gradient.
By integrating seconds Fick's law from \autoref{eq:2.11} under certain boundary conditions and using the substition $\mu = \frac{x}{\sqrt{4Dt}}$, \autoref{eq:2.14} for the profile of concentration can be obtained.
\begin{equation} 
c = \frac{c_2}{2} \cdot \left(1-\frac{2}{\sqrt{\pi}}\int_{0}^{\mu} e^{-\mu^2} \,d \nu \right)
\label{eq:2.14} 
\end{equation}
By deriving \autoref{eq:2.14}, the function is simplified into \autoref{eq:2.15} which leads to the concentration gradient profile having the shape of a Gaussian bell curve, as is illustrated in \autoref{fig: profiles}. 
\begin{equation} 
\frac{dc}{dz} = -\frac{c_2}{2\sqrt{\pi Dt}} \cdot e^{-\mu^2} 
\label{eq:2.15} 
\end{equation}
Assuming that the refractive index $n$ is proportional to $c$ and knowing that the gradient has an extreme value at $z=0$, as can be seen in \autoref{fig: profiles}, \autoref{eq:2.18} describes the refractive index gradient at $z=0$ with the indices of water $n_1$ and of the salt solution $n_2$.\supercite{Skript}
\begin{equation} 
\left( \frac{dn}{dz} \right)_{z=0} = -\frac{n_1 - n_2}{2\sqrt{\pi Dt}} \label{eq:2.18} 
\end{equation}

\begin{figure}[H]
    \centering
    \includegraphics[width=0.65\textwidth]{Bilder/profiles.png}
    \caption{Profiles of the concentration and concentration gradient.\supercite{Skript}} 
    \label{fig: profiles}
\end{figure}

%\subsection{The Schlieren Method}
Following this argumentation, diffusion leads to a continuously changing refractive index profile which causes a light beam directed to a cuvette of a salt solution with a concentration gradient to be refracted toward the optically denser medium. That is called Snell's law of refraction and shown in \autoref{fig: setup}. 
The exit angle $\beta$ und the degree of curvature can be described with \autoref{eq:2.21} and \ref{eq:2.19}, where $\alpha$ is the angle of incidence and $r$ the radius of the curvature of the light beam.\supercite{Skript}
\begin{align}
    \beta &= \frac{n}{n_0} \alpha \label{eq:2.21} \\
    \frac{1}{r} &= \frac{1}{n} \frac{dn}{dz} = \frac{d \ln(n)}{dz} \label{eq:2.19}
\end{align}

\begin{figure}[H]
    \centering
    \includegraphics[width=0.45\textwidth]{Bilder/setup.png}
    \caption{Scheme of the experimental setup geometry.\supercite{Skript}} 
    \label{fig: setup}
\end{figure}
The Huygen's principle states that every point of a wavefront can be seen a the origin of a spherical elementary wave. That's why the path of the light beam is curved in a medium with a spatially dependent refractive index $n(z)$ due to a lateral phase shift within the wavefront. The strongest deflection is experienced by the part of the beam that passes through the steepest concentration gradient region.
With the small-angle approximation and \autoref{eq:2.21} and \ref{eq:2.19}, the refractive index gradient can be described by \autoref{eq:2.23}.
\begin{equation}
    \frac{dn}{dz} \approx \frac{Z}{L} \cdot \frac{n_0}{K}
    \label{eq:2.23}
\end{equation}
$L$ is the distance between the cuvette and the screen, $K$ the thinkness of the cuvette, $n_0$ the refractive index of air and $Z$ the distance between the straight extension of the beam at the exit point and the impact point on the screen.
By equating \autoref{eq:2.23} and \autoref{eq:2.18} and rearranging to the diffusion constant $D$, \autoref{eq:2.24} is derived to calculate $D$ using the Schlieren Method.\supercite{Skript}
\begin{equation} 
D = \frac{(n_1-n_2)^2L^2K^2}{4\pi n_0^2 Z^2 t}
\label{eq:2.24} 
\end{equation}



\section{Procedure}

The experimental setup was already built prior to the start of the measurements as shown in \autoref{fig: setup}. The laser beam was light path was tilted at
about 45°.
\begin{figure}[H]
    \centering
    \includegraphics[width=0.85\textwidth]{Bilder/setupcam.png}
    \caption{Scheme of the laser beam geometry.\supercite{Skript}} 
    \label{fig: setupcam}
\end{figure}
The experiment began with 




\section{Analysis}

To evaluate \autoref{fig: }, \autoref{eq:2.24} is rearranged to $Z^{-2}$ to get \autoref{eq:Plot Formel}, which shows the connection between the slope $m$ and the diffusion coefficient.
\begin{align} 
Z^{-2} &= \frac{4\pi D n_0^2}{(n_1-n_2)^2L^2K^2} \cdot t \\
&= m \cdot t
\label{eq:Plot Formel} 
\end{align}

\section{Error Discussion}

\section{Conclusion}


\printbibliography[title={References}]


\end{document}
