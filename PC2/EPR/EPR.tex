\documentclass[a4paper,12pt,bibliography=totocnumbered]{scrartcl}

\usepackage[utf8]{inputenc} 
\usepackage[T1]{fontenc}
\usepackage[ngerman]{babel} 
\usepackage{amsmath, amssymb,amsfonts}
\usepackage{graphicx}
\usepackage{csquotes}
\usepackage[bookmarks,colorlinks=true]{hyperref}
\usepackage{geometry}
\usepackage{float}
\usepackage[final]{pdfpages}
\usepackage{framed, color} 
\usepackage{scrlayer-scrpage}
\usepackage{siunitx}
\usepackage{subfigure}
\renewcaptionname{ngerman}{\figurename}{Abb.}
\sisetup{detect-weight=true, detect-family=true,locale=DE,range-phrase={\,bis\,},list-final-separator ={\,\linebreak[0] \text{und}\,},separate-uncertainty=true,per-mode = symbol-or-fraction}
\DeclareSIUnit\curie{Ci}
\usepackage[backend=biber, style=chem-angew]{biblatex} 
\addbibresource{lit.bib} 

\usepackage{chemgreek}
\usepackage{chemformula}
\geometry{left = 2.5cm} \geometry{top = 3cm}

\urlstyle{same}
%Hyperlinks-Setup
\hypersetup{
	colorlinks,
	linktocpage,
	citecolor=black,
	filecolor=black,
	linkcolor=black,
	urlcolor=black
}

%\numberwithin{equation}{section}

\setlength{\parindent}{0 mm}
\setlength{\parskip}{2 mm} 



\pagestyle{scrheadings}
%Header oben links auf linker Seite (ungerade Seitenzahl) und oben rechts auf rechter Seite (gerade Seitenzahl), beinhaltet gruppennummer und Versuchskürzel. Im Fall eine einseitigen Dokuments: Header oben rechts
\ihead{\VERSUCHSNR} %Header oben rechts auf linker Seite und oben links auf rechter Seite. Beinhaltet die Namen der Verfasser. Im Fall eine einseitigen Dokuments: Header oben links!
\ohead{\GRUPPENNR}
\ofoot{\thepage} 
\cfoot{\empty}  
\ifoot{\empty} 


\newcommand{\VERSUCHSDATUM}{15.12.2025}
\newcommand{\PROTOKOLLDATUM}{\today}

\newcommand{\VerfasserEINS}{Vincent Kümmerle}
\newcommand{\MatNoEINS}{3712667}
\newcommand{\EmailEINS}{st187541@stud.uni-stuttgart.de}
\newcommand{\StudiengangEINS}{B.Sc. Chemie}

\newcommand{\VerfasserZWEI}{Elvis Gnaglo}
\newcommand{\MatNoZWEI}{3710504}
\newcommand{\EmailZWEI}{st189318@stud.uni-stuttgart.de}
\newcommand{\StudiengangZWEI}{B.Sc. Chemie}

\newcommand{\VerfasserDREI}{Julian Brügger}
\newcommand{\MatNoDREI}{3715444}
\newcommand{\EmailDREI}{st190050@stud.uni-stuttgart.de}
\newcommand{\StudiengangDREI}{B.Sc. Chemie}

\newcommand{\BETREUER}{Valentin Bayer}
\newcommand{\GRUPPENNR}{A05}

\newcommand{\VERSUCHSNR}{ESR}
\newcommand{\VERSUCHSNAME}{Elektronenspin-Resonanz-Spektroskopie}


\begin{document}
\thispagestyle{empty}


\begin{titlepage}

\begin{center}
\Huge{\textbf{\VERSUCHSNR\ - \VERSUCHSNAME}}\\
\vspace{10mm}% Abstand
\Large{Protokoll zum Versuch des PC 2 Praktikums von \\ \textbf{\VerfasserEINS\;\& \VerfasserZWEI\;\& \VerfasserDREI}}\\
\vspace{10mm} 
\Large{Universität Stuttgart}\\
\end{center}
\vspace{1cm}
\begin{center}
\begin{tabular}{ll}
\large{Autoren:}		& \large{\VerfasserEINS,} \large{\MatNoEINS} \\
 						& \large{\EmailEINS} \\
 						\vspace{0cm}\\
						& \large{\VerfasserZWEI,} \large{\MatNoZWEI} \\
                        & \large{\EmailZWEI} \\
						\vspace{0cm}\\
						& \large{\VerfasserDREI,} \large{\MatNoDREI} \\
                        & \large{\EmailDREI} \\
						\vspace{0cm}\\
\large{Gruppennummer:}	& \large{\GRUPPENNR} \\
\vspace{0cm}\\
\large{Versuchsdatum:}	& \large{\VERSUCHSDATUM} \\
\vspace{0cm}\\
\large{Betreuer:}		& \large{\BETREUER} \\
\vspace{0cm}\\
\large{Erstabgabedatum:} & \large{\PROTOKOLLDATUM}
\end{tabular}
\end{center}
\vspace{1cm}
\textbf{Abstract:}
In diesem Versuch wurden verschiedene Proben mit Elektronenspin-Resonanz-Spektroskopie untersucht. Dabei wurde der g-Faktor der DPPH-Probe als $g_\mathrm{DPPH} = 2{,}0022 \pm 0{,}0045$ bestimmt.


\end{titlepage}


\thispagestyle{empty}

\tableofcontents 

\clearpage

\renewcommand{\thepage}{\arabic{page}}
\setcounter{page}{1}


\section{Theorie}
Die Elektronenspinresonanz ist eine spektroskopische Methode, die sich mit der Untersuchung von Substanzen befasst, die ein ungepaartes Elektron aufweisen. Darunter fallen paramagnetische Verbindungen, organische Radikale und Übergangsmetallkomplexe. Dabei kann das Elektron mit der Spinquantenzahl $s = \frac{1}{2}$ zwei Ausrichtungsrichtungen haben. Diese werden durch die Magnetspinquantenzahl $m_s$ beschrieben und als ''Spin up'' ($m_s = + \frac{1}{2}$) und ''Spin down'' ($m_s = - \frac{1}{2}$) bezeichnet. Diese Zustände sind energetisch entartet, doch in der ESR-Spektroskopie werden diese durch Anlegen eines externen Magnetfelds aufgehoben. Diese Aufhebung der Entartung nennt sich Zeeman-Effekt. Die Energiedifferenz der beiden Niveaus lässt sich durch die Gleichung
\begin{equation}
    \Delta E = E_{+\frac{1}{2}} - E_{-\frac{1}{2}} = g \mu_B B_0
    \label{espin}
\end{equation}
berechnen. Dabei ist $g$ der g-Faktor, $\mu_B = \SI{9,2741e-24}{JT^{-1}}$ das Bohr'sche Magneton und $B_0$ die Magnetfeldstärke. Wird ein weiteres Magnetfeld, welches senkrecht zum statischen steht, angelegt, so richtet sich der Spin des Elektrons ebenfalls nach diesem aus. Dieses Phänomen wird als magnetische Resonanz bezeichnet, wobei die Resonanzbedingung auf der allgemeinen Bohrschen Frequenzbedingung basiert. Sie lässt sich durch die Formel
\begin{equation}
    \Delta E =  g \mu_B B_0 = h \nu
    \label{eres}
\end{equation}
beschreiben, wobei $h = \SI{6.6261e-34}{Js}$ das Plank'sche Wirkungsquantum und $\nu$ die Mikrowellenfrequenz ist.
\subsection{g-Faktor}
Der g-Faktor, auch Landé-Faktor, ist ein Maß zur Angabe des effektiven magnetischen Moments. Er ist direkt mit dem Eigendrehimpuls und Bahndrehimpuls eines Elektrons verknüpft und liegt für ein freies Elektron bei $g_e = \SI{2,002319}{}$. Bei vielen ESR-aktiven Verbindungen liegt der g-Faktor eines ungepaarten Elektrons auf Grund von gequenchter Spin-Bahn-Kopplung nahe an dem eines freien Elektrons, doch bei Übergangsmetallkomplexen ist diese nicht gequencht, weswegen es dort häufig zu großen Abweichungen kommt. Die Bestimmung des g-Faktors für eine unbekannte Substanz erfolgt durch Messung der Verschiebung der zentralen Resonanzlinie zu der eines Standards mit bekanntem g-Faktor. Dabei wird die \autoref{eq: gB} nach dem g-Faktor der unbekannten Substanz umgestellt, um \autoref{eq: gx} zu erhalten. 
\begin{equation}
    g_x \cdot B_x = g_S \cdot B_S
    \label{eq: gB}
\end{equation}
\begin{equation}
    g_x = g_S \cdot \frac{B_S}{B_x}
    \label{eq: gx}
\end{equation}
Dabei ist $g_x$ der g-Faktor der unbekannten Substanz, $g_S$ der g-Faktor der bekannten Substanz, $B_x$ die Resonanzfeldstärke der unbekannten Substanz und $B_S$ die Resonanzfeldstärke der bekannten Substanz.
\subsection{Hyperfeinstruktur}
Neben der Elektron-Zeeman-Wechselwirkung mit dem äußeren Magnetfeld treten ebenfalls Wechselwirkungen mit den Kernspins der anwesenden Kerne auf. Diese Wechselwirkungen werden als Kern-Zeeman-Effekt bezeichnet und lassen sich durch \autoref{eq: Zeh} beschreiben.
\begin{equation}
    E = g \mu_B \overrightarrow{B}_0 \overrightarrow{S} - g_K \mu_K \overrightarrow{B}_0 \overrightarrow{I} + a \overrightarrow{I} \cdot \overrightarrow{s}
    \label{eq: Zeh}
\end{equation}
Dabei ist $a$ die isotrope Hyperfeinstrukturkopplungskonstante, $\overrightarrow{I}$ der Kernspin, $\overrightarrow{s}$ der Elektronenspin, $g_K$ der Kern-g-Faktor und $\mu_K$ das Kernmagneton. 
Diese zusätzlichen Wechselwirkungen führen zu einer weiteren Aufspaltung der Energieniveaus und werden als Hyperfeinstruktur bezeichnet. Das einfache Spinsystem besitzt nun vier mögliche Energieeigenwerte aus denen sich die in \autoref{eq: aus} dargestellten Auswahlregeln für ESR-Übergänge ableiten lassen.
\begin{equation}
\Delta m_s = \pm 1~~~~~~~~\Delta m_I = 0
    \label{eq: aus}
\end{equation}
Durch die Kopplung der Elektronenspins mit den Kernspins werden, abhängig von der Anzahl der Kernspins, mehrere Übergänge mit unterschiedlichen Energien möglich. Diese werden im Spektrum durch Multipletts sichtbar. Sind die Kerne äquivalent, dann lassen sich die erwarteten Multipletts durch \autoref{eq: äqKern} berechnen.
\begin{equation}
    N = 2 n I + 1
    \label{eq: äqKern}
\end{equation}
Dabei ist $N$ die Zahl der Multipletts, $n$ die Anzahl der äquivalenten Kerne und $I$ der Kernspin. Sind die Kerne nicht äquivalent, so wird, wie in \autoref{eq: ineqKern} dargestellt, das Multiplett für jede Kernsorte berechnet und anschließend multipliziert. 
\begin{equation}
    N = \prod_i (2 n_i I_i + 1)
    \label{eq: ineqKern}
\end{equation}
\subsection{Heisenbergscher Spinaustausch}
Der Heisenbergsche Spinaustausch beschreibt die Umkehr des Elektronenspins beim Zusammenstoß von Radikalen in Lösung. Dabei kommt es zu einer Überlappung der Orbitale der ungepaarten Elektronen. Dies führt zu einer Verkürzung der Lebenszeit des Energiezustandes, wodurch die Unschärfe der Energie zunimmt. Dadurch werden im ESR-Spektrum die Linien so lange breiter, bis sie eine einzelne Linie bilden.

\newpage

\section{Versuchsdurchführung}
Es wurden verschiedene Proben im ESR-Spektrometer untersucht, wobei jede Probe in den Resonator eingeführt wurde und nach Abgleichung der Mikrowellenbrücke und Einstellung der Parameter gemessen wurde.
Der Aufbau des Spektrometers ist in \autoref{fig: setup} dargestellt.

\begin{figure}[H]
    \centering
    \includegraphics[width=0.8\textwidth]{Bilder/setup.png}
    \caption{Schema des Versuchsaufbaus mit Mikrowellen-Generator, Zirkulator, Resonator und Diode.\supercite{Skript}} 
    \label{fig: setup}
\end{figure}

\subsection{2,2-Diphenyl-1-pikrylhydrazyl}
Von der 2,2-Diphenyl-1-pikrylhydrazyl(DPPH)-Probe wurde zuerst ein ESR-Spektrum bei der Standardeinstellung des Geräts mit folgenden Parametern aufgenommen: $B_0$: 338 mT, $Sweep$: 5,0 mT, $Sweep~time$: 60 s, $Modulation$: 0,01 mT, $MW attenuation$: 20,0 dB, $Gain$: $5 \cdot 10^0$, $Smooth$: 0,1 s. Dann wurden die Parameter $B_0-Field$, $Sweep$ und $Gain$ so lange verändert, bis das Spektrum zentriert, die Bildschirmbreite gut genutzt und die Spektrenhöhe optimiert war. Dies war bei den Parametern $B_0$: 337.91 mT, $Sweep$: 1.3 mT, $Gain$: $3 \cdot 10^1$ der Fall.
% 137.91 mT, Sweep: 1.3 mT, Gain:3*10^1
Dann wurde zuerst die Mikrowellenleistung mit jeder Messung durch Abschwächung der Mikrowellenstrahlung (MW attenuation) um 3 dB verdoppelt und anschließend die Modulationsamplitude von 0,01  auf 0,64 mT durch Verdopplung von Messung zu Messung variiert.
Die Parameter dieser Messungen sind mit den Parametern der restlichen Versuchsteile in \autoref{tab: parameter} aufgeführt.

\begin{table}[H]
\centering
\caption{Experimentelle Parameter der ESR Messungen verschiedener Proben. \\}
\begin{tabular}{c|c|c|c|c|c}
Probe & $B_0$-Field & Sweep & Modulation & MW attenuation & Gain \\
& / mT & / mT & / mT & / dB & - \\
\hline
DPPH - start        & 338 & 5   & 0,01        & 20     & $5\cdot10^{0}$ \\
DPPH - optimal        & 337,91 & 1,3   & 0,01        & 20     & $3\cdot10^{1}$ \\
DPPH - power var.   & 337,91 & 1,3   & 0,01       & 20 - 5    & $3\cdot10^{1}$ \\
DPPH - modul. var.    & 337,91 & 1,3   & 0,01 - 0,64    & 30     & $3\cdot10^{1}$ \\
\hline
Cu(AcAc)$_2$ + DPPH  & 320 & 60  & 0,4         & 20     & $5\cdot10^{1}$ \\
VO(AcAc)$_2$ + DPPH  & 340 & 100 & 0,2         & 30     & $5\cdot10^{1}$ \\
\hline
Galvinoxyl           & 337 & 5   & 0,1         & 10     & $5\cdot10^{1}$ \\
Galvinoxyl deox      & 337 & 5   & 0,03        & 10     & $5\cdot10^{1}$ \\
\end{tabular}
\label{tab: parameter}
\end{table}


\subsection{Übergangsmetallkomplexe}


\subsection{Galvinoxylradikal}
Im letzten Versuchsteil wurde eine Probe aus Galvinoxyl und DPPH in Toluol (5 ml) gelöst und ein kleines Volumen dieser Probe im ESR-Spektrometer untersucht. Anschließend wurde die Lösung eingefroren, unter vermindertem Druck entgast und wieder aufgetaut. Dieser Schritt wurde fünf Mal wiederholt und die Probe wurde erneut im Spektrometer untersucht. Da das Spektrum nach fünf Zyklen noch nicht der Erwartung entsprach, wurden zwei weiter Zyklen durchgeführt, um restlichen Sauerstoff zu entfernen.Zum Schluss wurde das Spektrum von Galvinoxyl mit EPRsim simuliert um einen Vergleich zu idealen Bedingungen zu schaffen.

\section{Auswertung}

Für die Auswertung ist zu beachten, dass der Wert des $B_0$-Felds aufgrund der näheren Position der Hallsonde zum Magneten als der Position der Probe 1,258 mT höher ist als der reale Wert für $B_0$.
Somit ergeben sich für die $B_0$-Werte aus \autoref{tab: parameter} folgende reale $B_0$-Werte:
\begin{table}[H]
\centering
\caption{Reale $B_0$-Werte für die verschiedenen Messungen. \\}
\begin{tabular}{c|c}
Probe & $B_0$-Field / mT \\
\hline
DPPH - start        & 336,742 \\
DPPH - optimal        & 336,652 \\
DPPH - power var.   & 336,652 \\
DPPH - modul. var.    & 336,652 \\
Cu(AcAc)$_2$ + DPPH  & 318,742 \\
VO(AcAc)$_2$ + DPPH  & 338,742  \\
Galvinoxyl           & 335,742 \\
Galvinoxyl deox & 335,742 \\
\end{tabular}
\label{tab: zentren}
\end{table}

\subsection{ESR-Spektren von 2,2-Diphenyl-1-pikrylhydrazyl}
\autoref{fig: DPPH-start} zeigt die aufgenommenen ESR-Spektren der DPPH-Probe mit den Startparametern und optimalen Parametern aus \autoref{tab: parameter}.

\begin{figure}[H]
    \centering
    \includegraphics[width=1.0\textwidth]{DPPH/A)/01_DPPH_start_0p01mod_20db_5gain_471701.pdf}
    \caption{ESR-Spektrum der DPPH-Probe mit Standardparametern und optimalen Parametern.} 
    \label{fig: DPPH-start}
\end{figure}
Dabei wurde der Wert für $Gain$ versechsfacht, um die Spektrenhöhe zu vergrößern, $B_0$-Field leicht verringert für die Zentrierung des Spektrums und $Sweep$ auf 1,3 mT herabgesetzt, um nur den relevanten Bereich des ESR-Spektrums zu messen. \\
\autoref{fig: DPPH-power-var} zeigt die aufgenommenen ESR-Spektren der DPPH-Probe mit Werten der MW attenuation zwischen 20 dB und 5 dB. Dabei verdoppelt sich die Mikrowellenleistung von Messung zu Messung in 3 dB Schritten.

\begin{figure}[H]
    \centering
    \includegraphics[width=1.0\textwidth]{DPPH/B)/DPPH_MW_Attenuation.pdf}
    \caption{ESR-Spektren der DPPH-Probe mit variierender MW attenuation zwischen 20 dB und 5 dB.} 
    \label{fig: DPPH-power-var}
\end{figure}
Aus \autoref{fig: DPPH-power-var} geht hervor, dass die Spektrenhöhe mit höherer Mikrowellenleistung ansteigt, da das ESR-Spektrum bei geringeren Werten der MW attenuation durch eine geringere Abschwächung der Mikrowellen-Strahlung eine höhere Signalintensität aufweist. Zudem lässt sich beobachten, dass auch die Spektrenbreite mit zunehmender Mikrowellenleistung zunimmt.\\
\autoref{fig: DPPH-modul-var} zeigt die aufgenommenen ESR-Spektren der DPPH-Probe mit Werten der Modulationsamplitude zwischen 0,01 mT und 0,064 mT mit Verdopplung von Messung zu Messung.

\begin{figure}[H]
    \centering
    \includegraphics[width=1.0\textwidth]{DPPH/C)/DPPH_Modulation.pdf}
    \caption{ESR-Spektren der DPPH-Probe mit variierender Modulationsamplitude zwischen 0,01 mT und 0,064 mT.} 
    \label{fig: DPPH-modul-var}
\end{figure}
% Spektrenhöhe & Linienbreite
In \autoref{fig: DPPH-modul-var} lässt sich erkennen, dass die Spektrenhöhe mit zunehmender Modulationsamplitude ansteigt. Jedoch ist dieser Anstieg nicht linear, sondern flacht bei höheren Amplituden ab 0,32 mT ab. Die Linienbreite nimmt von der Modulationsamplitude 0,01 mT bis 0,64 mT zu.

Zur Bestimmung des g-Faktors von DPPH werden die Lagen aller Spektrenzentren der Messungen mit variierender MW-Abschwächung und Modulationsamplitude in \autoref{tab: zentren} aufgeführt und der Mittelwert berechnet. 

\begin{table}[H]
\centering
\caption{Lage der Spektrenzentren mit den zugehörigen Parametern.}
\begin{tabular}{c|c|c}
Modulation / mT & MW attenuation / dB & Spektrenzentrum / mT \\
\hline
0,01 & 20 & 336,647 \\
0,01 & 17 & 336,657 \\
0,01 & 14 & 336,662 \\
0,01 & 11 & 336,662 \\
0,01 & 8  & 336,667 \\
0,01 & 5  & 336,667 \\
0,01 & 30 & 336,647 \\
0,02 & 30 & 336,652 \\
0,04 & 30 & 336,662 \\
0,08 & 30 & 336,652 \\
0,16 & 30 & 336,652 \\
0,32 & 30 & 336,652 \\
0,64 & 30 & 336,647
\end{tabular}
\label{tab:zentren}
\end{table}
Die Lage des Spektrenzentrums von DPPH befindet sich also am Mittelwert von $B_0 = \SI{336,656}{mT}$.
Da als halbe Frequenz auf dem Display des Frequenzzählers 4,71701 GHz angezeigt wurde, wird für die Berechnung des g-Faktors die Frequenz $\nu = \SI{9,43402}{GHz}$ verwendet.
Durch Einsetzen des Bohrschen Magnetons $\mu_\mathrm{B}$ und des Planck'schen Wirkungsquantum zusammen mit dem Spektrenzentrum und der Frequenz in \autoref{eq:g}, die sich durch Umstellen der Gleichung der Resonanzbedingung \ref{eq:eres} nach g ergibt, wird der g-Faktor berechnet.
\begin{align}
	g &= \frac{h \cdot \nu}{\mu_\mathrm{B} \cdot B_0}
    \label{eq:g} \\
	g_\mathrm{DPPH}&= \frac{\SI{6,6261e-34}{Js} \cdot \SI{9,43402e9}{Hz}}{\SI{9,2741e-24}{\frac{J}{T}} \cdot \SI{336,656e-3}{T}} \nonumber \\
	&= 2{,}0022 \nonumber
\end{align}

Mit dem Literaturwert von DPPH von $g_{\mathrm{Lit}} = 2{,}0037$ ergibt sich eine absolute Abweichung von 0,0015.\supercite{Skript} Die Ungenauigkeit des g-Faktors wird in der Fehlerrechnung bestimmt.
Die Wellenlänge der verwendeten Mikrowelle lässt sich aus der Frequenz $\nu = \SI{9,43402}{GHz}$ mit der Lichtgeschwindigkeit berechnen.
\begin{align*}
	\lambda = \frac{c}{\nu} = \frac{\SI{2,99792}{\frac{m}{s}}}{\SI{9,43402e9}{Hz}} = \SI{0,0318}{m}
\end{align*}
Daraus lässt sich schließen, dass der Resonator 3,18 cm lang sein muss, um eine stehende Welle zu generieren, da dann die Wellenlänge der Mikrowelle und die Resonatorlänge ideal zueinander passen.

\subsection{ESR-Spektren von Kupfer- und Vanadylacetylacetonat}

Das aufgenommene ESR-Spektrum von Kupfer(II)acetylacetonat \ch{Cu(AcAc)2}  ist in \autoref{fig: Cu} dargestellt. Dabei ist anzumerken, dass der Korrekturfaktor für $B_0$ hier bereits miteinberechnet worden ist.

\begin{figure}[H]
\centering
\includegraphics[width = 1.0\textwidth]{Cu_VO/Kupfergraph.pdf}
\caption{Das ESR-Spektum der Kupfer(II)acetylacetonat-Probe mit internem DPPH Standard.}
\label{fig: Cu}
\end{figure}

In \autoref{fig: Cu} sind fünf Peaks zu beobachten. Dabei besitzt der Peak mit dem höchsten Magnetfeld eine deutlich höhere Absorptionsamplitude als die anderen vier Peaks.
Aufgrund der Position des höheren Peaks bei $B_0 = \SI{336,66}{mT}$ kann dieser dem internen Standard DPPH zugeordnet werden. Somit sind der Kupfer(II)acetylacetonat-Probe die vier Peaks mit der kleineren Amplitude zuzuordnen.
Die Positionen der Peaks, die sich der Kupferprobe zuordnen lassen, liegen bei $B_{1,\text{Cu}} \approx \SI{304,9}{mT}$, $B_{2,\text{Cu}} \approx \SI{313,2}{mT} $, $B_{3,\text{Cu}} \approx \SI{321,5}{mT}$ und $B_{4,\text{Cu}} \approx \SI{329,7}{mT}$. Das Spektrum der Kupferprobe ist hierbei um den Mittelwert von dem ersten und dem letzten Peak der Kupferprobe zentriert. Dieses Zentrum des Spektrums berechnet sich wie folgt:

\begin{equation*}
\overline{B_{\text{Cu}}} = \frac{\SI{329,7}{\milli\tesla} + \SI{304,9}{\milli\tesla}}{2} = \SI{317,2}{\milli\tesla}
\end{equation*}

Das Spektrum der Kupferprobe ist somit um den den Wert von $ \overline{B_{\text{Cu}}} = \SI{317,2}{\milli\tesla} $ zentriert.

Ausgehend von diesem Magnetfeld kann nun der $g$-Faktor mit der Mikrowellenfrequenz $ \nu \approx \SI{9,44e9} {\hertz}$ gemäß \autoref{eq:g} errechnet werden: 
\begin{equation*}
g_{\mathrm{Cu(AcAc)_2}} = \frac{\SI{6,6261e-34}{Js} \cdot \SI{9,42368e9} {\hertz} }{\SI{9,2741e-24}{\frac{J}{T}} \cdot \SI{317,2}{\milli\tesla}} = 2{,}1226
\end{equation*}


Nach \autoref{eq: äqKern} lässt sich nun der Kernspin wie folgt berechnen.

\begin{equation*}
N = \frac{N - 1}{2 \cdot n } = \frac{4-1}{2 \cdot 1} = \frac{3}{2}
\end{equation*}

Der Kernspin des Kupfers berechnet sich also zu $ N = \frac{3}{2}$. 




In \autoref{fig: Vo} ist das ESR-Spektrum der Vanadylacetylacetonat-Probe mit ebenfalls DPPH als internem Standard zu sehen. 

\begin{figure}[H]
\centering
\includegraphics[width = 1.0 \textwidth]{Cu_VO/VOgraph.pdf}
\caption{Das ESR-Spektum der Vanadylacetylacetonat-Probe mit internem DPPH Standard.}
\label{fig: Vo}
\end{figure}




\subsection{ESR-Spektren des Galvinoxylradikals}
%21_Galvin_O2_337B_5sweep_0p1mod_10db_5_10-1_4727720
% 1. Messung 337 B, Sweep 5, Modulation 0.1, 10 db, 5*10^1, 2*4727720 Strahlung
%23_Galvin_bisschen_O2_337B_5sweep_0p03mod_10db_5_10-1_4727720
% Alle anderen Messungen 337 B, Sweep 5, Modulation 0.03, 10 db, 5*10^1 2*4727720 Strahlung
\begin{figure}[H]
    \centering
    \includegraphics[width=1.0\textwidth]{Galvinoxyl/Galvinoxyl.pdf}
    \caption{ESR-Spektren der Galvinoxyl-Probe, wobei die erste mit einer Modulationsamplitude von 0,1 mT und alle anderen mit einer Modulationsamplitude von 0,03 mT aufgenommen wurden.} 
    \label{fig: Galvinoxyl}
\end{figure}

\section{Fehlerrechnung} % der berechneten g-Faktoren
Die Fehler der berechneten g-Faktoren werden durch Gaußsche Fehlerfortpflanzung ausgehend von \autoref{eq:g} mit den Messfehlern $\Delta \nu = \SI{0,021}{GHz}$ und $\Delta B_0 = \SI{0,005}{mT} $ und \autoref{eq:gfehler} bestimmt, wobei der Messfehler der Frequenz aus der Differenz der Frequenz bei der DPPH-Messung und der Frequenz der Galvinoxyl-Messung ermittelt und der Fehler des $B_0$-Werts abgeschätzt wurde.
\begin{align}
\Delta g &= \left| \frac{\partial g}{\partial \nu} \right| \cdot \Delta \nu
+ \left| \frac{\partial g}{\partial B_0} \right| \cdot \Delta B_0 \\ 
&= \left| \frac{h}{\mu_0 \cdot B_0} \right| \cdot \Delta \nu
+ \left| \frac{h \cdot \nu}{\mu_0 \cdot B_0^{2}} \right| \cdot \Delta B_0
\label{eq:gfehler}
\end{align}
Als Beispiel ist die Fehlerrechnung für den g-Faktor von DPPH mit \autoref{eq:gfehler} dargestellt.
\begin{align*}
\Delta g_\mathrm{DPPH} &= \left| \frac{h}{\mu_0 \cdot B_0} \right| \cdot \Delta \nu + \left| \frac{h \cdot \nu}{\mu_0 \cdot B_0^{2}} \right| \cdot \Delta B_0 \\
&= \left| \frac{\SI{6,6261e-34}{J s}}{ \SI{9,2741e-24}{\frac{J}{T}} \cdot \SI{336,656e-3}{T} } \right| \cdot \SI{0,022e9}{Hz} \\
&+ \left| \frac{\SI{6,6261e-34}{J s} \cdot \SI{9,43402e9}{Hz} }{ \SI{9,2741e-24}{\frac{J}{T}} \cdot (\SI{336,656e-3}{T})^{2} } \right| \cdot \SI{0,005e-3}{T} \\
&= 0{,}0045
\label{eq:gfehler}
\end{align*}

\begin{table}[H]
\centering
\caption{Berechnete Fehler der ermittelten g-Faktoren mit den zugehörigen Werten für $\nu$ und $B_0$ verschiedener Proben. \\}
\begin{tabular}{c|c|c|c}
Probe & $\nu$ / GHz & $B_0$ / mT & $\Delta g$ \\
\hline
DPPH     & 9,43402 & 336,656 & 0,0045 \\
Cu(AcAc)$_2$ + DPPH & 9,42368 & 317,2 & 0,0048 \\
VO(AcAc)$_2$ + DPPH  & 9,42368 & 338,2 &  0,0045 \\
Galvinoxyl           & 9,45544 & 336,25 & 0,0045
\end{tabular}
\label{tab: fehler}
\end{table}

\section{Zusammenfassung}
Im ersten Versuchsteil wurde der Einfluss der Mikrowellen-Abschwächung und der Modulationsamplitude auf das ESR-Spektrum der DPPH-Probe untersucht und der g-Faktor als $g_\mathrm{DPPH} = 2{,}0022 \pm 0{,}0045$ ermittelt.


$g_{\mathrm{Cu(AcAc)_2}} = 2{,}1226 \pm 0{,}0048$ \\
$g_{\mathrm{VO(AcAc)_2}} = 1{,}9908 \pm 0{,}0045$ \\
$g_{\mathrm{Galvinoxyl}} = 2{,}0091 \pm 0{,}0045$

\printbibliography[title={Literatur}]


\end{document}
