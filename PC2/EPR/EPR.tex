\documentclass[a4paper,12pt,bibliography=totocnumbered]{scrartcl}

\usepackage[utf8]{inputenc} 
\usepackage[T1]{fontenc}
\usepackage[ngerman]{babel} 
\usepackage{amsmath, amssymb,amsfonts}
\usepackage{graphicx}
\usepackage{csquotes}
\usepackage[bookmarks,colorlinks=true]{hyperref}
\usepackage{geometry}
\usepackage{float}
\usepackage[final]{pdfpages}
\usepackage{framed, color} 
\usepackage{scrlayer-scrpage}
\usepackage{siunitx}
\usepackage{subfigure}
\renewcaptionname{ngerman}{\figurename}{Abb.}
\sisetup{detect-weight=true, detect-family=true,locale=DE,range-phrase={\,bis\,},list-final-separator ={\,\linebreak[0] \text{und}\,},separate-uncertainty=true,per-mode = symbol-or-fraction}
\DeclareSIUnit\curie{Ci}
\usepackage[backend=biber, style=chem-angew]{biblatex} 
\addbibresource{lit.bib} 

\usepackage{chemgreek}
\usepackage{chemformula}
\geometry{left = 2.5cm} \geometry{top = 3cm}

\urlstyle{same}
%Hyperlinks-Setup
\hypersetup{
	colorlinks,
	linktocpage,
	citecolor=black,
	filecolor=black,
	linkcolor=black,
	urlcolor=black
}

%\numberwithin{equation}{section}

\setlength{\parindent}{0 mm}
\setlength{\parskip}{2 mm} 



\pagestyle{scrheadings}
%Header oben links auf linker Seite (ungerade Seitenzahl) und oben rechts auf rechter Seite (gerade Seitenzahl), beinhaltet gruppennummer und Versuchskürzel. Im Fall eine einseitigen Dokuments: Header oben rechts
\ihead{\VERSUCHSNR} %Header oben rechts auf linker Seite und oben links auf rechter Seite. Beinhaltet die Namen der Verfasser. Im Fall eine einseitigen Dokuments: Header oben links!
\ohead{\GRUPPENNR}
\ofoot{\thepage} 
\cfoot{\empty}  
\ifoot{\empty} 


\newcommand{\VERSUCHSDATUM}{15.12.2025}
\newcommand{\PROTOKOLLDATUM}{\today}

\newcommand{\VerfasserEINS}{Vincent Kümmerle}
\newcommand{\MatNoEINS}{3712667}
\newcommand{\EmailEINS}{st187541@stud.uni-stuttgart.de}
\newcommand{\StudiengangEINS}{B.Sc. Chemie}

\newcommand{\VerfasserZWEI}{Elvis Gnaglo}
\newcommand{\MatNoZWEI}{3710504}
\newcommand{\EmailZWEI}{st189318@stud.uni-stuttgart.de}
\newcommand{\StudiengangZWEI}{B.Sc. Chemie}

\newcommand{\VerfasserDREI}{Julian Brügger}
\newcommand{\MatNoDREI}{}
\newcommand{\EmailDREI}{st190010@stud.uni-stuttgart.de}
\newcommand{\StudiengangDREI}{B.Sc. Chemie}

\newcommand{\BETREUER}{Valentin Bayer}
\newcommand{\GRUPPENNR}{A05}

\newcommand{\VERSUCHSNR}{ESR}
\newcommand{\VERSUCHSNAME}{Elektronenspin-Resonanz-Spektroskopie}


\begin{document}
\thispagestyle{empty}


\begin{titlepage}

\begin{center}
\Huge{\textbf{\VERSUCHSNR\ - \VERSUCHSNAME}}\\
\vspace{10mm}% Abstand
\Large{Protokoll zum Versuch des PC 2 Praktikums von \\ \textbf{\VerfasserEINS\;\& \VerfasserZWEI\;\& \VerfasserDREI}}\\
\vspace{10mm} 
\Large{Universität Stuttgart}\\
\end{center}
\vspace{1cm}
\begin{center}
\begin{tabular}{ll}
\large{Autoren:}		& \large{\VerfasserEINS,} \large{\MatNoEINS} \\
 						& \large{\EmailEINS} \\
 						\vspace{0cm}\\
						& \large{\VerfasserZWEI,} \large{\MatNoZWEI} \\
                        & \large{\EmailZWEI} \\
						\vspace{0cm}\\
						& \large{\VerfasserDREI,} \large{\MatNoDREI} \\
                        & \large{\EmailDREI} \\
						\vspace{0cm}\\
\large{Gruppennummer:}	& \large{\GRUPPENNR} \\
\vspace{0cm}\\
\large{Versuchsdatum:}	& \large{\VERSUCHSDATUM} \\
\vspace{0cm}\\
\large{Betreuer:}		& \large{\BETREUER} \\
\vspace{0cm}\\
\large{Erstabgabedatum:} & \large{\PROTOKOLLDATUM}
\end{tabular}
\end{center}
\vspace{1cm}
\textbf{Abstract:}

\end{titlepage}


\thispagestyle{empty}

\tableofcontents 

\clearpage

\renewcommand{\thepage}{\arabic{page}}
\setcounter{page}{1}


\section{Theorie}


\section{Versuchsdurchführung}
Es wurden verschiedene Proben im ESR-Spektrometer untersucht, wobei jede Probe in den Resonator eingeführt wurde und nach Abgleichung der Mikrowellenbrücke und Einstellung der Parameter gemessen wurde.
Der Aufbau des Spektrometers ist in \autoref{fig: setup} dargestellt.

\begin{figure}[H]
    \centering
    \includegraphics[width=0.8\textwidth]{Bilder/setup.png}
    \caption{Schema des Versuchsaufbaus mit Mikrowellen-Generator, Zirkulator, Resonator und Diode.\supercite{Skript}} 
    \label{fig: setup}
\end{figure}

\subsection{2,2-Diphenyl-1-pikrylhydrazyl}
Von der 2,2-Diphenyl-1-pikrylhydrazyl(DPPH)-Probe wurde zuerst ein ESR-Spektrum bei der Standardeinstellung des Geräts mit folgenden Parametern aufgenommen: $B_0$: 338 mT, $Sweep$: 5,0 mT, $Sweep~time$: 60 s, $Modulation$: 0,01 mT, $MW attenuation$: 20,0 dB, $Gain$: $5 \cdot 10^0$, $Smooth$: 0,1 s. Dann wurden die Parameter $B_0-Field$, $Sweep$ und $Gain$ so lange verändert, bis das Spektrum zentriert, die Bildschirmbreite gut genutzt und die Spektrenhöhe optimiert war. Dies war bei den Parametern $B_0$: 337.91 mT, $Sweep$: 1.3 mT, $Gain$: $3 \cdot 10^1$ der Fall.
% 137.91 mT, Sweep: 1.3 mT, Gain:3*10^1
Dann wurde zuerst die Mikrowellenleistung mit jeder Messung durch Abschwächung der Mikrowellenstrahlung (MW attenuation) um 3 dB verdoppelt und anschließend die Modulationsamplitude von 0,01  auf 0,64 mT durch Verdopplung von Messung zu Messung variiert.
Die Parameter dieser Messungen sind mit den Parametern der restlichen Versuchsteile in \autoref{tab: parameter} aufgeführt.

\begin{table}[H]
\centering
\caption{Experimentelle Parameter der ESR Messungen verschiedener Proben. \\}
\begin{tabular}{c|c|c|c|c|c}
Probe & $B_0$-Field & Sweep & Modulation & MW attenuation & Gain \\
& / mT & / mT & / mT & / dB & - \\
\hline
DPPH - start        & 338 & 5   & 0,01        & 20     & $5\cdot10^{0}$ \\
DPPH - optimal        & 337,91 & 1,3   & 0,01        & 20     & $3\cdot10^{1}$ \\
DPPH - power var.   & 337,91 & 1,3   & 0,01       & 20 - 5    & $3\cdot10^{1}$ \\
DPPH - modul. var.    & 337,91 & 1,3   & 0,01 - 0,64    & 30     & $3\cdot10^{1}$ \\
\hline
Cu(AcAc)$_2$ + DPPH  & 320 & 60  & 0,4         & 20     & $5\cdot10^{1}$ \\
VO(AcAc)$_2$ + DPPH  & 340 & 100 & 0,2         & 30     & $5\cdot10^{1}$ \\
\hline
Galvinoxyl           & 337 & 5   & 0,1         & 10     & $5\cdot10^{1}$ \\
Galvinoxyl deox      & 337 & 5   & 0,03        & 10     & $5\cdot10^{1}$ \\
\end{tabular}
\label{tab: parameter}
\end{table}


\subsection{}


\subsection{Organische Radikale}


\section{Auswertung}

Für die Auswertung ist zu beachten, dass der Wert des $B_0$-Felds aufgrund der näheren Position der Hallsonde zum Magneten als der Position der Probe 1,258 mT höher ist als der reale Wert für $B_0$.
Somit ergeben sich für die $B_0$-Werte aus \autoref{tab: parameter} folgende reale $B_0$-Werte:
\begin{table}[H]
\centering
\caption{Reale $B_0$-Werte für die verschiedenen Messungen. \\}
\begin{tabular}{c|c}
Probe & $B_0$-Field \\
\hline
DPPH - start        & 336,742 \\
DPPH - optimal        & 336,652 \\
DPPH - power var.   & 336,652 \\
DPPH - modul. var.    & 336,652 \\
Cu(AcAc)$_2$ + DPPH  & 3 \\
VO(AcAc)$_2$ + DPPH  & 3  \\
Galvinoxyl           & 33 \\
Galvinoxyl deox & \\
\end{tabular}
\label{tab: zentren}
\end{table}

\subsection{ESR-Spektren von 2,2-Diphenyl-1-pikrylhydrazyl}
\autoref{fig: DPPH-start} zeigt die aufgenommenen ESR-Spektren der DPPH-Probe mit den Startparametern und optimalen Parametern aus \autoref{tab: parameter}.

\begin{figure}[H]
    \centering
    \includegraphics[width=1.0\textwidth]{DPPH/A)/01_DPPH_start_0p01mod_20db_5gain_471701.pdf}
    \caption{ESR-Spektrum der DPPH-Probe mit Standardparametern und optimalen Parametern.} 
    \label{fig: DPPH-start}
\end{figure}
Dabei wurde der Wert für $Gain$ versechsfacht, um die Spektrenhöhe zu vergrößern, $B_0$-Field leicht verringert für die Zentrierung des Spektrums und $Sweep$ auf 1,3 mT herabgesetzt, um nur den relevanten Bereich des ESR-Spektrums zu messen. \\
\autoref{fig: DPPH-power-var} zeigt die aufgenommenen ESR-Spektren der DPPH-Probe mit Werten der MW attenuation zwischen 20 dB und 5 dB. Dabei verdoppelt sich die Mikrowellenleistung von Messung zu Messung in 3 dB Schritten.

\begin{figure}[H]
    \centering
    \includegraphics[width=1.0\textwidth]{DPPH/B)/DPPH_MW_Attenuation.pdf}
    \caption{ESR-Spektren der DPPH-Probe mit variierender MW attenuation zwischen 20 dB und 5 dB.} 
    \label{fig: DPPH-power-var}
\end{figure}
Aus \autoref{fig: DPPH-power-var} geht hervor, dass die Spektrenhöhe mit höherer Mikrowellenleistung ansteigt, da das ESR-Spektrum bei geringeren Werten der MW attenuation durch eine geringere Abschwächung der Mikrowellen-Strahlung eine höhere Signalintensität aufweist. Zudem lässt sich beobachten, dass auch die Spektrenbreite mit zunehmender Mikrowellenleistung zunimmt.\\
\autoref{fig: DPPH-modul-var} zeigt die aufgenommenen ESR-Spektren der DPPH-Probe mit Werten der Modulationsamplitude zwischen 0,01 mT und 0,064 mT mit Verdopplung von Messung zu Messung.

\begin{figure}[H]
    \centering
    \includegraphics[width=1.0\textwidth]{DPPH/C)/DPPH_Modulation.pdf}
    \caption{ESR-Spektren der DPPH-Probe mit variierender Modulationsamplitude zwischen 0,01 mT und 0,064 mT.} 
    \label{fig: DPPH-modul-var}
\end{figure}
% Spektrenhöhe & Linienbreite
In \autoref{fig: DPPH-modul-var} lässt sich erkennen, dass die Spektrenhöhe mit zunehmender Modulationsamplitude ansteigt. Jedoch ist dieser Anstieg nicht linear, sondern flacht bei höheren Amplituden ab 0,32 mT ab. Die Linienbreite nimmt von der Modulationsamplitude 0,01 mT bis 0,64 mT zu.

Zur Bestimmung des g-Faktors von DPPH werden die Lagen aller Spektrenzentren der Messungen mit variierender MW-Abschwächung und Modulationsamplitude in \autoref{tab: zentren} aufgeführt und der Mittelwert berechnet. 

\begin{table}[H]
\centering
\caption{Lage der Spektrenzentren mit den zugehörigen Parametern.}
\begin{tabular}{c|c|c}
Modulation / mT & MW attenuation / dB & Spektrenzentrum / mT \\
\hline
0,01 & 20 & 336,647 \\
0,01 & 17 & 336,657 \\
0,01 & 14 & 336,662 \\
0,01 & 11 & 336,662 \\
0,01 & 8  & 336,667 \\
0,01 & 5  & 336,667 \\
0,01 & 30 & 336,647 \\
0,02 & 30 & 336,652 \\
0,04 & 30 & 336,662 \\
0,08 & 30 & 336,652 \\
0,16 & 30 & 336,652 \\
0,32 & 30 & 336,652 \\
0,64 & 30 & 336,647
\end{tabular}
\label{tab:zentren}
\end{table}
Die Lage des Spektrenzentrums von DPPH befindet sich also am Mittelwert von $B_0 = \SI{336,656}{mT}$.
Da als halbe Frequenz auf dem Display des Frequenzzählers 4,71701 GHz angezeigt wurde, wird für die Berechnung des g-Faktors die Frequenz $\nu = \SI{9,43402}{GHz}$ verwendet.
Durch Einsetzen des Bohrschen Magnetons $\mu_\mathrm{B}$ und des Planck'schen Wirkungsquantum zusammen mit dem Spektrenzentrum und der Frequenz in \autoref{eq: g}, die sich durch Umstellen der Gleichung der Resonanzbedingung \ref{eq:eres} nach g ergibt, wird der g-Faktor berechnet.
\begin{align}
	g &= \frac{h \cdot \nu}{\mu_\mathrm{B} \cdot B_0} \\
	g_\mathrm{DPPH}&= \frac{\SI{6.6261e-34}{Js} \cdot \SI{9,43402e9}{Hz}}{\SI{9,2741e-24}{\frac{J}{T}} \cdot \SI{336,656e-3}{T}} \nonumber \\
	&= 2{,}0022 \nonumber
	\label{eq: g}
\end{align}

Mit dem Literaturwert von DPPH von $g_{\mathrm{Lit}} = 2{,}0037$ ergibt sich eine absolute Abweichung von 0,0015.\supercite{Skript} Die Ungenauigkeit des g-Faktors wird in der Fehlerrechnung bestimmt.
Die Wellenlänge der verwendeten Mikrowelle lässt sich aus der Frequenz $\nu = \SI{9,43402}{GHz}$ mit der Lichtgeschwindigkeit berechnen.
\begin{align*}
	\lambda = \frac{c}{\nu} = \frac{\SI{2,99792}{\frac{m}{s}}}{\SI{9,43402e9}{Hz}} = \SI{0,0318}{m}
\end{align*}
Daraus lässt sich schließen, dass der Resonator 3,18 cm lang sein muss, um eine stehende Welle zu generieren, da dann die Wellenlänge der Mikrowelle und die Resonatorlänge ideal zueinander passen.

\subsection{ESR-Spektren von }


\subsection{ESR-Spektren des Galvinoxylradikals}
%21_Galvin_O2_337B_5sweep_0p1mod_10db_5_10-1_4727720
% 1. Messung 337 B, Sweep 5, Modulation 0.1, 10 db, 5*10^1, 2*4727720 Strahlung
%23_Galvin_bisschen_O2_337B_5sweep_0p03mod_10db_5_10-1_4727720
% Alle anderen Messungen 337 B, Sweep 5, Modulation 0.03, 10 db, 5*10^1 2*4727720 Strahlung
\begin{figure}[H]
    \centering
    \includegraphics[width=1.0\textwidth]{Galvinoxyl/Galvinoxyl.pdf}
    \caption{ESR-Spektren der Galvinoxyl-Probe wobei die Erste mit einer Modulationsamplitude von 0,1 mT und alle anderen mit einer Modulationsamplituden von 0,03 mT aufgenommen wurden.} 
    \label{fig: DPPH-modul-var}
\end{figure}

\section{Fehlerrechnung} % der berechneten g-Faktoren
Die Fehler der berechneten g-Faktoren werden durch Gaußsche Fehlerfortpflanzung ausgehend von \autoref{eq: g} mit den Messfehlern $\Delta \nu = \SI{0,022}{GHz}$ und $\Delta B_0 = \SI{0,005}{mT} $ und \autoref{eq:gfehler} bestimmt, wobei der Messfehler der Frequenz aus der Differenz der Frequenz bei der DPPH-Messung und der Frequenz der Galvinoxyl-Messung ermittelt und der Fehler des $B_0$-Werts abgeschätzt wurde.
\begin{align}
\Delta g &= \left| \frac{\partial g}{\partial \nu} \right| \cdot \Delta \nu
+ \left| \frac{\partial g}{\partial B_0} \right| \cdot \Delta B_0 \\ 
&= \left| \frac{h}{\mu_0 \cdot B_0} \right| \cdot \Delta \nu
+ \left| \frac{h \cdot \nu}{\mu_0 \cdot B_0^{2}} \right| \cdot \Delta B_0
\label{eq:gfehler}
\end{align}
Als Beispiel ist die Fehlerrechnung für den g-Faktor von DPPH mit \autoref{eq:gfehler} dargestellt.
\begin{align*}
\Delta g_\mathrm{DPPH} &= \left| \frac{h}{\mu_0 \cdot B_0} \right| \cdot \Delta \nu + \left| \frac{h \cdot \nu}{\mu_0 \cdot B_0^{2}} \right| \cdot \Delta B_0 \\
&= \left| \frac{\SI{6,6261e-34}{J s}}{ \SI{9,2741e-24}{\frac{J}{T}} \cdot \SI{336,656e-3}{T} } \right| \cdot \SI{0,022e9}{Hz} \\
&+ \left| \frac{\SI{6,6261e-34}{J s} \cdot \SI{9,43402e9}{Hz} }{ \SI{9,2741e-24}{\frac{J}{T}} \cdot (\SI{336,656e-3}{T})^{2} } \right| \cdot \SI{0,005e-3}{T} \\
&= 0{,}0047
\label{eq:gfehler}
\end{align*}

\begin{table}[H]
\centering
\caption{Berechnete Fehler der ermittelten g-Faktoren verschiedener Proben. \\}
\begin{tabular}{c|c}
Probe & $\Delta g$ \\
\hline
DPPH       & 0,0047 \\
Cu(AcAc)$_2$ + DPPH  &  \\
VO(AcAc)$_2$ + DPPH  &   \\
Galvinoxyl           & 
\end{tabular}
\label{tab: fehler}
\end{table}

\section{Zusammenfassung}


\printbibliography[title={Literatur}]


\end{document}
