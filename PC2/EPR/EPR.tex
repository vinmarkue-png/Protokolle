\documentclass[a4paper,12pt,bibliography=totocnumbered]{scrartcl}

\usepackage[utf8]{inputenc} 
\usepackage[T1]{fontenc}
\usepackage[ngerman]{babel} 
\usepackage{amsmath, amssymb,amsfonts}
\usepackage{graphicx}
\usepackage{csquotes}
\usepackage[bookmarks,colorlinks=true]{hyperref}
\usepackage{geometry}
\usepackage{float}
\usepackage[final]{pdfpages}
\usepackage{framed, color} 
\usepackage{scrlayer-scrpage}
\usepackage{siunitx}
\usepackage{subfigure}
\renewcaptionname{ngerman}{\figurename}{Abb.}
\sisetup{detect-weight=true, detect-family=true,locale=UK,range-phrase={\.bis\.},list-final-separator ={\,\linebreak[0] \text{und}\,},separate-uncertainty=true,per-mode = symbol-or-fraction}
\DeclareSIUnit\curie{Ci}
\usepackage[backend=biber, style=chem-angew]{biblatex} 
\addbibresource{lit.bib} 

\usepackage{chemgreek}
\usepackage{chemformula}
\geometry{left = 2.5cm} \geometry{top = 3cm}

\urlstyle{same}
%Hyperlinks-Setup
\hypersetup{
	colorlinks,
	linktocpage,
	citecolor=black,
	filecolor=black,
	linkcolor=black,
	urlcolor=black
}

%\numberwithin{equation}{section}

\setlength{\parindent}{0 mm}
\setlength{\parskip}{2 mm} 



\pagestyle{scrheadings}
%Header oben links auf linker Seite (ungerade Seitenzahl) und oben rechts auf rechter Seite (gerade Seitenzahl), beinhaltet gruppennummer und Versuchskürzel. Im Fall eine einseitigen Dokuments: Header oben rechts
\ihead{\VERSUCHSNR} %Header oben rechts auf linker Seite und oben links auf rechter Seite. Beinhaltet die Namen der Verfasser. Im Fall eine einseitigen Dokuments: Header oben links!
\ohead{\GRUPPENNR}
\ofoot{\thepage} 
\cfoot{\empty}  
\ifoot{\empty} 


\newcommand{\VERSUCHSDATUM}{15.12.2025}
\newcommand{\PROTOKOLLDATUM}{\today}

\newcommand{\VerfasserEINS}{Vincent Kümmerle}
\newcommand{\MatNoEINS}{3712667}
\newcommand{\EmailEINS}{st187541@stud.uni-stuttgart.de}
\newcommand{\StudiengangEINS}{B.Sc. Chemie}

\newcommand{\VerfasserZWEI}{Elvis Gnaglo}
\newcommand{\MatNoZWEI}{3710504}
\newcommand{\EmailZWEI}{st189318@stud.uni-stuttgart.de}
\newcommand{\StudiengangZWEI}{B.Sc. Chemie}

\newcommand{\VerfasserDREI}{Julian Brügger}
\newcommand{\MatNoDREI}{}
\newcommand{\EmailDREI}{st190010@stud.uni-stuttgart.de}
\newcommand{\StudiengangDREI}{B.Sc. Chemie}

\newcommand{\BETREUER}{Valentin Bayer}
\newcommand{\GRUPPENNR}{A05}

\newcommand{\VERSUCHSNR}{ESR}
\newcommand{\VERSUCHSNAME}{Elektronen Spin Resonanz Spektroskopie}


\begin{document}
\thispagestyle{empty}


\begin{titlepage}

\begin{center}
\Huge{\textbf{\VERSUCHSNR\ - \VERSUCHSNAME}}\\
\vspace{10mm}% Abstand
\Large{Protokoll zum Versuch des PC 2 Praktikums von \\ \textbf{\VerfasserEINS\;\& \VerfasserZWEI\;\& \VerfasserDREI}}\\
\vspace{10mm} 
\Large{Universität Stuttgart}\\
\end{center}
\vspace{1cm}
\begin{center}
\begin{tabular}{ll}
\large{Autoren:}		& \large{\VerfasserEINS,} \large{\MatNoEINS} \\
 						& \large{\EmailEINS} \\
 						\vspace{0cm}\\
						& \large{\VerfasserZWEI,} \large{\MatNoZWEI} \\
                        & \large{\EmailZWEI} \\
						\vspace{0cm}\\
						& \large{\VerfasserDREI,} \large{\MatNoDREI} \\
                        & \large{\EmailDREI} \\
						\vspace{0cm}\\
\large{Gruppennummer:}	& \large{\GRUPPENNR} \\
\vspace{0cm}\\
\large{Versuchsdatum:}	& \large{\VERSUCHSDATUM} \\
\vspace{0cm}\\
\large{Betreuer:}		& \large{\BETREUER} \\
\vspace{0cm}\\
\large{Erstabgabedatum:} & \large{\PROTOKOLLDATUM}
\end{tabular}
\end{center}
\vspace{1cm}
\textbf{Abstract:}

\end{titlepage}


\thispagestyle{empty}

\tableofcontents 

\clearpage

\renewcommand{\thepage}{\arabic{page}}
\setcounter{page}{1}


\section{Theorie}


\section{Versuchsdurchführung}
Es wurden verschiedene Proben im ESR-Spektrometer untersucht, wobei jede Probe in den Resonator eingeführt wurde und nach Abgleichung der Mikrowellenbrücke und Einstellung der Parameter gemessen wurde.

\subsection{DPPH}
Von der Diphenylpikrylhydrazyl-Probe wurde zuerst ein ESR-Spektrum bei der Standardeinstellung des Geräts mit folgenden Parametern aufgenommen: $B_0$: 338 mT, $Sweep$: 5,0 mT, $Sweep~time$: 60 s, $Modulation$: 0,01 mT, $MW attenuation$: 20,0 dB, $Gain$: $5 \cdot 10^0$, $Smooth$: 0,1 s. Dann wurden die Parameter $B_0-Field$, $Sweep$ und $Gain$ so lange verändert, bis das Spektrum zentriert, die Bildschirmbreite gut genutzt und die Spektrenhöhe optimiert war. Dies war bei den Parametern $B_0$: 337.91 mT, $Sweep$: 1.3 mT, $Gain$: $3 \cdot 10^1$ der Fall.
% 137.91 mT, Sweep: 1.3 mT, Gain:3*10^1

\subsection{}


\subsection{}


\subsection{Organische Radikale}


\section{Auswertung}

\subsection{DPPH}


\subsection{}


\subsection{}


\subsection{Organische Radikale}
%21_Galvin_O2_337B_5sweep_0p1mod_10db_5_10-1_4727720
% 1. Messung 337 B, Sweep 5, Modulation 0.1, 10 db, 5*10^1, 2*4727720 Strahlung
%23_Galvin_bisschen_O2_337B_5sweep_0p03mod_10db_5_10-1_4727720
% Alle anderen Messungen 337 B, Sweep 5, Modulation 0.03, 10 db, 5*10^1 2*4727720 Strahlung

\section{Fehlerrechnung}


\section{Zusammenfassung}


\printbibliography[title={Literatur}]


\end{document}
