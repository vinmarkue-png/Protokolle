\documentclass[a4paper,12pt,bibliography=totocnumbered]{scrartcl}

\usepackage[utf8]{inputenc} 
\usepackage[T1]{fontenc}
\usepackage[english]{babel}
\usepackage{amsmath, amssymb,amsfonts}
\usepackage{graphicx}
\usepackage{csquotes}
\usepackage[bookmarks,colorlinks=true]{hyperref}
\usepackage{geometry}
\usepackage{float}
\usepackage[final]{pdfpages}
\usepackage{framed, color} 
\usepackage{scrlayer-scrpage}
\usepackage{siunitx}
\usepackage{subcaption}
\renewcaptionname{english}{\figurename}{Fig.}
\renewcaptionname{english}{\tablename}{Tab.}
\sisetup{
    detect-weight=true, 
    detect-family=true,
    locale=UK,
    exponent-product = \cdot,
    range-phrase={\,bis\,},
    list-final-separator ={\,\linebreak[0] \text{and}\,},
    separate-uncertainty=true,
    per-mode = symbol-or-fraction
}
%macht komata anstatt kreuze bei Zehnerpotenzen

\DeclareSIUnit{\angstrom}{\text{\AA}}
\usepackage[backend=biber, style=chem-angew]{biblatex} 
\addbibresource{lit.bib} 

\usepackage{chemgreek}
\usepackage{chemformula}
\geometry{left = 2.5cm} \geometry{top = 3cm}

\urlstyle{same}
%Hyperlinks-Setup
\hypersetup{
	colorlinks,
	linktocpage,
	citecolor=black,
	filecolor=black,
	linkcolor=black,
	urlcolor=black
}

%\numberwithin{equation}{section}

\setlength{\parindent}{0 mm}
\setlength{\parskip}{2 mm} 



\pagestyle{scrheadings}
%Header oben links auf linker Seite (ungerade Seitenzahl) und oben rechts auf rechter Seite (gerade Seitenzahl), beinhaltet gruppennummer und Versuchskürzel. Im Fall eine einseitigen Dokuments: Header oben rechts
\ihead{\VERSUCHSNR} %Header oben rechts auf linker Seite und oben links auf rechter Seite. Beinhaltet die Namen der Verfasser. Im Fall eine einseitigen Dokuments: Header oben links!
\ohead{\GRUPPENNR}
\ofoot{\thepage} 
\cfoot{\empty}  
\ifoot{\empty} 


\newcommand{\VERSUCHSDATUM}{28.01.2026}
\newcommand{\PROTOKOLLDATUM}{\today}

\newcommand{\VerfasserEINS}{Vincent Kümmerle}
\newcommand{\MatNoEINS}{3712667}
\newcommand{\EmailEINS}{st187541@stud.uni-stuttgart.de}
\newcommand{\StudiengangEINS}{B.Sc. Chemie}

\newcommand{\VerfasserZWEI}{Elvis Gnaglo}
\newcommand{\MatNoZWEI}{3710504}
\newcommand{\EmailZWEI}{st189318@stud.uni-stuttgart.de}
\newcommand{\StudiengangZWEI}{B.Sc. Chemie}

\newcommand{\VerfasserDREI}{Julian Brügger}
\newcommand{\MatNoDREI}{3715444}
\newcommand{\EmailDREI}{st190050@stud.uni-stuttgart.de}
\newcommand{\StudiengangDREI}{B.Sc. Chemie}

\newcommand{\BETREUER}{Mansha Shafquath}
\newcommand{\GRUPPENNR}{A05}

\newcommand{\VERSUCHSNR}{Opt}
\newcommand{\VERSUCHSNAME}{Optical Spectroscopy}


\begin{document}
\thispagestyle{empty}


\begin{titlepage}

\begin{center}
\Huge{\textbf{\VERSUCHSNAME}}\\
\vspace{10mm}% Abstand
\Large{Protocol for the PC 2 lab course by \\ \textbf{\VerfasserEINS\;\& \VerfasserZWEI\;\& \VerfasserDREI}}\\
\vspace{10mm} 
\Large{University of Stuttgart}\\
\end{center}
\vspace{0cm}
\begin{center}
\begin{tabular}{ll}
\large{authors:}		& \large{\VerfasserEINS,} \large{\MatNoEINS} \\
 						& \large{\EmailEINS} \\
						\vspace{0cm}\\
						& \large{\VerfasserZWEI,} \large{\MatNoZWEI} \\
                        & \large{\EmailZWEI} \\
						\vspace{0cm}\\
						& \large{\VerfasserDREI,} \large{\MatNoDREI} \\
                        & \large{\EmailDREI} \\
						\vspace{0cm}\\
\large{group number:}	& \large{\GRUPPENNR} \\
\vspace{0cm}\\
\large{date of experiment:}	& \large{\VERSUCHSDATUM} \\
\vspace{0cm}\\
\large{supervisor:}		& \large{\BETREUER} \\
\vspace{0cm}\\
\large{submission date:} & \large{\PROTOKOLLDATUM}
\end{tabular}
\end{center}

\vspace{1cm}


\textbf{Abstract:}
In this experiment, 
\end{titlepage}


\thispagestyle{empty}

\tableofcontents 

\clearpage

\renewcommand{\thepage}{\arabic{page}}
\setcounter{page}{1}


\section{Theory}
\subsection{Optical Spectroscopy}
Optical spectroscopy studies the interactions between matter and light of the visible and ultraviolet spectrum. This is done by analyzing the characteristic absorption and transmission spectra of different materials, which enables a qualitative and quantitative measurement of that material. 

\subsection{Wave interference at gratings}
Optical gratings are often used in optical spectroscopy because they, like the classical double slit experiment, create characteristic interference patterns. But because of the higher number of slits, the gratings allow for a more complex pattern, which in turn results in better measurements. The interference patterns are created when light in the form of a linear wavefront passes trough the grating. Each slit acts as the starting point of a new spherical wave that interferes with the other spherical waves. If the path width between those waves is an integer multiple the waves interfere constructively, which leads to bright spots. If the path width is a half-integer multiple they interfere destructively, which leads to dark spots. An illustration of a grating is shown in \autoref{fig: interference}.

\begin{figure}[H]
    \centering
    \includegraphics[width=0.75\textwidth]{grate.png}
    \caption{A grating with the distance $d$ between the slits, the deviation angle $\theta$ of the light and the path width $\Delta S$.\supercite{Skript}} 
    \label{fig: interference}
\end{figure}

The deviation angle of the constructive interference can be calculated by using \autoref{eq: theta}.

\begin{equation}
    \sin{\theta_k} = \frac{k ~\cdot~ \lambda}{d}
    \label{eq: theta}
\end{equation}

Where $k$ is the order of diffraction, $\lambda$ is the wavelength and $d$ is the distance between the slits.



\subsection{The spectrometer}

\autoref{fig: setup} shows the setup of the DIY spectrometer used in the experiment.

\begin{figure}[H]
    \centering
    \includegraphics[width=0.85\textwidth]{setup.png}
    \caption{Scheme of the different components, that make up the spectrometer used in the experiment.\supercite{Skript}} 
    \label{fig: setup}
\end{figure}

The figure shows, that the light from a LED passes trough a condenser and iris diaphragm, which collimate the light and regulate the intensity of the light used in the measurements. Afterwards it passes trough two lenses, where the first lens refracts it back into parallel beams and the second focuses it again. The focused light then passes the entrance slit, as well as the sample, which is positioned directly behind the slit. After that it gets reflected by a mirror onto the rotating reflexion grating, where it gets reflected to another mirror. After that it passes trough the exit slit and another lens to be focused onto the detector.

\subsection{Grating Equation for a Rotatable Grating}

The reflexion grating used in the experiment has to be rotated in order to measure the sample at the different wavelengths. This is achieved by using a stepper motor, where each step rotates the grating by $1.8^{\circ}$. Each step also consists of 16 micro steps $z$ and have to factor in the gearbox of the motor which has a ratio of $\frac{1}{20}$. Using those factors, it is possible to calculate the degree of rotation $\alpha$, as is shown in \autoref{eq: alpha1}.

\begin{equation}
    \alpha = \frac{1.8 ~ \cdot \pi}{16 ~ \cdot ~ 20 ~ \cdot ~ 180} ~ \cdot z
    \label{eq: alpha1}
\end{equation}

To account for possible measurement errors the auxiliary values $f$ and $\Delta \alpha$ are introduced into the equation, which leads to \autoref{eq: alpha2}.

\begin{equation}
	\alpha = \frac{1.8 ~ \cdot \pi}{16 ~ \cdot ~ 20 ~ \cdot ~ 180} ~ \cdot z ~ \cdot ~ f+ \Delta \alpha
\label{eq: alpha2}
\end{equation}
Where $f$ is a scaling factor and $\Delta \alpha$ accounts for the fact that the original angle of the grating is not known.\\
The rotation angle can now be used to calculate the outgoing wavelength of the light for every possible angle. This is shown in \autoref{eq: klambda}.

\begin{equation}
    k ~ \cdot ~ \lambda = \sin{(\theta ~ + ~ \alpha)} ~ + ~ \sin{(\alpha)}
    \label{eq: klambda}
\end{equation}

\section{Procedure}
The DIY spectrometer was already built prior to the start of the experiment as shown in \autoref{fig: setup} and was set up so that the zeroth diffraction order was centered on the exit slit. First, the reference spectrum of the lamp was recorded with the Python script \texttt{Spektrometersoftware.py}, using a step count of 6000 and a stepping speed of 1000 steps per second.
Then the bandpass filter was inserted between the entrance slit and the concave mirror and the spectrum of the filtered light was recorded. For calibration, the transmission spectrum was compared to theoretical transmission data using the Python script \texttt{calibration.py} and the parameters scale, $f$ and $\Delta \alpha$ were adjusted according to \autoref{eq: alpha2} to fit the measured spectrum to the theoretical spectrum. For all subsequent measurements, the calibration parameters scale $= 1.0$, $f = 0.925$ and $\Delta \alpha = 0.59$ were used. \\
Next, a cuvette was filled with distilled water and placed between exit slit and concave mirror. The spectrum was recorded and used as a reference for the potassium permanganate and chlorophyll solutions. After replacing the water cuvette with a cuvette filled with concentrated and then diluted potassium permanganate solution, their spectrum were recorded separately.
In the next part, a single drop of chlorophyll solution was added to the water cuvette and the spectrum was recorded. Then, another drop was added and the spectrum recorded again. This was repeated until a total of five drops were added to the cuvette.
In the last part of the experiment, both slits were closed first, then the entrance slit was opened by half a turn and the exit slit by 6 turns. Then the background spectra of pure nitric acid and the solution of holmium oxide in nitric acid were recorded separately. This was repeated with the exit slit opened by 3 turns and one and a half turn, respectively. 



\section{Results and Analysis}

\subsection{Lamp and filter spectra}



\subsection{Potassium permanganate spectra}



\subsection{Chlorophyll spectra}




\newpage
\subsection{Holmium oxide spectra}
The absorption spectra of Holmium oxide in nitric acid for different exit slit widths are generated from the measured transmission spectra analog to the spectra of \ch{KMnO4} und Chlorophyll and are plotted together in \autoref{fig: Holm}.
\begin{figure}[H]
    \centering
    \includegraphics[width=1.0\textwidth]{Holm/Holm_spektrum.pdf}
    \caption{Absorption spectrum of Holmium oxide in nitric acid for different exit slit widths between 400 and 800 nm.} 
    \label{fig: Holm}
\end{figure}
All three spectra show three distinct absorption peaks around 460 nm, 545 nm and 650 nm. With decreasing slit width, the absorption peaks become narrower and more pronounced. 
The absorption spectrum of Holmium oxide with an exit slit of 1.5 turns shows additional oscillations, which can be attributed to noise due to the lower light intensity reaching the detector at this slit width. \\
To analyze the absorption peaks further, a four-fold Lorentzian model is fitted to each of the three spectra. The fitted curves are shown together with the measured spectra in \autoref{fig: lorentz_6}, \autoref{fig: lorentz_3} and \autoref{fig: lorentz_1.5}.

\begin{figure}[H]
    \centering
    \includegraphics[width=0.8\textwidth]{Holm/lorentz_6.pdf}
    \caption{Absorption signals of Holmium oxide in nitric acid with a four-fold Lorentzian model fit for 6 turns between 400 and 700 nm.} 
    \label{fig: lorentz_6}
\end{figure}


\begin{figure}[H]
    \centering
    \includegraphics[width=0.8\textwidth]{Holm/lorentz_3.pdf}
    \caption{Absorption signals of Holmium oxide in nitric acid with a four-fold Lorentzian model fit for 3 turns between 400 and 700 nm.} 
    \label{fig: lorentz_3}
\end{figure}


\begin{figure}[H]
    \centering
    \includegraphics[width=0.8\textwidth]{Holm/lorentz_1.5.pdf}
    \caption{Absorption signals of Holmium oxide in nitric acid with a four-fold Lorentzian model fit for 1.5 turns between 400 and 700 nm.} 
    \label{fig: lorentz_1.5}
\end{figure}

% % 6 turns:
% pos1:   460.445002 +/- 0.21344301 (0.05%) (init = 465)
%     amp1:   0.38713089 +/- 0.02473514 (6.39%) (init = 1)
%     gam1:   8.11726514 +/- 0.66354365 (8.17%) (init = 10)
%     pos2:   547.541296 +/- 0.21466242 (0.04%) (init = 543)
%     amp2:   0.49427792 +/- 0.02538574 (5.14%) (init = 1)
%     gam2:   9.66980766 +/- 0.65457677 (6.77%) (init = 10)
%     pos3:   651.812625 +/- 0.29605846 (0.05%) (init = 655)
%     amp3:   0.45454466 +/- 0.03233743 (7.11%) (init = 1)
%     gam3:   11.4035363 +/- 0.98141707 (8.61%) (init = 10)
% % 3 turns:
% pos1:   459.901005 +/- 0.21431692 (0.05%) (init = 460)
%     amp1:   0.24627937 +/- 0.02482315 (10.08%) (init = 1)
%     gam1:   4.78939399 +/- 0.65004725 (13.57%) (init = 10)
%     pos2:   546.775708 +/- 0.21138035 (0.04%) (init = 546)
%     amp2:   0.42467135 +/- 0.02894564 (6.82%) (init = 1)
%     gam2:   6.81844220 +/- 0.62296199 (9.14%) (init = 10)
%     pos3:   651.818526 +/- 0.38812702 (0.06%) (init = 650)
%     amp3:   0.54989441 +/- 0.04850025 (8.82%) (init = 1)
%     gam3:   12.2692980 +/- 1.29884269 (10.59%) (init = 10)
% % 1.5 turns:
%     pos1:   459.797356 +/- 0.45040935 (0.10%) (init = 461)
%     amp1:   0.31189578 +/- 0.05468159 (17.53%) (init = 1)
%     gam1:   5.94686605 +/- 1.38356683 (23.27%) (init = 10)
%     pos2:   546.539203 +/- 0.34363002 (0.06%) (init = 544)
%     amp2:   0.40841212 +/- 0.05122455 (12.54%) (init = 1)
%     gam2:   5.92466118 +/- 0.99773533 (16.84%) (init = 10)
%     pos3:   651.638623 +/- 0.66444947 (0.10%) (init = 650)
%     amp3:   0.51584446 +/- 0.08421779 (16.33%) (init = 1)
%     gam3:   10.8758799 +/- 2.17152483 (19.97%) (init = 10)
The extracted fit parameters, namely the position, half width at half maximum (HWHM, gam) and the peak amplitude of the three main absorption peaks for 6 turns in \autoref{fig: lorentz_6} are summarized in \autoref{tab: 6}.
\begin{table}[H]
	\centering
	\caption{Fit parameters of the Lorentzian model for the absorption spectrum of Holmium oxide in nitric acid with an exit slit width of 6 turns.}
	\begin{tabular}{c|c|c}
		pos / nm & gam / nm & amp / a.u. \\
		\hline
		460.45 & 8.12 & 0.39 \\
		547.54 & 9.67 & 0.49 \\
		651.81 & 11.40 & ~0.46
		\label{tab: 6}
	\end{tabular}
\end{table}
The corresponding fit parameters for the fits in \autoref{fig: lorentz_3} and \autoref{fig: lorentz_1.5} are summarized in \autoref{tab: 3} and \autoref{tab: 1.5}.
\begin{table}[H]
	\centering
	\caption{Fit parameters of the Lorentzian model for the absorption spectrum of Holmium oxide in nitric acid with an exit slit width of 3 turns.}
	\begin{tabular}{c|c|c}
		pos / nm & gam / nm & amp / a.u. \\
		\hline
		460.45 & 4.79 & 0.25 \\
		547.54 & 6.82 & 0.43 \\
		651.81 & 12.27 & ~0.55
		\label{tab: 3}
	\end{tabular}
\end{table}
\begin{table}[H]
	\centering
	\caption{Fit parameters of the Lorentzian model for the absorption spectrum of Holmium oxide in nitric acid with an exit slit width of 1.5 turns.}
	\begin{tabular}{c|c|c}
		pos / nm & gam / nm & amp / a.u. \\
		\hline
		460.45 & 5.95 & 0.31 \\
		547.54 & 5.92 & 0.41 \\
		651.81 & 10.88 & ~0.52
		\label{tab: 1.5}
	\end{tabular}
\end{table}
What can be observed from the fit parameters is that with decreasing slit width, the HWHM of the two absorption peaks at 460.5 nm and 547.5 nm decreases, indicating a higher resolution of the spectrometer at smaller slit widths. 
However, the peak amplitudes do not show a clear trend with changing slit width. Only the peak amplitude of the absorption peak at 547.5 nm decreases with decreasing slit width, matching the expectation that less light reaches the detector at smaller slit widths. The peak amplitude of the absorption peak at 460.5 nm first decreases from 6 turns to 3 turns, but then increases again for 1.5 turns and for the peak at 651.8 nm, the peak amplitude first increases from 6 turns to 3 turns and then decreases again for 1.5 turns. A possible explanation for this is the manual adjustments of the slit widths, which can lead to slight misalignments of the spectrometer and thus affect the measured intensities. \\
The line width is characterized by the full width at half maximum (FWHM) of the fitted Lorentzian peaks, which is 2 times the HWHM, expressed in the fit parameter 'gamma' and can be calculated for every absorption peak and slit width from \autoref{tab: 6}, \autoref{tab: 3} and \autoref{tab: 1.5} using the equation:
\begin{equation}
	\mathrm{FWHM} = 2 \cdot \mathrm{gam}
\end{equation}
The calculated FWHM values for the three main absorption peaks at different slit widths are listed with the corresponding peak amplitudes in \autoref{tab: FWHM}.
\begin{table}[H]
	\centering
	\caption{Line widths (FWHM) and peak amplitudes of the Lorentzian model fits for different slit widths.}
	\begin{tabular}{c|cc|cc|cc}
		& \multicolumn{2}{c|}{6-turns} & \multicolumn{2}{c|}{3-turns} & \multicolumn{2}{c}{1.5-turns} \\
		\hline
		pos / nm & FWHM/nm & amp/a.u. & FWHM/nm & amp/a.u. & FWHM/nm & amp/a.u. \\
		\hline
		460.45 & 16.24 & 0.39 & 9.58 & 0.25 & 11.90 & 0.39 \\
		547.54 & 19.34 & 0.49 & 13.64 & 0.43 & 11.84 & 0.49 \\
		651.81 & 22.80 & 0.46 & 24.54 & 0.55 & 21.76 & ~0.52 
		\label{tab: FWHM}
	\end{tabular}
\end{table}

The resolution of the spectra for the different slit widths is calculated by inserting the full width at half maximum (FWHM) with the extracted position of the absorption peak into \autoref{eq: R}.
\begin{equation}
	R = \frac{\lambda}{\Delta \lambda} = \mathrm{\frac{pos}{FWHM}}
	\label{eq: R2}
\end{equation}
The calculation for the resolution of the spectrometer at a slit width of 6 turns yields:
\begin{equation*}
	R_{\mathrm{6-turns}} = \frac{1}{3} \cdot \left(\frac{460.45\;\mathrm{nm}}{16.24\;\mathrm{nm}} + \frac{547.54\;\mathrm{nm}}{19.34\;\mathrm{nm}} + \frac{651.81\;\mathrm{nm}}{22.80\;\mathrm{nm}}\right) = 26.6
\end{equation*}
By calculating the resolution for the slit widths of 3 turns and 1.5 turns in the same way with the mean value of the resolution for each peak from \autoref{eq: R2}, the resolution $R_{\mathrm{3-turns}} = 38.3$ and $R_{\mathrm{1.5-turns}} = 38.3$ are determined. The calculation matches the expectation that the resolution of the spectrometer increases with decreasing slit width, as less light reaches the detector and thus a more precise measurement of the absorption peaks is possible. But this trend is not observed for the reduction of the slit widths from 3 turns to 1.5 turns, indicating that a further decrease in slit width to 0.5 turns as described in the script \supercite{Skript} could have led to an even higher spectrometer resolution. 

\section{Conclusion}
In this experiment, 

\newpage

\printbibliography[title={References}]

\end{document}
